\documentclass[a4paper,11pt,exos]{nsi} % COMPILE WITH DRAFT


\pagestyle{empty}
\begin{document}


\classe{\premiere spé}
\titre{Ceinture orange 05 - Corrigé}
\maketitle


%Exercice 2N52-1 pour la question 1
%Exercice 2N52-4 pour les deux questions suivantes


\begin{exercice}[ : Équations se ramenant au produit-nul]
    Résoudre les équations suivantes :
    %\begin{multicols}{3}
        \begin{enumerate}
            \item $(-3x+6)(-9x+9)=0$
            \item $(8x-9)^{2}+(8x-9)(-7x+7)=0$ 
	        \item $(-7x-7)(7x-5)=(-7x-7)(-2x+9)$
        \end{enumerate}
    %\end{multicols}
    
\end{exercice}

\begin{enumerate}
    \item On reconnaît une équation produit-nul, donc on applique la propriété :\\
    {\color[HTML]{f15929}Un produit est nul si et seulement si au moins un de ses facteurs est nul.}\\$(-3x+6)(-9x+9)=0$\\$\iff -3x+6=0\quad$ ou $\quad-9x+9=0$\\$\iff -3x=-6\quad$ ou $\quad -9x=-9$\\$\iff x=\dfrac{-6}{-3}\quad$ ou $\quad x=\dfrac{-9}{-9}$\\$\iff x=2\quad$ ou $\quad x=1$\\On en déduit :  $S=\left\{1;2\right\}$

    \item  $(8x-9)^{2}+(8x-9)(-7x+7)=0$\\ $\iff (\underline{8x-9})(8x-9)+(\underline{8x-9)}( -7x+7)=0$\\ $\iff (\underline{8x-9})\Big((8x-9)+(-7x+7)\Big)=0$\\ $\iff (8x-9)( 8x-9-7x+7)=0$\\ $\iff (8x-9)( x-2)=0$\\$\iff 8x-9=0\quad$ ou $\quad x-2=0$\\$\iff x=\dfrac{9}{8}\quad$ ou $\quad x=2$\\
    On en déduit :  $S=\left\{\dfrac{9}{8};2\right\}$

    \item Deux nombres sont égaux si et seulement si leur différence est nulle.\\$\phantom{\iff}(-7x-7)(7x-5)=(-7x-7)(-2x+9)$\\$\iff (\underline{-7x-7})(7x-5)-(\underline{-7x-7})(-2x+9)=0$\\$\iff (\underline{-7x-7})\Big((7x-5)-(-2x+9)\Big)=0$\\$\iff (-7x-7)(7x-5+2x-9)=0$\\$\iff (-7x-7)(9x-14)=0$\\$(-7x-7)(9x-14)=0$\\$\iff -7x-7=0\quad$ ou $\quad 9x-14=0$\\$\iff -7x=7\quad$ ou $\quad 9x=14$\\$\iff x=\dfrac{7}{-7}\quad$ ou $\quad x=\dfrac{14}{9}$\\On en déduit :  $S=\left\{-1;\dfrac{14}{9}\right\}$
\end{enumerate}


\end{document}