\documentclass[a4paper,11pt,exos]{nsi} % COMPILE WITH DRAFT


\pagestyle{empty}
\begin{document}


\classe{\premiere spé}
\titre{Ceinture orange 04 - Corrigé}
\maketitle


%Exercice 2N52-1 pour la question 1
%Exercice 2N52-4 pour les deux questions suivantes


\begin{exercice}[ : Équations se ramenant au produit-nul]
    Résoudre les équations suivantes :
    %\begin{multicols}{3}
        \begin{enumerate}
            \item $(-2x-2)(2x+9)=0$
            \item $(5x+7)^{2}+(5x+7)(-4x+5)=0$ 
	        \item $(4x+6)(9x-6)=(4x+6)(-6x+8)$
        \end{enumerate}
    %\end{multicols}
    
\end{exercice}

\begin{enumerate}
    \item On reconnaît une équation produit-nul, donc on applique la propriété :\\
                    {\color[HTML]{f15929}Un produit est nul si et seulement si au moins un de ses facteurs est nul.}\\$(-2x-2)(2x+9)=0$\\$\iff -2x-2=0$ ou $2x+9=0$\\$\iff -2x=2$ ou $ 2x=-9$\\$\iff x=\dfrac{2}{-2}$ ou $ x=\dfrac{-9}{2}$\\On en déduit :  $S=\left\{-\dfrac{9}{2};-1\right\}$
    \item  $(5x+7)^{2}+(5x+7)(-4x+5)=0$\\ $\iff (5x+7)(5x+7)+(5x+7)(-4x+5)=0$\\ On observe que $(5x+7)$ est un facteur commun dans les deux termes :\\ $\phantom{\iff} (\underline{5x+7})(5x+7)+(\underline{5x+7)}( -4x+5)=0$\\ $\iff (\underline{5x+7})\Big((5x+7)+(-4x+5)\Big)=0$\\ $\iff (5x+7)( 5x+7)-4x+5)=0$\\ $\iff (5x+7)( x+12)=0$\\$\iff 5x+7=0\quad$ ou $\quad x+12=0$\\$\iff x=-\dfrac{7}{5}\quad$ ou $\quad x=-12$\\
                    On en déduit :  $S=\left\{-12;-\dfrac{7}{5}\right\}$
     
     \item Deux nombres sont égaux si et seulement si leur différence est nulle.\\$\phantom{\iff}(4x+6)(9x-6)=(4x+6)(-6x+8)$\\$\iff (\underline{4x+6})(9x-6)-(\underline{4x+6})(-6x+8)=0$\\$\iff (\underline{4x+6})\Big((9x-6)-(-6x+8)\Big)=0$\\$\iff (4x+6)(9x-6+6x-8)=0$\\$\iff (4x+6)(15x-14)=0$\\$\iff 4x+6=0$ ou $15x-14=0$\\$\iff 4x=-6$ ou $ 15x=14$\\$\iff x=\dfrac{-6}{4}$ ou $ x=\dfrac{14}{15}$\\On en déduit :  $S=\left\{-\dfrac{3}{2};\dfrac{14}{15}\right\}$

\end{enumerate}

\end{document}