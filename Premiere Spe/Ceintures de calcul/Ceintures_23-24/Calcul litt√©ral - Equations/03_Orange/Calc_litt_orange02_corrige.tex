\documentclass[a4paper,11pt,exos]{nsi} % COMPILE WITH DRAFT


\pagestyle{empty}
\begin{document}


\classe{\premiere spé}
\titre{Ceinture orange 02 - Corrigé}
\maketitle

\begin{exercice}[ : Équations se ramenant au produit-nul]
    Résoudre les équations suivantes :
    %\begin{multicols}{3}
        \begin{enumerate}
            \item $(2x-1)(x+7)=0$
            \item $(8x-8)( x-5)+(8x-8)(4x-9)=0$
            \item $(-8x+4)( -4x-9)=(-8x+4)(-3x-5)$
        \end{enumerate}
    %\end{multicols}
    
\end{exercice}

\begin{enumerate}
    \item On reconnaît une équation produit-nul, donc on applique la propriété :\\
    {\color[HTML]{f15929}Un produit est nul si et seulement si au moins un de ses facteurs est nul.}\\$(2x-1)(x+7)=0$\\$\iff 2x-1=0$ ou $x+7=0$\\$\iff 2x=1$ ou $ x=-7$\\$\iff x=\dfrac{1}{2}$ ou $ x=-7$\\On en déduit :  $S=\left\{-7;\dfrac{1}{2}\right\}$

    \item  $(8x-8)( x-5)+(8x-8)(4x-9)=0$\\ 
    On observe que $(8x-8)$ est un facteur commun dans les deux termes :
    \\ $\phantom{\iff} (\underline{8x-8})( x-5)+(\underline{8x-8)}( 4x-9)=0$\\ 
    $\iff (\underline{8x-8})\Big(( x-5)+(4x-9)\Big)=0$\\ 
    $\iff (8x-8)( 5x-14)=0$\\
    $\iff 8x-8=0\quad$ ou $\quad 5x-14=0$\\
    $\iff x=\dfrac{8}{8}\quad$ ou $\quad x=\dfrac{14}{5}$\\
    On en déduit :  $S=\left\{1;\dfrac{14}{5}\right\}$

    \item  $(-8x+4)( -4x-9)=( -8x+4)( -3x-5)$\\ 
    $\iff(\underline{-8x+4})( -4x-9)- (\underline{-8x+4)}( -3x-5)=0$\\ 
    $\iff (\underline{-8x+4})\Big(( -4x-9)-( -3x-5)\Big)=0$\\ 
    $\iff (-8x+4)( -4x-9+3x+5)=0$\\ $\iff (-8x+4)( -x-4)=0$\\
    $\iff -8x+4=0\quad$ ou $\quad -x-4=0$\\
    $\iff x=\dfrac{4}{8}\quad$ ou $\quad x=-\dfrac{4}{1}$\\
    On en déduit :  $S=\left\{-4;\dfrac{1}{2}\right\}$

\end{enumerate}

\end{document}