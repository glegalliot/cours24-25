\documentclass[a4paper,11pt,exos]{nsi} % COMPILE WITH DRAFT


\pagestyle{empty}
\begin{document}


\classe{\premiere spé}
\titre{Ceinture marron 03 - Corrigé}
\maketitle

\begin{exercice}[ : Utiliser la forme canonique pour résoudre une équation du second degré]%1AL23-20
    Résoudre dans $\R$ l'équation $4x^2+5x+4=0$ sans utiliser le discriminant, mais en utilisant la forme canonique du polynôme.
    
\end{exercice}

On veut résoudre dans $\mathbb{R}$ l'équation $4x^2+5x+4=0\quad(1)$.\\On reconnaît une équation du second degré sous la forme $ax^2+bx+c = 0$.\\La consigne nous amène à commencer par écrire le polynôme du second degré sous forme canonique, \\c'est à dire sous la forme :  $a(x-\alpha)^2+\beta$,\\On commence par diviser les deux membres de l'égalité par le coefficient $a$ qui vaut ici $4$.\\$(1)\iff\quad x^2 +\dfrac{5}{4} x +1=0$\\[.5em]On reconnaît le début d'une identité remarquable :\\$\left(x +\dfrac{5}{8}\right)^2=x^2 +\dfrac{5}{4}x+\dfrac{25}{64} $\\On en déduit que :  $x^2 +\dfrac{5}{4}x= \left(x +\dfrac{5}{8}\right)^2    -\dfrac{25}{64} $\\Il vient alors :\\$\phantom{\iff}\quad x^2 +\dfrac{5}{4} x +1=0$\\$\iff\quad  \left(x +\dfrac{5}{8}\right)^2    -\dfrac{25}{64}+1=0$\\$\iff\quad  \left(x +\dfrac{5}{8}\right)^2    +\dfrac{39}{64}=0$\\L'équation revient à ajouter deux nombres positifs, dont un non-nul. Cette somme ne peut pas être égale à zéro.\\On en déduit que $S=\emptyset$




\end{document}