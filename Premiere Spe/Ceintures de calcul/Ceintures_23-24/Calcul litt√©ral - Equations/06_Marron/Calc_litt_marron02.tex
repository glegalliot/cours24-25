\documentclass[a4paper,11pt,exos]{nsi} 


\pagestyle{empty}
\begin{document}
\subsection*{NOM, Prénom : \dotfill} 


\classe{\premiere spé}
\titre{Ceinture marron 02}
\maketitle




\begin{exercice}[ : Utiliser la forme canonique pour résoudre une équation du second degré]%1AL23-20
    Résoudre dans $\R$ l'équation $2x^2+3x-5=0$ sans utiliser le discriminant, mais en utilisant la forme canonique du polynôme.
    
\end{exercice}

\carreauxseyes{17.6}{16.8}

\end{document}