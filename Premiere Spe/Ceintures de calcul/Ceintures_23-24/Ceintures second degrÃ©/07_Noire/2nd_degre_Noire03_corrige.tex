\documentclass[a4paper,11pt,exos]{nsi} % COMPILE WITH DRAFT


\pagestyle{empty}
\begin{document}

%Exercice 1E15


%\subsection*{NOM, Prénom : \dotfill} 

\classe{\premiere spé}
\titre{Ceinture noire 03 - Corrigé}
\maketitle\

\begin{exercice}[ : Équation à paramètre]
    Déterminer, suivant la valeur du paramètre $m$, le \textbf{nombre de solutions} de l'équation\\ $-2~x^{2}-2~m+2~x=0$.
\end{exercice}


Écrivons l'équation sous la forme $ax^2+bx+c=0$ :\\$-2~x^{2}+2~x-2~m=0$\\On a donc $a=-2$, $b=2$ et $c=-2~m$\\Le discriminant vaut $\Delta(m)=b^2-4\times a\times c = 4-8~2~m$\\Ou encore, sous forme développée : $\Delta(m) = -16~m+4$\\Cherchons la valeur de $m$ qui annule cette expression du premier degré : $m=\dfrac{1}{4}$\\$\Delta$ est une fonction affine décroissante de taux d'accroisssement $-16$.\\$\underline{\text{Conclusion}}$ :\\- Si $m < \dfrac{1}{4}$, l'équation a 2 solutions réelles;\\- Si $m = \dfrac{1}{4}$, l'équation a une unique solution réelle;\\- Si $m > \dfrac{1}{4}$, l'équation n'a pas de solution réelle;


\end{document}