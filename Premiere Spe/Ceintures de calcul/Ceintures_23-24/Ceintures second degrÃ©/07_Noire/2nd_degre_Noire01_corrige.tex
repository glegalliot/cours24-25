\documentclass[a4paper,11pt,exos]{nsi} % COMPILE WITH DRAFT


\pagestyle{empty}
\begin{document}

%Exercice 1E15


%\subsection*{NOM, Prénom : \dotfill} 

\classe{\premiere spé}
\titre{Ceinture noire 01 - Corrigé}
\maketitle\

\begin{exercice}[ : Équation à paramètre]
    Déterminer, suivant la valeur du paramètre $m$, le \textbf{nombre de solutions} de l'équation\\ $-m~x-x^{2}+2~m+3~x+1=0$.
\end{exercice}


Écrivons l'équation sous la forme $ax^2+bx+c=0$ :\\$-x^{2}+(-m+3)~x+2~m+1=0$\\On a donc $a=-1$, $b=-m+3$ et $c=2~m+1$\\Le discriminant vaut $\Delta=b^2-4\times a\times c = \left(-m+3\right)^{2}+4 (2~m+1)$\\Ou encore, sous forme développée : $\Delta = m^{2}+2~m+13$\\Cherchons les valeurs de $m$ qui annulent cette expression du second degré :\\Le discriminant $\Delta^\prime$ vaut : $\Delta^\prime =-48$\\Celui-ci étant strictement négatif, l'équation n'a pas de solution et $\Delta$ ne change pas de signe.\\Comme le coefficient devant $m^2$ est positif, $\Delta > 0$.\\$\underline{\text{Conclusion}}$ : L'équation du départ admet toujours 2 solutions.



\end{document}