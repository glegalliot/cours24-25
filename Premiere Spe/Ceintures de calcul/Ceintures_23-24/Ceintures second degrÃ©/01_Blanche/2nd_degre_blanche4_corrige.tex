\documentclass[a4paper,11pt,exos]{nsi} % COMPILE WITH DRAFT


\pagestyle{empty}
\begin{document}

%Exercice 1AL20-10

\classe{\premiere spé}
\titre{Ceinture blanche 04 - Corrigé}
\maketitle

\begin{exercice}[ : Discriminant d'un polynôme du second degré]
    Calculer le discriminant de chacune de ces expressions :
    \begin{multicols}{2}
        \begin{enumerate}
            \item $A(x) = -x^2-2$.
	        \item $B(x) = 5x^2-3x+3$.
        \end{enumerate}
    \end{multicols}
    
\end{exercice}

    \begin{multicols}{2}
        \begin{enumerate}
            \item$\Delta_A = 0-4\times\left(-1\right)\times\left(-2\right)$\\
                $\Delta_A = 0-8$\\
                $\Delta_A=-8$
        
            \item $\Delta_B =\left(-3\right)^2-4\times5\times3$\\
            $\Delta_B=9-60$\\
            $\Delta_B=-51$
        \end{enumerate}
    \end{multicols}

\end{document}