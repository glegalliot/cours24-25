\documentclass[a4paper,11pt,exos]{nsi} % COMPILE WITH DRAFT


\pagestyle{empty}
\begin{document}

%Exercice 1E11-1


\subsection*{NOM, Prénom : \dotfill} 

\classe{\premiere spé}
\titre{Ceinture blanche 04}
\maketitle

\begin{exercice}[ : Discriminant d'un polynôme du second degré]
    Calculer le discriminant de chacune de ces expressions :
    \begin{multicols}{2}
        \begin{enumerate}
            \item $A(x) = -x^2-2$.
	        \item $B(x) = 5x^2-3x+3$.
        \end{enumerate}
    \end{multicols}
    
\end{exercice}

\carreauxseyes{17.6}{5.6}\\



\subsection*{NOM, Prénom : \dotfill} 


\classe{\premiere spé}
\titre{Ceinture blanche 04}
\maketitle


\begin{exercice}[ : Discriminant d'un polynôme du second degré]
    Calculer le discriminant de chacune de ces expressions :
    \begin{multicols}{2}
        \begin{enumerate}
            \item $A(x) = -x^2-2$.
	        \item $B(x) = 5x^2-3x+3$.
        \end{enumerate}
    \end{multicols}
    
\end{exercice}
\carreauxseyes{17.6}{5.6}

\end{document}