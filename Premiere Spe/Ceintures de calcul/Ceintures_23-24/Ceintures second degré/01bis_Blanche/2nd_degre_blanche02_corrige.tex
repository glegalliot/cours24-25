\documentclass[a4paper,11pt,exos]{nsi} % COMPILE WITH DRAFT


\pagestyle{empty}
\begin{document}



\classe{\premiere spé}
\titre{Ceinture blanche 02 - Corrigé}
\maketitle




\begin{exercice}[ : Forme canonique]
    Déterminer la forme canonique du polynôme $P$, défini pour tout $x \in \mathbb{R}$ par : 
    \begin{multicols}{2}
        \begin{enumerate}
            \item $P(x)=4x^2-8x+4$
            \item $P(x)=3x^2-24x+49$
        \end{enumerate}
    \end{multicols}
\end{exercice}



\begin{enumerate}[itemsep=1em]
    \item On sait que si le polynôme, sous forme développée, s'écrit $P(x)=ax^2+bx+c$, alors sa forme canonique est de la forme $P(x)=a(x-\alpha)^2+\beta$,\\avec $\alpha=\dfrac{-b}{2a}$ et $\beta=P(\alpha).$\\Avec l'énoncé : $a=4$ et $b=-8$, on en déduit que $\alpha=1$.\\On calcule alors $\beta=P(1)$, et on obtient au final que $\beta=0$.\\d'où, $P(x)=4\big(x-1\big)^2$\\Finalement, $P(x)={\color[HTML]{f15929}\boldsymbol{4(x -1)^2}}$
    
    \item On sait que si le polynôme, sous forme développée, s'écrit $P(x)=ax^2+bx+c$, alors sa forme canonique est de la forme $P(x)=a(x-\alpha)^2+\beta$,\\avec $\alpha=\dfrac{-b}{2a}$ et $\beta=P(\alpha).$\\Avec l'énoncé : $a=3$ et $b=-24$, on en déduit que $\alpha=4$.\\On calcule alors $\beta=P(4)$, et on obtient au final que $\beta=1$.\\d'où, $P(x)=3\big(x-4\big)^2+1$\\Finalement, $P(x)={\color[HTML]{f15929}\boldsymbol{3(x -4)^2+1}}$
    
\end{enumerate}
    
    \end{document}
    