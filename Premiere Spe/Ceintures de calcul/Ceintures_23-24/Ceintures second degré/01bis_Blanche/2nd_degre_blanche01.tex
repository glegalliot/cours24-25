\documentclass[a4paper,11pt,exos]{nsi} % COMPILE WITH DRAFT


\pagestyle{empty}
\begin{document}
\subsection*{NOM, Prénom : \dotfill} 


\classe{\premiere spé}
\titre{Ceinture blanche 01}
\maketitle




\begin{exercice}[ : Forme canonique]
Déterminer la forme canonique du polynôme $P$, défini pour tout $x \in \mathbb{R}$ par : 
\begin{multicols}{2}
    \begin{enumerate}
        \item $P(x)=x^2+6x+9$
        \item $P(x)=-2x^2+20x-45$
    \end{enumerate}
\end{multicols}
\end{exercice}

\carreauxseyes{17.6}{17.6}

\end{document}