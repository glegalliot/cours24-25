\documentclass[a4paper,11pt,exos]{nsi} % COMPILE WITH DRAFT


\pagestyle{empty}
\begin{document}



\classe{\premiere spé}
\titre{Ceinture blanche 01 - Corrigé}
\maketitle




\begin{exercice}[ : Forme canonique]
    Déterminer la forme canonique du polynôme $P$, défini pour tout $x \in \mathbb{R}$ par : 
    \begin{multicols}{2}
        \begin{enumerate}
            \item $P(x)=x^2+6x+9$
            \item $P(x)=-2x^2+20x-45$
        \end{enumerate}
    \end{multicols}
\end{exercice}



\begin{enumerate}[itemsep=1em]
    \item On reconnaît une identité remarquable de la forme $a^2+2ab+b^2$.
    \begin{tabbing}
        $P(x)$  \=  $=x^2+6x+9$\\
        \>  $=x^2+2\times x\times 3 +3^2$\\
        \>  $=\color[HTML]{f15929}\boldsymbol{(x+3)^2}$
    \end{tabbing}
    \item On sait que si le polynôme, sous forme développée, s'écrit $P(x)=ax^2+bx+c$, alors sa forme canonique est de la forme $P(x)=a(x-\alpha)^2+\beta$,\\avec $\alpha=\dfrac{-b}{2a}$ et $\beta=P(\alpha).$\\Avec l'énoncé : $a=-2$ et $b=20$, on en déduit que $\alpha=5$.\\On calcule alors $\beta=P(5)$, et on obtient au final que $\beta=5$.\\d'où, $P(x)=-2\big(x-5\big)^2+5$\\Finalement, $P(x)={\color[HTML]{f15929}\boldsymbol{-2(x -5)^2+5}}$
\end{enumerate}
    
    \end{document}
    