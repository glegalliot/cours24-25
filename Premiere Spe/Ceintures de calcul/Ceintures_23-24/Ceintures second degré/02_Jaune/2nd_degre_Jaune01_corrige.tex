\documentclass[a4paper,11pt,exos]{nsi} % COMPILE WITH DRAFT


\pagestyle{empty}
\begin{document}

%Exercice 1E11-2



\classe{\premiere spé}
\titre{Ceinture jaune 01 - Corrigé}
\maketitle

\begin{exercice}[ : Résoudre une équation du second degré]
    Résoudre dans $\R$ les équations suivantes :
    \begin{multicols}{2}
        \begin{enumerate}
            \item $5x^2+30x+48=0$
	    \item $-2x^2+10x-12=0$
        \end{enumerate}
    \end{multicols}
    
\end{exercice}

\begin{enumerate}
    \item $\Delta = 30^2-4\times5\times48=-60$\\$\Delta<0$ donc l'équation n'admet pas de solution.\\$\mathcal{S}=\emptyset$
\item $\Delta = 10^2-4\times(-2)\times(-12)=4$\\$\Delta>0$ donc l'équation admet deux solutions : $x_1 = \dfrac{-b-\sqrt{\Delta}}{2a}$ et $x_2 = \dfrac{-b+\sqrt{\Delta}}{2a}$\\$x_1 =\dfrac{-10-\sqrt{4}}{-4}=3$\\$x_2 =\dfrac{-10+\sqrt{4}}{-4}=2$\\L'ensemble des solutions de cette équation est : $\mathcal{S}=\left\{2 ; 3\right\}$.
\end{enumerate}


\end{document}