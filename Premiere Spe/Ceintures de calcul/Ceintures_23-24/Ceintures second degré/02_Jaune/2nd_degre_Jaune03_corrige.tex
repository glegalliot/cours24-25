\documentclass[a4paper,11pt,exos]{nsi} % COMPILE WITH DRAFT


\pagestyle{empty}
\begin{document}

%Exercice 1E11-2



\classe{\premiere spé}
\titre{Ceinture jaune 03 - Corrigé}
\maketitle

\begin{exercice}[ : Résoudre une équation du second degré]
    Résoudre dans $\R$ les équations suivantes :
    \begin{multicols}{2}
        \begin{enumerate}
            \item $2x^2-4x-6=0$
	        \item $2x^2-8x+12=0$
        \end{enumerate}
    \end{multicols}
    
\end{exercice}

\begin{enumerate}
    \item $\Delta = (-4)^2-4\times2\times(-6)=64$\\$\Delta>0$ donc l'équation admet deux solutions : $x_1 = \dfrac{-b-\sqrt{\Delta}}{2a}$ et $x_2 = \dfrac{-b+\sqrt{\Delta}}{2a}$\\$x_1 =\dfrac{4-\sqrt{64}}{4}=-1$\\$x_2 =\dfrac{4+\sqrt{64}}{4}=3$\\L'ensemble des solutions de cette équation est : $\mathcal{S}=\left\{-1 ; 3\right\}$.
\item $\Delta = (-8)^2-4\times2\times12=-32$\\$\Delta<0$ donc l'équation n'admet pas de solution.\\$\mathcal{S}=\emptyset$
\end{enumerate}


\end{document}