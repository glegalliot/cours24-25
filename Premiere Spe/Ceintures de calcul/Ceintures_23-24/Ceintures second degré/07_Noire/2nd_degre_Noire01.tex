\documentclass[a4paper,11pt,exos]{nsi} % COMPILE WITH DRAFT


\pagestyle{empty}
\begin{document}

%Exercice 1E12


\subsection*{NOM, Prénom : \dotfill} 

\classe{\premiere spé}
\titre{Ceinture noire 01}
\maketitle


\begin{exercice}[ : Équation à paramètre]
    Déterminer, suivant la valeur du paramètre $m$, le \textbf{nombre de solutions} de l'équation\\ $-m~x-x^{2}+2~m+3~x+1=0$.
\end{exercice}

\carreauxseyes{16.8}{14.4}
\end{document}