\documentclass[a4paper,11pt,exos]{nsi} % COMPILE WITH DRAFT


\pagestyle{empty}
\begin{document}

%Exercice 1AL21-10


\subsection*{NOM, Prénom : \dotfill} 

\classe{\premiere spé}
\titre{Ceinture bleue 03 - Corrigé}
\maketitle

\begin{exercice}[ : Trouver l'équation d'une parabole]
    Quelle est l'expression de la fonction polynomiale $f$ du second degré qui s'annule en $x=1$ et en $x=-4$ et dont la parabole passe par le point de coordonnées $(3;7)$ ?\\
    Donner la forme développée de $f$.

    
\end{exercice}


Comme $f\left(1\right) = 0$ et $f\left(-4\right) = 0$, $1$ et $-4$ sont des racines de $f$. On peut donc écrire $f$ sous forme factorisée.\[f(x) = a(x-1)(x-(-4))\]On sait de plus que $f(3) = 7$. Donc : \[7 = f(3) = a(3-1)(3-(-4)) \]On trouve alors : $a = \dfrac{7}{(3-1)(3+4)} = \dfrac{1}{2}$. La forme factorisée de $f$ est donc :\[f(x) = \dfrac{1}{2}(x-1)(x+4)\]On développe et on trouve $f(x) = \dfrac{x^2}{2}+\dfrac{3x}{2}-2$


\end{document}