\documentclass[a4paper,11pt,exos]{nsi} 


\pagestyle{empty}
\begin{document}
\subsection*{NOM, Prénom : \dotfill} 


\classe{\premiere spé}
\titre{Ceinture verte 01}
\maketitle

%Exercice 2N52-5 en selectionnant 4 questions


\begin{exercice}[ : Équations avec un quotient]
    Pour chacune des équations suivantes, préciser les valeurs interdites éventuelles puis résoudre l'équation :
    \begin{multicols}{3}
        \begin{enumerate}
            \item $\dfrac{3x-7}{-3x}=-5$.
	        \item $\dfrac{x^2+1}{4x-16}=0$.
	        %\item $\dfrac{4x-6}{-x+9}=0$.
	        \item  \textit{BONUS :} $\dfrac{4}{9x+1}=\dfrac{4}{3x-2}$.
        \end{enumerate}
    \end{multicols}
    
\end{exercice}

\carreauxseyes{17.6}{16.8}

\end{document}