\documentclass[a4paper,11pt,exos]{nsi} % COMPILE WITH DRAFT


\pagestyle{empty}
\begin{document}


\classe{\premiere spé}
\titre{Ceinture bleue 04 - Corrigé}
\maketitle

\begin{exercice}[ : Utiliser les identités remarquables]
    Résoudre les équations suivantes :
    \begin{multicols}{2}
        \begin{enumerate}
            \item $(2x-3)^2-9x^2=0$
	        \item $4x^2-2x+\dfrac{1}{4}=0$
        \end{enumerate}
    \end{multicols}
    
\end{exercice}

\begin{enumerate}
    \item \begin{tabbing}
        $(2x-3)^2-9x^2=0 \quad$    \=  $\iff\quad (2x-3)^2-(3x)^2=0$\\
        \>  $\iff\quad  (2x-3+3x)(2x-3-3x)=0$\\
        \>  $\iff\quad  (5x-3)(-x-3)=0$\\
        \>  $\iff\quad  5x-3=0 \quad$ ou $\quad -x-3=0$\\
        \>  $\iff\quad  5x=3\quad$ ou $\quad -x = 3$\\
        \>  $\iff\quad  x=\dfrac{3}{5}\quad$ ou $\quad x=-3$\\
        \>  $\iff\quad  x=0,6\quad$ ou $\quad x=-3$
    \end{tabbing}
    Donc $\mathcal{S}=\left\{-3\ ;\dfrac{3}{5} \right\}$.

    \item \begin{tabbing}
        $4x^2-2x+\dfrac{1}{4}=0 \quad$    \=  $\iff\quad \left(2x\right)^2-2\times 2x\times \dfrac{1}{2} +\left(\dfrac{1}{2}\right)^2=0$\\[.5em]
        \>  $\iff\quad  \left(2x-\dfrac{1}{2}\right)^2=0$\\[.5em]
        \>  $\iff\quad  2x-\dfrac{1}{2}=0$\\[.5em]
        \>  $\iff\quad  2x=\dfrac{1}{2}$\\[.5em]
        \>  $\iff\quad  x=\dfrac{1}{4}$
    \end{tabbing}
    Donc $\mathcal{S}=\left\{\dfrac{1}{4} \right\}$.
\end{enumerate}

\end{document}