\documentclass[a4paper,11pt,exos]{nsi} % COMPILE WITH DRAFT


\pagestyle{empty}
\begin{document}


\classe{\premiere spé}
\titre{Ceinture bleue 03 - Corrigé}
\maketitle

\begin{exercice}[ : Utiliser les identités remarquables]
    Résoudre les équations suivantes :
    \begin{multicols}{2}
        \begin{enumerate}
            \item $(x-5)^2-4x^2=0$
	        \item $\dfrac{1}{4}x^2-4x+16=0$
        \end{enumerate}
    \end{multicols}
    
\end{exercice}

\begin{enumerate}
    \item \begin{tabbing}
        $(x-5)^2-4x^2=0 \quad$    \=  $\iff\quad (x-5)^2-(2x)^2=0$\\
        \>  $\iff\quad  (x-5+2x)(x-5-2x)=0$\\
        \>  $\iff\quad  (3x-5)(-x-5)=0$\\
        \>  $\iff\quad  3x-5=0 \quad$ ou $\quad -x-5=0$\\
        \>  $\iff\quad  3x=5\quad$ ou $\quad -x = 5$\\
        \>  $\iff\quad  x=\dfrac{5}{3}\quad$ ou $\quad x=-5$
    \end{tabbing}
    Donc $\mathcal{S}=\left\{-5\ ;\dfrac{5}{3} \right\}$.

    \item \begin{tabbing}
        $\dfrac{1}{4}x^2-4x+16=0 \quad$    \=  $\iff\quad \left(\dfrac{1}{2}x\right)^2-2\times \dfrac{1}{2}x\times 4+4^2=0$\\[.5em]
        \>  $\iff\quad  \left(\dfrac{1}{2}x-4\right)^2=0$\\[.5em]
        \>  $\iff\quad  \dfrac{1}{2}x-4=0$\\[.5em]
        \>  $\iff\quad  \dfrac{1}{2}x=4$\\[.5em]
        \>  $\iff\quad  2\times \dfrac{1}{2}x=2\times 4$\\[.5em]
        \>  $\iff\quad  x=8$
    \end{tabbing}
    Donc $\mathcal{S}=\left\{8 \right\}$.
\end{enumerate}

\end{document}