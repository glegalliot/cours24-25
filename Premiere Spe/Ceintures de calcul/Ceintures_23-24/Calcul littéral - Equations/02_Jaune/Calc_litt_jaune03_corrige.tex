\documentclass[a4paper,11pt,exos]{nsi} % COMPILE WITH DRAFT


\pagestyle{empty}
\begin{document}


\classe{\premiere spé}
\titre{Ceinture jaune 03 - Corrigé}
\maketitle

\begin{exercice}[ : Équations du premier degré (utilisant la distributivité)]
    Résoudre les équations suivantes :
    \begin{multicols}{3}
        \begin{enumerate}
            
            \item $4-(-x+6)=-9x-1$
            \item $8(-2x+2)=x+7$
            \item $4-3(7x-9)=-6x-5$
        \end{enumerate}
    \end{multicols}
    
\end{exercice}

\begin{enumerate}
    \item 	\begin{tabbing}
        $ 4-(-x+6)=-9x-1 \quad$		\=	$\Leftrightarrow\quad 4+x-6=-9x-1 $\\
        \>	$\Leftrightarrow\quad	x-2=-9x-1 $\\[.5em]
        \>	$\Leftrightarrow\quad	x-2{\color[HTML]{f15929}+9x}=-9x+-1{\color[HTML]{f15929}+9x} $\\[.5em]
        \>	$\Leftrightarrow\quad	10x-2=-1 $\\[.5em]
        \>	$\Leftrightarrow\quad	10x-2{\color[HTML]{f15929}+2}=-1{\color[HTML]{f15929}+2} $\\[.5em]
        \>	$\Leftrightarrow\quad	10x=1 $\\[.5em]
        \>	$\Leftrightarrow\quad	\dfrac{10x}{\color[HTML]{f15929}10}=\dfrac{1}{\color[HTML]{f15929}10} $\\[.5em]
        \>	$\Leftrightarrow\quad	x=\dfrac{1}{10}$\\
        \>	$\Leftrightarrow\quad	x=0,1 $	\hspace{4cm} $\mathcal{S}_1=\left\{ \dfrac{1}{10} \right\}$
    \end{tabbing}	
    
            
    \item 	\begin{tabbing}
        $ 8(-2x+2)=x+7 \quad$		\=	$\Leftrightarrow\quad -16x+16=x+7 $\\
        \>	$\Leftrightarrow\quad	-16x+16{\color[HTML]{f15929}-x}=x+7{\color[HTML]{f15929}-x} $\\[.5em]
        \>	$\Leftrightarrow\quad	-17x+16=7 $\\[.5em]
        \>	$\Leftrightarrow\quad	-17x+16{\color[HTML]{f15929}-16}=7{\color[HTML]{f15929}-16} $\\[.5em]
        \>	$\Leftrightarrow\quad	-17x=-9 $\\[.5em]
        \>	$\Leftrightarrow\quad	-17x{\color[HTML]{f15929}\div(-17)}=-9{\color[HTML]{f15929}\div(-17)} $\\[.5em]
        \>	$\Leftrightarrow\quad	x=\dfrac{9}{17} $ \hspace{4cm} $\mathcal{S}_2=\left\{ \dfrac{9}{17} \right\}$
    \end{tabbing}
    
    
    \item 	\begin{tabbing}
        $ 4-3(7x-9)=-6x-5\quad$		\=	$\Leftrightarrow\quad 4-21x+27=-6x-5 $\\
        \>	$\Leftrightarrow\quad	-21x+31=-6x-5 $\\
        \>	$\Leftrightarrow\quad	-21x+31{\color[HTML]{f15929}+6x}=-6x-5{\color[HTML]{f15929}+6x} $\\[.5em]
        \>	$\Leftrightarrow\quad	-15x+31=-5 $\\[.5em]
        \>	$\Leftrightarrow\quad	-15x+31{\color[HTML]{f15929}-31}=-5{\color[HTML]{f15929}-31} $\\[.5em]
        \>	$\Leftrightarrow\quad	-15x=-36 $\\[.5em]
        \>	$\Leftrightarrow\quad	\dfrac{-15x}{\color[HTML]{f15929}-15}=\dfrac{-36}{\color[HTML]{f15929}-15} $\\[.5em]
        \>	$\Leftrightarrow\quad	x=\dfrac{12}{5} $\\
        \>	$\Leftrightarrow\quad	x=2,4 $ \hspace{4cm} $\mathcal{S}_3=\left\{ \dfrac{12}{5} \right\}$
    \end{tabbing}
\end{enumerate}

\end{document}