\documentclass[a4paper,11pt,exos]{nsi} % COMPILE WITH DRAFT


\pagestyle{empty}
\begin{document}


\classe{\premiere spé}
\titre{Ceinture marron 02 - Corrigé}
\maketitle

\begin{exercice}[ : Utiliser la forme canonique pour résoudre une équation du second degré]%1AL23-20
    Résoudre dans $\R$ l'équation $2x^2+3x-5=0$ sans utiliser le discriminant, mais en utilisant la forme canonique du polynôme.
    
\end{exercice}

On veut résoudre dans $\mathbb{R}$ l'équation $2x^2+3x-5=0\quad(1)$.\\On reconnaît une équation du second degré sous la forme $ax^2+bx+c = 0$.\\La consigne nous amène à commencer par écrire le polynôme du second degré sous forme canonique, \\c'est à dire sous la forme :  $a(x-\alpha)^2+\beta$,\\On commence par diviser les deux membres de l'égalité par le coefficient $a$ qui vaut ici $2$.\\$(1)\iff\quad x^2 +\dfrac{3}{2} x -\dfrac{5}{2}=0$\\[.5em]On reconnaît le début d'une identité remarquable :\\$\left(x +\dfrac{3}{4}\right)^2=x^2 +\dfrac{3}{2}x+\dfrac{9}{16} $\\On en déduit que :  $x^2 +\dfrac{3}{2}x= \left(x +\dfrac{3}{4}\right)^2    -\dfrac{9}{16} $\\Il vient alors :\\$\phantom{\iff}\quad x^2 +\dfrac{3}{2} x -\dfrac{5}{2}=0$\\$\iff\quad  \left(x +\dfrac{3}{4}\right)^2    -\dfrac{9}{16}-\dfrac{5}{2}=0$\\$\iff\quad  \left(x +\dfrac{3}{4}\right)^2    -\dfrac{49}{16}=0$\\On reconnaît l'identité remarquable $a^2-b^2$ :\\avec  $a= \left(x +\dfrac{3}{4}\right)$ et $b =\dfrac{7}{4}$\\L'équation à résoudre est équivalente à :\\ $\left(x +\dfrac{3}{4}-\dfrac{7}{4}\right)\left(x +\dfrac{3}{4}+\dfrac{7}{4}\right)=0$\\ $\left(x -1\right)\left(x +\dfrac{5}{2}\right)=0$\\ On applique la propriété du produit nul :\\ Soit $x -1=0$ , soit $x +\dfrac{5}{2}=0$\\ Soit $x = 1$ , soit $x = -\dfrac{5}{2}$\\ $S =\left\{-\dfrac{5}{2};1\right\}$



\end{document}