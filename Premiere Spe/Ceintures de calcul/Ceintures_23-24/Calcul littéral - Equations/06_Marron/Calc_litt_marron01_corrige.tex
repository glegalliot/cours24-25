\documentclass[a4paper,11pt,exos]{nsi} % COMPILE WITH DRAFT


\pagestyle{empty}
\begin{document}


\classe{\premiere spé}
\titre{Ceinture marron 01 - Corrigé}
\maketitle

\begin{exercice}[ : Utiliser la forme canonique pour résoudre une équation du second degré]
    Résoudre dans $\R$ l'équation $2x^2-x-3=0$ sans utiliser le discriminant, mais en utilisant la forme canonique du polynôme.
    
\end{exercice}

On veut résoudre dans $\mathbb{R}$ l'équation $2x^2-x-3=0\quad(1)$.\\On reconnaît une équation du second degré sous la forme $ax^2+bx+c = 0$.\\La consigne nous amène à commencer par écrire le polynôme du second degré sous forme canonique, \\c'est à dire sous la forme :  $a(x-\alpha)^2+\beta$,\\On commence par diviser les deux membres de l'égalité par le coefficient $a$ qui vaut ici $2$.\\$(1)\iff\quad x^2 -\dfrac{1}{2} x -\dfrac{3}{2}=0$\\On reconnaît le début d'une identité remarquable :\\$\left(x -\dfrac{1}{4}\right)^2=x^2 -\dfrac{1}{2}x+\dfrac{1}{16} $\\On en déduit que :  $x^2 -\dfrac{1}{2}x= \left(x -\dfrac{1}{4}\right)^2    -\dfrac{1}{16} $\\Il vient alors :\\$\phantom{\iff}\quad x^2 -\dfrac{1}{2} x -\dfrac{3}{2}=0$\\$\iff\quad  \left(x -\dfrac{1}{4}\right)^2    -\dfrac{1}{16}-\dfrac{3}{2}=0$\\$\iff\quad  \left(x -\dfrac{1}{4}\right)^2    -\dfrac{25}{16}=0$\\On reconnaît l'identité remarquable $a^2-b^2$ :\\avec  $a= \left(x -\dfrac{1}{4}\right)$ et $b =\dfrac{5}{4}$\\L'équation à résoudre est équivalente à :\\ $\left(x -\dfrac{1}{4}-\dfrac{5}{4}\right)\left(x -\dfrac{1}{4}+\dfrac{5}{4}\right)=0$\\ $\left(x -\dfrac{3}{2}\right)\left(x +1\right)=0$\\ On applique la propriété du produit nul :\\ Soit $x -\dfrac{3}{2}=0$ , soit $x +1=0$\\ Soit $x = \dfrac{3}{2}$ , soit $x = -1$\\ $S =\left\{-1;\dfrac{3}{2}\right\}$


\end{document}