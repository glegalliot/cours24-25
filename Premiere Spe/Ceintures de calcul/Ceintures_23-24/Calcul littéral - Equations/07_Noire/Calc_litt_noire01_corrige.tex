\documentclass[a4paper,11pt,exos]{nsi} % COMPILE WITH DRAFT


\pagestyle{empty}
\begin{document}


\classe{\premiere spé}
\titre{Ceinture noire 01 - Corrigé}
\maketitle

\begin{exercice}[ : Mettre au même dénominateur des expressions littérales]%2N41-8
    Préciser les valeurs interdites éventuelles, puis écrire l'expression sous la forme d'un quotient et réduire le numérateur.
    \begin{multicols}{2}
        \begin{enumerate}
            \item $5x+\dfrac{1}{-x-2}$
            \item $\dfrac{-1}{9x+1}-\dfrac{7}{2x+8}$
        \end{enumerate}
    \end{multicols}
    
\end{exercice}

\begin{enumerate}
    \item Déterminer les valeurs interdites de cette expression, revient à
    déterminer les valeurs qui annulent le dénominateur de $\dfrac{1}{-x-2}$,
    puisque la division par $0$ n'existe pas.\\L'équation $-x-2=0$ a pour solution $-2$. \\
    $-2$ est une valeur interdite pour le quotient $\dfrac{1}{-x-2}$.\\
    Pour $x\in \mathbb{R}\smallsetminus\left\{-2\right\}$, \\
    $\begin{aligned}
    5x+\dfrac{1}{-x-2}&=\dfrac{5x(-x-2)}{-x-2}+\dfrac{1}{-x-2}\\
    &=\dfrac{-5x^2-10x+1}{-x-2}\\
    &=\dfrac{-5x^2-10x+1}{-x-2}
    \end{aligned}$

    \item Déterminer les valeurs interdites de cette expression, revient à déterminer les valeurs qui annulent les dénominateurs de $\dfrac{-1}{9x+1}$ et de $\dfrac{7}{2x+8}$, puisque la division par $0$ n'existe pas.\\
    L'équation $9x+1=0$ a pour solution $\dfrac{-1}{9}$. \\
  L'équation $2x+8=0$ a pour solution $-4$. \\
  $\dfrac{-1}{9}$ et $-4$ sont donc des valeurs interdites pour l'expression. \\
  Pour $x\in \mathbb{R}\smallsetminus\left\{-4\,;\,\dfrac{-1}{9}\right\}$, \\
  $\begin{aligned}
  \dfrac{-1}{9x+1}-\dfrac{7}{2x+8}
  &= \dfrac{-1(2x+8)}{(9x+1)(2x+8)}-\dfrac{7(9x+1)}{(9x+1)(2x+8)}\\
  &=\dfrac{-1(2x+8)-7(9x+1)}{(9x+1)(2x+8)}\\
  &=\dfrac{-65x-15}{(9x+1)(2x+8)}
  \end{aligned}$
\end{enumerate}
\end{document}