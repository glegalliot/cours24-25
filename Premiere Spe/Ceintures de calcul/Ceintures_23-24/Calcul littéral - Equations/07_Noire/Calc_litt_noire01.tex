\documentclass[a4paper,11pt,exos]{nsi} 


\pagestyle{empty}
\begin{document}
\subsection*{NOM, Prénom : \dotfill} 


\classe{\premiere spé}
\titre{Ceinture noire 01}
\maketitle




\begin{exercice}[ : Mettre au même dénominateur des expressions littérales]%2N41-8
    Préciser les valeurs interdites éventuelles, puis écrire l'expression sous la forme d'un quotient et réduire le numérateur.
    \begin{multicols}{2}
    \begin{enumerate}
        \item $5x+\dfrac{1}{-x-2}$
        \item $\dfrac{-1}{9x+1}-\dfrac{7}{2x+8}$
    \end{enumerate}
    \end{multicols}
\end{exercice}

\carreauxseyes{17.6}{16.8}

\end{document}