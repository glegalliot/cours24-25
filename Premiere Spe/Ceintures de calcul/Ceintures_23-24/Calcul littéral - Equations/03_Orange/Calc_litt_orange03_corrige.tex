\documentclass[a4paper,11pt,exos]{nsi} % COMPILE WITH DRAFT


\pagestyle{empty}
\begin{document}


\classe{\premiere spé}
\titre{Ceinture orange 03 - Corrigé}
\maketitle


%Exercice 2N52-1 pour la question 1
%Exercice 2N52-4 pour les deux questions suivantes


\begin{exercice}[ : Équations se ramenant au produit-nul]
    Résoudre les équations suivantes :
    %\begin{multicols}{3}
        \begin{enumerate}
            \item $(x-4)(-6x+9)=0$
            \item  ($3x+6)( -4x-2)+(3x+6)(-x-4)=0$
	        \item  ($-8x-6)( -3x-5)=( -8x-6)( -6x+4)$
        \end{enumerate}
    %\end{multicols}
    
\end{exercice}

\begin{enumerate}
    \item On reconnaît une équation produit-nul, donc on applique la propriété :\\
        {\color[HTML]{f15929}Un produit est nul si et seulement si au moins un de ses facteurs est nul.}\\$(x-4)(-6x+9)=0$\\$\iff x-4=0$ ou $-6x+9=0$\\$\iff x=4$ ou $ -6x=-9$\\$\iff x=4$ ou $ x=\dfrac{-9}{-6}$\\On en déduit :  $S=\left\{\dfrac{3}{2};4\right\}$
    \item  $(3x+6)( -4x-2)+(3x+6)(-x-4)=0$\\ On observe que $(3x+6)$ est un facteur commun dans les deux termes :\\ $\phantom{\iff} (\underline{3x+6})( -4x-2)+(\underline{3x+6)}( -x-4)=0$\\ $\iff (\underline{3x+6})\Big(( -4x-2)+(-x-4)\Big)=0$\\ $\iff (3x+6)( -5x-6)=0$\\On reconnaît une équation produit-nul, donc on applique la propriété :\\{\color[HTML]{f15929}Un produit est nul si et seulement si au moins un de ses facteurs est nul.}\\ $\iff 3x+6=0\quad$ ou $\quad -5x-6=0$\\$\iff x=-\dfrac{6}{3}\quad$ ou $\quad x=-\dfrac{6}{5}$\\
                    On en déduit :  $S=\left\{-2;-\dfrac{6}{5}\right\}$
    \item  $(-8x-6)( -3x-5)=( -8x-6)( -6x+4)$\\ 
    $\iff (-8x-6)(-3x-5)-(-8x-6)(-6x+4)=0$\\
    On observe que $(-8x-6)$ est un facteur commun dans les deux termes :\\ $\phantom{\iff} (\underline{-8x-6})( -3x-5)- (\underline{-8x-6)}( -6x+4)=0$\\ $\iff (\underline{-8x-6})\Big(( -3x-5)-( -6x+4)\Big)=0$\\ $\iff (-8x-6)( -3x-5+6x-4)=0$\\ $\iff (-8x-6)( 3x-9)=0$\\On reconnaît une équation produit-nul, donc on applique la propriété :\\{\color[HTML]{f15929}Un produit est nul si et seulement si au moins un de ses facteurs est nul.}\\ $\iff -8x-6=0\quad$ ou $\quad 3x-9=0$\\$\iff x=-\dfrac{6}{8}\quad$ ou $\quad x=\dfrac{9}{3}$\\
                On en déduit :  $S=\left\{-\dfrac{3}{4};3\right\}$
\end{enumerate}

\end{document}