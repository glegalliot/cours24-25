\documentclass[a4paper,11pt,exos]{nsi} % COMPILE WITH DRAFT


\pagestyle{empty}
\begin{document}


\classe{\premiere spé}
\titre{Ceinture orange 06 - Corrigé}
\maketitle


%Exercice 2N52-1 pour la question 1
%Exercice 2N52-4 pour les deux questions suivantes


\begin{exercice}[ : Équations se ramenant au produit-nul]
    Résoudre les équations suivantes :
    %\begin{multicols}{3}
        \begin{enumerate}
            \item $(5x+5)(9x+3)=0$
            \item ($5x+5)( 8x+4)+(5x+5)(-4x+6)=0$
	        \item $(-3x-1)(8x-6)=(-3x-1)(x-3)$
        \end{enumerate}
    %\end{multicols}
    
\end{exercice}

\begin{enumerate}
    \item On reconnaît une équation produit-nul, donc on applique la propriété :\\
    {\color[HTML]{f15929}Un produit est nul si et seulement si au moins un de ses facteurs est nul.}\\$(5x+5)(9x+3)=0$\\$\iff 5x+5=0\quad$ ou $\quad 9x+3=0$\\$\iff 5x=-5\quad$ ou $\quad 9x=-3$\\$\iff x=\dfrac{-5}{5}\quad$ ou $\quad x=\dfrac{-3}{9}$\\On en déduit :  $S=\left\{-1;-\dfrac{1}{3}\right\}$
    
    \item $(5x+5)( 8x+4)+(5x+5)(-4x+6)=0$\\ On observe que $(5x+5)$ est un facteur commun dans les deux termes :\\ $\phantom{\iff} (\underline{5x+5})( 8x+4)+(\underline{5x+5)}( -4x+6)=0$\\ $\iff (\underline{5x+5})\Big(( 8x+4)+(-4x+6)\Big)=0$\\ $\iff (5x+5)( 4x+10)=0$\\$\iff 5x+5=0\quad$ ou $\quad 4x+10=0$\\$\iff x=-\dfrac{5}{5}\quad$ ou $\quad x=-\dfrac{10}{4}$\\
    On en déduit :  $S=\left\{-\dfrac{5}{2};-1\right\}$
    
    \item Deux nombres sont égaux si et seulement si leur différence est nulle.\\$\phantom{\iff}(-3x-1)(8x-6)=(-3x-1)(x-3)$\\$\iff (\underline{-3x-1})(8x-6)-(\underline{-3x-1})(x-3)=0$\\$\iff (\underline{-3x-1})\Big((8x-6)-(x-3)\Big)=0$\\$\iff (-3x-1)(8x-6-x+3)=0$\\$\iff (-3x-1)(7x-3)=0$\\$\iff -3x-1=0\quad$ ou $\quad 7x-3=0$\\$\iff -3x=1 \quad$ ou $\quad 7x=3$\\$\iff x=\dfrac{1}{-3}\quad$ ou $\quad x=\dfrac{3}{7}$\\On en déduit :  $S=\left\{-\dfrac{1}{3};\dfrac{3}{7}\right\}$
\end{enumerate}



\end{document}