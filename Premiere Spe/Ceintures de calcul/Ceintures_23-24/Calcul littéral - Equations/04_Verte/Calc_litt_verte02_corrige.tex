\documentclass[a4paper,11pt,exos]{nsi} % COMPILE WITH DRAFT


\pagestyle{empty}
\begin{document}


\classe{\premiere spé}
\titre{Ceinture verte 02 - Corrigé}
\maketitle


%Exercice 2N52-5 en selectionnant 4 questions


\begin{exercice}[ : Équations avec un quotient]
    Pour chacune des équations suivantes, préciser les valeurs interdites éventuelles puis résoudre l'équation :
    \begin{multicols}{2}
        \begin{enumerate}
            %\item $\dfrac{-x-3}{5x+9}=4$.
	        \item $\dfrac{3x^2-48}{3x+9}=0$.
        
	        %\item $\dfrac{4x-6}{-x+9}=0$.
	        \item  $\dfrac{6}{-x+1}=\dfrac{-3}{-x-9}$.
        \end{enumerate}
    \end{multicols}
    
\end{exercice}

\begin{enumerate}
    %\item Déterminer les valeurs interdites revient à déterminer les valeurs qui annulent le dénominateur du quotient, puisque la division par $0$ n'existe pas.\\Or $\quad 5x+9=0 \quad\iff\quad x=-\dfrac{9}{5}$. 
          %Donc l'ensemble des valeurs interdites est  $\left\{-\dfrac{9}{5}\right\}$. \\
          %Pour tout $x\in \mathbb{R}\smallsetminus\left\{-\dfrac{9}{5}\right\}$,\\
           %$\phantom{\iff}\dfrac{-x-3}{5x+9}=4$\\
           %$ \iff -x-3=4\times(5x+9)\,\,\,\,\,\,\,\text{ car les produits en croix sont égaux.}$\\
            %$\iff -x-3= 20x+36$\\
            %$\iff -21x= 39$\\
           %$\iff x=-\dfrac{13}{7}$\\
           %$-\dfrac{13}{7}$ n'est pas une valeur interdite, donc $\mathcal{S}=\left\{-\dfrac{13}{7}\right\}$.
    
           \item $3x+9=0 \quad\iff\quad x=-3$. 
           Donc l'ensemble des valeurs interdites est  $\left\{-3\right\}$.\\
           Pour tout $x\in \mathbb{R}\smallsetminus\left\{-3\right\}$, \\
            $\phantom{\iff} \dfrac{3x^2-48}{3x+9}=0$\\
            $\iff 3x^2-48=0\,\,\,\,\,\,\, \text{ car }\dfrac{A(x)}{B(x)}=0 \text { si et seulement si } A(x)=0 \text { et } B(x)\neq 0$\\
            $\iff 3x^2=48$\\
            $\iff x^2=16$\\
            $\iff x=\sqrt{16}\quad$ ou $\quad x=-\sqrt{16}$\\
            $\iff x=4\quad$ ou $\quad x=-4$\\
            $4$ et $-4$ ne sont pas des valeurs interdites, donc l'ensemble des solutions est  $\mathcal{S}=\left\{-4\ ; 4\right\}$.

            \item $-x+1=0 \quad\iff\quad x=1\quad$ et $\quad-x-9=0 \quad\iff\quad x=-9$. \\
            Donc l'ensemble des valeurs interdites est  $\left\{-9\,;\,1\right\}$. \\Pour tout $x\in \mathbb{R}\smallsetminus\left\{-9\,;\,1\right\}$,\\
    $\phantom{\iff}\dfrac{6}{-x+1}=\dfrac{-3}{-x-9}$\\
    $\iff -3\times (-x+1)=6\times (-x-9)\,\,\,\,\,\,\, \text{ car les produits en croix sont égaux.}$\\
    $\iff 3x-3=-6x-54$\\
    $\iff9x= -51$\\
    $\iff x=-\dfrac{17}{3}$\\ $-\dfrac{17}{3}$ n'est pas une valeur interdite, donc l'ensemble des solutions est  $\mathcal{S}=\left\{-\dfrac{17}{3}\right\}$.
  
\end{enumerate}

\end{document}