\documentclass[a4paper,11pt,exos]{nsi} % COMPILE WITH DRAFT


\pagestyle{empty}
\begin{document}


\classe{\premiere spé}
\titre{Ceinture verte 01 - Corrigé}
\maketitle


%Exercice 2N52-5 en selectionnant 4 questions


\begin{exercice}[ : Équations avec un quotient]
    Pour chacune des équations suivantes, préciser les valeurs interdites éventuelles puis résoudre l'équation :
    \begin{multicols}{3}
        \begin{enumerate}
            \item $\dfrac{3x-7}{-3x}=-5$.
	        \item $\dfrac{x^2+1}{4x-16}=0$.
	        %\item $\dfrac{4x-6}{-x+9}=0$.
	        \item  \textit{BONUS :} $\dfrac{4}{9x+1}=\dfrac{4}{3x-2}$.
        \end{enumerate}
    \end{multicols}
    
\end{exercice}

\begin{enumerate}
    \item Déterminer les valeurs interdites revient à déterminer les valeurs qui annulent le dénominateur du quotient, puisque la division par $0$ n'existe pas.\\Or $-3x=0 \quad \iff \quad x=0$. Donc l'ensemble des valeurs interdites est  $\left\{0\right\}$. \\
          Pour tout $x\in \mathbb{R}\smallsetminus\left\{0\right\}$,
        \begin{tabbing}
            $\dfrac{3x-7}{-3x}=-5 \quad$ \= $\iff \quad \dfrac{3x-7}{-3x}=\dfrac{-5}{1}$\\
            \>  $\iff \quad 3x-7=-5\times(-3x)\quad \text{ car les produits en croix sont égaux.}$\\
            \>  $\iff \quad 3x-7= 15x$\\
            \>  $\iff \quad -12x= 7$\\
            \>  $\iff \quad x=-\dfrac{7}{12}$
        \end{tabbing}
    Le nombre $-\dfrac{7}{12}$ n'est pas une valeur interdite, donc l'ensemble des solutions de cette équation est  $\mathcal{S}=\left\{-\dfrac{7}{12}\right\}$.

    \item $4x-16=0 \quad \iff \quad x=4$. Donc l'ensemble des valeurs interdites est  $\left\{4\right\}$.\\
    Pour tout $x\in \mathbb{R}\smallsetminus\left\{4\right\}$,
    \begin{tabbing}
        $\dfrac{x^2+1}{4x-16}=0 \quad$ \=  $\iff \quad x^2+1=0\quad$  car $\quad\dfrac{A(x)}{B(x)}=0 \iff  \quad A(x)=0 \text { et } B(x)\neq 0$\\
        \>  $\iff \quad x^2=-1$
    \end{tabbing}
    Puisque $-1<0$, cette équation n'a pas de solution, donc l'ensemble des solutions est  $\mathcal{S}=\varnothing$.

    %\item Déterminer les valeurs interdites revient à déterminer les valeurs qui annulent le dénominateur du quotient, puisque la division par $0$ n'existe pas.\\Or $-x+9=0$ si et seulement si  $x=9$. \\
    %      Donc l'ensemble des valeurs interdites est  $\left\{9\right\}$. \\Pour tout $x\in \mathbb{R}\smallsetminus\left\{9\right\}$,\\
    %$\begin{aligned}
    %\dfrac{4x-6}{-x+9}&=0\\
    %4x-6&=0\,\,\,\,\,\,\, \text{ car }\dfrac{A(x)}{B(x)}=0 \text { si et seulement si } A(x)=0 \text { et } B(x)\neq 0\\
    %x&= \dfrac{3}{2}
    %\end{aligned}$\\ $\dfrac{3}{2}$ n'est pas une valeur interdite, donc l'ensemble des solutions de cette équation est  $\mathcal{S}=\left\{\dfrac{3}{2}\right\}$.

    \item $9x+1=0 \quad\iff\quad x=-\dfrac{1}{9} \quad$ et $\quad 3x-2=0 \quad\iff\quad x=\dfrac{2}{3}$. \\
          Donc l'ensemble des valeurs interdites est  $\left\{-\dfrac{1}{9}\,;\,\dfrac{2}{3}\right\}$. \\Pour tout $x\in \mathbb{R}\smallsetminus\left\{-\dfrac{1}{9}\,;\,\dfrac{2}{3}\right\}$,
    \begin{tabbing}
        $\dfrac{4}{9x+1}=\dfrac{4}{3x-2} \quad$ \=  $\iff \quad 4\times (9x+1)=4\times (3x-2)\quad \text{ car les produits en croix sont égaux.}$\\
        \>  $\iff\quad  36x+4=12x-8$\\
        \>  $\iff\quad  24x= -12$\\
        \>  $\iff\quad  x=-\dfrac{1}{2}$
    \end{tabbing}
    $-\dfrac{1}{2}$ n'est pas une valeur interdite, donc l'ensemble des solutions de cette équation est  $\mathcal{S}=\left\{-\dfrac{1}{2}\right\}$.
\end{enumerate}

\end{document}

