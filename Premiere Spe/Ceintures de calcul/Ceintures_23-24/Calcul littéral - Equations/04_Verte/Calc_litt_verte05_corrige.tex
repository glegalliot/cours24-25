\documentclass[a4paper,11pt,exos]{nsi} % COMPILE WITH DRAFT


\pagestyle{empty}
\begin{document}


\classe{\premiere spé}
\titre{Ceinture verte 05 - Corrigé}
\maketitle


%Exercice 2N52-5 en selectionnant 4 questions


\begin{exercice}[ : Équations avec un quotient]
    Pour chacune des équations suivantes, préciser les valeurs interdites éventuelles puis résoudre l'équation :
    \begin{multicols}{3}
        \begin{enumerate}
            \item $\dfrac{8}{9x-5}=\dfrac{7}{-2x+7}$.
	        \item $\dfrac{3x^2-3}{8x+24}=0$.
	       
        \end{enumerate}
    \end{multicols}
    
\end{exercice}

\begin{enumerate}
    \item Déterminer les valeurs interdites revient à déterminer les valeurs qui annulent les dénominateurs des quotients, puisque la division par $0$ n'existe pas.\\Or $9x-5=0$ si et seulement si  $x=\dfrac{5}{9}$ et $-2x+7=0$ si et seulement si  $x=\dfrac{7}{2}$. \\
    Donc l'ensemble des valeurs interdites est  $\left\{\dfrac{5}{9}\,;\,\dfrac{7}{2}\right\}$. \\Pour tout $x\in \mathbb{R}\smallsetminus\left\{\dfrac{5}{9}\,;\,\dfrac{7}{2}\right\}$,\\
\begin{tabbing}
$\dfrac{8}{9x-5}=\dfrac{7}{-2x+7} \quad$ \= $\iff\quad 
7\times (9x-5)=8\times (-2x+7) \qquad\text{ car les produits en croix sont égaux.}$\\
\>  $\iff \quad63x-35=-16x+56$\\
\>  $\iff\quad79x= 91$\\
\>  $\iff\quad x=\dfrac{91}{79}$
\end{tabbing}
$\dfrac{91}{79}$ n'est pas une valeur interdite, donc l'ensemble des solutions de cette équation est  $\mathcal{S}=\left\{\dfrac{91}{79}\right\}$.



    \item Déterminer les valeurs interdites revient à déterminer les valeurs qui annulent le dénominateur du quotient, puisque la division par $0$ n'existe pas.\\Or $8x+24=0$ si et seulement si  $x=-3$. \\
    Donc l'ensemble des valeurs interdites est  $\left\{-3\right\}$.\\
    Pour tout $x\in \mathbb{R}\smallsetminus\left\{-3\right\}$, \\
      \begin{tabbing}
      $\dfrac{3x^2-3}{8x+24}=0\quad$    \= $ \iff\quad
      3x^2-3=0 \qquad \text{ car }\dfrac{A(x)}{B(x)}=0 \text { si et seulement si } A(x)=0 \text { et } B(x)\neq 0.$\\
        \>    $\iff\quad 3x^2=3$\\
        \>    $\iff\quad x^2=1$\\
        \>  $\iff\quad x= 1\text{ ou } x= -1$
     \end{tabbing}
       $-1$ et $1$ ne sont pas des valeurs interdites, donc l'ensemble des solutions de cette équation est  $\mathcal{S}=\left\{-1\,;\,1\right\}$.

  
\end{enumerate}

\end{document}