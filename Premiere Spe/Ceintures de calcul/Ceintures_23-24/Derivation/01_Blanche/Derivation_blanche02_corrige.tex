\documentclass[a4paper,11pt,exos]{nsi} % COMPILE WITH DRAFT


\pagestyle{empty}
\begin{document}

%Exercices can1F08 can1F10 canF11

\classe{\premiere spé}
\titre{Ceinture blanche 02 - Corrigé}
\maketitle

\begin{exercice}[ : Dérivée d'une fonction polynôme]
    Pour chacune des fonctions suivantes définies sur $\R$, déterminer l'expression algébrique de sa fonction dérivée :
    \begin{multicols}{3}
        \begin{enumerate}
            \item $f_1(x)=-4x+1{,}6$
        
            \item $f_2(x)= -4x^2+10x-10$
            \item $f_3(x)= -x^3-6x^2-5$
        \end{enumerate}
    \end{multicols}
    
    \end{exercice}

    %\begin{multicols}{2}
        \begin{enumerate}
            \item On reconnaît une fonction affine de la forme $f_1(x)=mx+p$ avec $m=-4$ et $p=1{,}6$.\\
            La fonction dérivée est donnée par $f_1'(x)=m$, soit ici $f_1'(x)=-4$.
        
            \item $f_2$ est une fonction polynôme du second degré de la forme $f_2(x)=ax^2+bx+c$.\\
            La fonction dérivée est donnée par la somme des dérivées des fonctions $u$ et $v$ définies par $u(x)=-4x^2$ et $v(x)=10x-10$.\\
             Comme $u'(x)=-8x$ et $v'(x)=10$, on obtient  $f_2'(x)=-8x+10$.
            
            \item $f_3$ est une fonction polynôme du troisième degré de la forme $f_3(x)=ax^3+bx^2+cx+d$ avec $c=0$.\\
            La fonction dérivée est donnée par la somme des dérivées des fonctions $u$, $v$ et $w$ définies par $u(x)=-x^3$, $v(x)=-6x^2$ et $w(x)=-5$.\\
             Comme $u'(x)=-3x^2$, $v'(x)=-12x$ et $w'(x)=0$, on obtient  $f_3'(x)=-3x^2-12x$. 
        \end{enumerate}
    %\end{multicols}

\end{document}