\documentclass[a4paper,11pt,exos]{nsi} % COMPILE WITH DRAFT


\pagestyle{empty}
\begin{document}

%Exercices can1F08 can1F10 canF11

\classe{\premiere spé}
\titre{Ceinture blanche 03 - Corrigé}
\maketitle

\begin{exercice}[ : Dérivée d'une fonction polynôme]
    Pour chacune des fonctions suivantes définies sur $\R$, déterminer l'expression algébrique de sa fonction dérivée :
    \begin{multicols}{3}
        \begin{enumerate}
            \item $f_1(x)=-8-x$
        
            \item $f_2(x)= -3x^2+8$
            \item $f_3(x)= -3x^2-5x^3-2x+3$
        \end{enumerate}
    \end{multicols}
    
    \end{exercice}

    %\begin{multicols}{2}
        \begin{enumerate}
            \item On reconnaît une fonction affine de la forme $f_1(x)=mx+p$ avec $m=-1$ et $p=8$.\\
            La fonction dérivée est donnée par $f_1'(x)=m$, soit ici $f_1'(x)=-1$.
        
            \item $f_2$ est une fonction polynôme du second degré de la forme $f_2(x)=ax^2+b$.\\
            La fonction dérivée est donnée par la somme des dérivées des fonctions $u$ et $v$ définies par $u(x)=-3x^2$ et $v(x)=8$.\\
             Comme $u'(x)=-6x$ et $v'(x)=0$, on obtient  $f_2'(x)=-6x$.
            
            \item $f_3$ est une fonction polynôme du troisième degré de la forme $f_3(x)=ax^3+bx^2+cx+d$ avec $c=0$.\\
            La fonction dérivée est donnée par la somme des dérivées des fonctions $u$, $v$ et $w$ définies par $u(x)=-5x^3$, $v(x)=-3x^2$ et $w(x)=-2x+3$.\\
             Comme $u'(x)=-15x^2$, $v'(x)=-6x$ et $w'(x)=-2$, on obtient  $f_3'(x)=-15x^2-6x-2$. 
        \end{enumerate}
    %\end{multicols}

\end{document}