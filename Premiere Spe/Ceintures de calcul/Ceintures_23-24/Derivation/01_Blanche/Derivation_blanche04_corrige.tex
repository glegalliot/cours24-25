\documentclass[a4paper,11pt,exos]{nsi} % COMPILE WITH DRAFT


\pagestyle{empty}
\begin{document}

%Exercices can1F08 can1F10 canF11

\classe{\premiere spé}
\titre{Ceinture blanche 04 - Corrigé}
\maketitle

\begin{exercice}[ : Dérivée d'une fonction polynôme]
    Pour chacune des fonctions suivantes définies sur $\R$, déterminer l'expression algébrique de sa fonction dérivée :
    \begin{multicols}{3}
        \begin{enumerate}
            \item $f_1(x)=1{,}6-1{,}3x$
        
            \item $f_2(x)= 10x+3-2x^2$
            \item $f_3(x)= -8x^2+4x^3$
        \end{enumerate}
    \end{multicols}
    
    \end{exercice}

    %\begin{multicols}{2}
        \begin{enumerate}
            \item On reconnaît une fonction affine de la forme $f_1(x)=mx+p$ avec $m=-1,3$ et $p=1,6$.\\
            La fonction dérivée est donnée par $f_1'(x)=m$, soit ici $f_1'(x)=-1,3$.
        
            \item $f_2$ est une fonction polynôme du second degré de la forme $f_2(x)=ax^2+bx+c$.\\
            La fonction dérivée est donnée par la somme des dérivées des fonctions $u$ et $v$ définies par $u(x)=-2x^2$ et $v(x)=10x+3$.\\
             Comme $u'(x)=-4x$ et $v'(x)=10$, on obtient  $f_2'(x)=-4x+10$.
            
            \item $f_3$ est une fonction polynôme du troisième degré de la forme $f_3(x)=ax^3+bx^2+cx+d$ avec $c=0$ et $d=0$.\\
            La fonction dérivée est donnée par la somme des dérivées des fonctions $u$ et $v$  définies par $u(x)=4x^3$ et $v(x)=-8x^2$.\\
             Comme $u'(x)=12x^2$, $v'(x)=-16x$, on obtient  $f_3'(x)=12x^2-16x$. 
        \end{enumerate}
    %\end{multicols}

\end{document}