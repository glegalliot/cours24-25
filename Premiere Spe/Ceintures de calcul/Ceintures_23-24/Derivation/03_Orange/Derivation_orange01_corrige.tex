\documentclass[a4paper,11pt,exos]{nsi} % COMPILE WITH DRAFT


\pagestyle{empty}
\begin{document}

%Exercices can1F14

\classe{\premiere spé}
\titre{Ceinture orange 01 - Corrigé}
\maketitle

\begin{exercice}[ : Déterminer un nombre dérivé]
    Soit $f$ la fonction définie sur $\mathbb{R}$ par : $f(x)= 3x^2-5x+10$.\\
        Déterminer $f'(2)$.
\end{exercice}

$f$ est une fonction polynôme du second degré de la forme $f(x)=ax^2+bx+c$.\\
    La fonction dérivée est donnée par la somme des dérivées des fonctions $u$ et $v$ définies par $u(x)=3x^2$ et $v(x)=-5x+10$.\\
     Comme $u'(x)=6x$ et $v'(x)=-5$, on obtient  $f'(x)=6x-5$. \\
     Ainsi, $f'(2)=6\times 2-5=7$.


\end{document}