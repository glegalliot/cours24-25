\documentclass[a4paper,11pt,exos]{nsi} % COMPILE WITH DRAFT


\pagestyle{empty}
\begin{document}

%Exercices 1F10

\classe{\premiere spé}
\titre{Ceinture jaune 03 - Corrigé}
\maketitle

\begin{exercice}[ : Dérivée d'une fonction de base]
    Pour chacune des fonctions suivantes, dire sur quel ensemble elle est dérivable, puis déterminer l'expression de sa fonction dérivée.
    \begin{multicols}{3}
        \begin{enumerate}
            \item $f:x\longmapsto { x}^{3}-\dfrac{9}{{ x}}$
        
            \item $g:x\longmapsto 3~{ x}^{2}+2~ x-8$
            \item $h:x\longmapsto x-\sqrt{ x}$
        \end{enumerate}
    \end{multicols}
    
\end{exercice}

\begin{enumerate}[itemsep=1em]
    \item $f$ est définie et dérivable sur $\mathbb{R}^{\text{*}}$ et $ f':x\longmapsto 3x^2+\dfrac{9}{x^2}$
    \item $g$ est définie et dérivable sur $\mathbb{R}$ et $ g':x\longmapsto 6 x+2$
    
    \item $h$ est définie sur $\fio{0}{+\infty}$ et dérivable sur $\oio{0}{+\infty}$ et $ h':x\longmapsto 1-\dfrac{1}{2\sqrt{ x}}$
    \end{enumerate}


\end{document}