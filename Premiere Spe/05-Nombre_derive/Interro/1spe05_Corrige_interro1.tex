\documentclass[a4paper,11pt,exos]{nsi} 
\usepackage{pifont}
\usepackage{fontawesome5}

\pagestyle{empty}



\begin{document}



\classe{\premiere spé}
\titre{Corrigé de l'interrogation - Sujet A}
\maketitle


\exo{ Questions de cours}
On se donne une fonction $f$ définie que $\R$, $a$ un nombre réel et $h$ un nombre réel non nul.\\
On note $\mathcal{C}_f$ la courbe représentative de $f$ dans un repère orthonormé.
\begin{enumerate}
    \item Donner le taux de variation de $f$ entre $a$ et $a+h$.\\[.5em]
    \textcolor{UGLiBlue}{
        Le taux de variation de $f$ entre $a$ et $a+h$ est $\dfrac{f(a+h)-f(a)}{h}$.
    }
    \item Donner deux manières de définir le nombre dérivé de $f$ en $a$.\\[.5em]
    \textcolor{UGLiBlue}{
        Le nombre dérivé de $f$ en $a$ est la limite du taux de variation de $f$ entre $a$ et $a+h$ lorsque $h$ tend vers 0.\\
        Le nombre dérivé de $f$ en $a$ est le coefficient directeur de la tangente à $\mathcal{C}_f$ en $a$.
    }
    \item Donner l'équation de la tangente à $\mathcal{C}_f$ en $a$.\\[.5em]
    \textcolor{UGLiBlue}{
        L'équation de la tangente à $\mathcal{C}_f$ en $a$ est $y=f'(a)(x-a)+f(a)$.
    }
\end{enumerate}


\dleft{10cm}{
    \exo{}
    $f$ est une fonction dérivable sur $\R$ dont la courbe $\mathcal{C}_f$ est représentée ci-contre ainsi que trois de ses tangentes.\\[.5em]	
	Lire $f'(0)$, $f'(2)$ et $f'(6)$.\\[.5em]
	\textcolor{UGLiBlue}{
        $f'(0)=-\dfrac{1}{2}$, $\quad f'(2)=0\quad$ et $\quad f'(6)=\dfrac{9}{2}$.}
}
{
    \begin{tikzpicture}[scale=0.6]
		\reperevl{-2}{-2}{8}{10}
		\clip (-2,-2) rectangle (8,10);
		\draw [thick, domain = -2:8, samples = 1000] plot(\x,{1/24*(\x)^3-0.5*\x+2});
		\draw [domain = -2:8, samples = 2] plot(\x,{4/3});
		\draw [domain = -2:8, samples = 2] plot(\x,{2-0.5*\x});
		\draw [domain = -2:8, samples = 2] plot(\x,{4.5*(\x-6)+8});
		\draw [ball color=gray] (0,2) circle (.05) (2,4/3) circle (.05) (6,8) circle(.05);
	\end{tikzpicture}
}


\begin{minipage}{10.5cm}
    \exo{}
	Le plan est muni d'un repère $\rep$.\\
	On considère la fonction $f$ définie sur $\R$ par $f(x)=x^2-4x+\dfrac{3}{2}$.\\
	
	$\mathcal{C}_f$ est sa courbe représentative dans le repère $\rep$.
	\begin{enumerate}[]
		\item 	Soit $h$ un réel non nul.\\
        Calculer le taux d'accroissement de $f$ entre $3$ et $3+h$.
		\item En déduire $f'(3)$.
		\item 	Donner l'équation de la tangente $(T_3)$ à $\mathcal{C}_f$ au point d'abscisse 3 et la tracer sur ce graphique.\\
	\end{enumerate}
\end{minipage}
\begin{minipage}{6.5cm}
	\begin{center}
		\begin{tikzpicture}
			\clip (-1,-3) rectangle (5,4);
			\reperev{-1}{-3}{5}{4}
			\draw [thick, domain = -2:7, samples = 1000] plot(\x,{\x*\x-4*\x+1.5});
			\draw (4,3.5) node{$\courbe{f}$};
		\end{tikzpicture}
	\end{center}
\end{minipage}
%\vspace{1cm}
\textcolor{UGLiBlue}{
    \begin{enumerate}
        \item Soit $h$ un réel non nul.
        \begin{multicols}{2}
            \begin{tabbing}
                $f(3+h)$ \= $=(3+h)^2-4(3+h)+\dfrac{3}{2}$\\
                \> $= (9+6h+h^2)-12-4h+\dfrac{3}{2}$\\
                \> $= h^2+2h-\dfrac{3}{2}$   
            \end{tabbing}
            \begin{tabbing}
                et $\quad f(3)$ \= $=3^2-4\times 3+\dfrac{3}{2}$\\
                \> $=9-12+\dfrac{3}{2}$\\
                \> $=-\dfrac{3}{2}$
            \end{tabbing}
        \end{multicols}
        \begin{tabbing}
            D'où $\quad \dfrac{f(3+h)-f(3)}{h}$ \= $= \dfrac{h^2+2h-\dfrac{3}{2}+\dfrac{3}{2}}{h}$\\
            \> $= \dfrac{h^2+2h}{h}$\\
            \> $= h+2$
        \end{tabbing}
        \item \begin{tabbing}
            $f'(3)$ \= $= \lim\limits_{h\to 0} \dfrac{f(3+h)-f(3)}{h}$\\[.5em]
            \> $= \lim\limits_{h\to 0} h+2$\\[.5em]
            \> $= 2$
        \end{tabbing}
        \item L'équation de la tangente $(T_3)$ à $\mathcal{C}_f$ au point d'abscisse 3 est $y=2(x-3)-\dfrac{3}{2}$.
        \begin{tabbing}
            $(T_3)$ \= $: y=2(x-3)-\dfrac{3}{2}$\\[.5em]
            \> $: y=2x-6-\dfrac{3}{2}$\\[.5em]
            \> $: y=2x-\dfrac{15}{2}$
        \end{tabbing}
        \begin{tikzpicture}
            \clip (-1,-3) rectangle (5,4);
            \reperev{-1}{-3}{5}{4}
            \draw [thick, domain = -2:7, samples = 1000] plot(\x,{\x*\x-4*\x+1.5});
            \draw [UGLiRed,domain = -1:5, samples = 2] plot(\x,{2*(\x-3)-1.5});
            \draw (4,3.5) node{$\courbe{f}$};
            \draw[UGLiRed] (4.2,1.5) node[below right]{$(T_3)$};
        \end{tikzpicture}
    \end{enumerate}
}

\newpage

\setcounter{section}{0}


\titre{Corrigé de l'interrogation - Sujet B}
\maketitle


\exo{ Questions de cours}
On se donne une fonction $f$ définie que $\R$, $a$ un nombre réel et $h$ un nombre réel non nul.\\
On note $\mathcal{C}_f$ la courbe représentative de $f$ dans un repère orthonormé.
\begin{enumerate}
    \item Donner le taux de variation de $f$ entre $a$ et $a+h$.\\[.5em]
    \textcolor{UGLiBlue}{
        Le taux de variation de $f$ entre $a$ et $a+h$ est $\dfrac{f(a+h)-f(a)}{h}$.
    }
    \item Donner deux manières de définir le nombre dérivé de $f$ en $a$.\\[.5em]
    \textcolor{UGLiBlue}{
        Le nombre dérivé de $f$ en $a$ est la limite du taux de variation de $f$ entre $a$ et $a+h$ lorsque $h$ tend vers 0.\\
        Le nombre dérivé de $f$ en $a$ est le coefficient directeur de la tangente à $\mathcal{C}_f$ en $a$.
    }
    \item Donner l'équation de la tangente à $\mathcal{C}_f$ en $a$.\\[.5em]
    \textcolor{UGLiBlue}{
        L'équation de la tangente à $\mathcal{C}_f$ en $a$ est $y=f'(a)(x-a)+f(a)$.
    }
\end{enumerate}


\dleft{10cm}{
    \exo{}
    $f$ est une fonction dérivable sur $\R$ dont la courbe $\mathcal{C}_f$ est représentée ci-contre ainsi que trois de ses tangentes.\\[.5em]	
	Lire $f'(0)$, $f'(2)$ et $f'(6)$.\\[.5em]
	\textcolor{UGLiBlue}{
        $f'(0)=-\dfrac{1}{2}$, $\quad f'(2)=0\quad$ et $\quad f'(6)=\dfrac{9}{2}$.}
}
{
    \begin{tikzpicture}[scale=0.6]
		\reperevl{-2}{-2}{8}{10}
		\clip (-2,-2) rectangle (8,10);
		\draw [thick, domain = -2:8, samples = 1000] plot(\x,{1/24*(\x)^3-0.5*\x+2});
		\draw [domain = -2:8, samples = 2] plot(\x,{4/3});
		\draw [domain = -2:8, samples = 2] plot(\x,{2-0.5*\x});
		\draw [domain = -2:8, samples = 2] plot(\x,{4.5*(\x-6)+8});
		\draw [ball color=gray] (0,2) circle (.05) (2,4/3) circle (.05) (6,8) circle(.05);
	\end{tikzpicture}
}


\begin{minipage}{10.5cm}
    \exo{}
	Le plan est muni d'un repère $\rep$.\\
	On considère la fonction $f$ définie sur $\R$ par $f(x)=x^2-2x-\dfrac{3}{2}$.\\
	
	$\mathcal{C}_f$ est sa courbe représentative dans le repère $\rep$.
	\begin{enumerate}[]
		\item 	Soit $h$ un réel non nul.\\
        Calculer le taux d'accroissement de $f$ entre $3$ et $3+h$.
		\item En déduire $f'(3)$.
		\item 	Donner l'équation de la tangente $(T_3)$ à $\mathcal{C}_f$ au point d'abscisse 3 et la tracer sur ce graphique.\\
	\end{enumerate}
\end{minipage}
\begin{minipage}{6.5cm}
	\begin{center}
		\begin{tikzpicture}
			\clip (-1,-3) rectangle (5,4);
			\reperev{-1}{-3}{5}{4}
			\draw [thick, domain = -2:7, samples = 1000] plot(\x,{\x*\x-2*\x-1.5});
			\draw (3,3.5) node{$\courbe{f}$};
		\end{tikzpicture}
	\end{center}
\end{minipage}
\textcolor{UGLiBlue}{
    \begin{enumerate}
        \item Soit $h$ un réel non nul.
        \begin{multicols}{2}
            \begin{tabbing}
                $f(3+h)$ \= $=(3+h)^2-2(3+h)-\dfrac{3}{2}$\\
                \> $= (9+6h+h^2)-6-2h-\dfrac{3}{2}$\\
                \> $= h^2+4h+\dfrac{3}{2}$   
            \end{tabbing}
            \begin{tabbing}
                et $\quad f(3)$ \= $=3^2-2\times 3-\dfrac{3}{2}$\\
                \> $=9-6-\dfrac{3}{2}$\\
                \> $=\dfrac{3}{2}$
            \end{tabbing}
        \end{multicols}
        \begin{tabbing}
            D'où $\quad \dfrac{f(3+h)-f(3)}{h}$ \= $= \dfrac{h^2+4h+\dfrac{3}{2}-\dfrac{3}{2}}{h}$\\
            \> $= \dfrac{h^2+4h}{h}$\\
            \> $= h+4$
        \end{tabbing}
        \item \begin{tabbing}
            $f'(3)$ \= $= \lim\limits_{h\to 0} \dfrac{f(3+h)-f(3)}{h}$\\[.5em]
            \> $= \lim\limits_{h\to 0} h+4$\\[.5em]
            \> $= 4$
        \end{tabbing}
        \item L'équation de la tangente $(T_3)$ à $\mathcal{C}_f$ au point d'abscisse 3 est $y=4(x-3)+\dfrac{3}{2}$.
        \begin{tabbing}
            $(T_3)$ \= $: y=4(x-3)+\dfrac{3}{2}$\\[.5em]
            \> $: y=4x-12+\dfrac{3}{2}$\\[.5em]
            \> $: y=4x-\dfrac{21}{2}$
        \end{tabbing}
        \begin{tikzpicture}
            \clip (-1,-3) rectangle (5,4);
            \reperev{-1}{-3}{5}{4}
            \draw [thick, domain = -2:7, samples = 1000] plot(\x,{\x*\x-2*\x-1.5});
            \draw [UGLiRed,domain = -1:5, samples = 2] plot(\x,{4*(\x-3)+1.5});
            \draw (3,3.5) node{$\courbe{f}$};
            \draw[UGLiRed] (2,-2.5) node[right]{$(T_3)$};
        \end{tikzpicture}
    \end{enumerate}
}
\end{document}