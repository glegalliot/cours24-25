\documentclass[a4paper,11pt,eval]{nsi} 
\usepackage{pifont}
\usepackage{fontawesome5}

\pagestyle{empty}


\newcounter{exoNum}
\setcounter{exoNum}{0}
%
\newcommand{\exo}[1]
{
	\addtocounter{exoNum}{1}
	{\titlefont\color{UGLiBlue}\Large Exercice\ \theexoNum\ \normalsize{#1}}\smallskip	
}



\begin{document}



\textcolor{UGLiBlue}{Jeudi 27/02/2025}\\
\classe{\premiere spé}
\titre{Interrogation - Sujet A}
\maketitle
\vspace*{.1cm}

\exo{ Questions de cours}\bareme{4 pts}\\
On se donne une fonction $f$ définie que $\R$, $a$ un nombre réel et $h$ un nombre réel non nul.\\
On note $\mathcal{C}_f$ la courbe représentative de $f$ dans un repère orthonormé.
\begin{enumerate}
    \item Donner le taux de variation de $f$ entre $a$ et $a+h$.\\[.5em]
    \carreauxseyes{16}{1.6}
    \item Donner deux manières de définir le nombre dérivé de $f$ en $a$.\\[.5em]
    \carreauxseyes{16}{3.2}
    \item Donner l'équation de la tangente à $\mathcal{C}_f$ en $a$.\\[.5em]
    \carreauxseyes{16}{1.6}
\end{enumerate}


\dleft{10cm}{
    \exo{}\bareme{1,5 pt}\\
    $f$ est une fonction dérivable sur $\R$ dont la courbe $\mathcal{C}_f$ est représentée ci-contre ainsi que trois de ses tangentes.\\[.5em]	
	Lire $f'(0)$, $f'(2)$ et $f'(6)$.\\[.5em]
	\carreauxseyes{9.6}{4}
}
{
    \begin{tikzpicture}[scale=0.6]
		\reperevl{-2}{-2}{8}{10}
		\clip (-2,-2) rectangle (8,10);
		\draw [thick, domain = -2:8, samples = 1000] plot(\x,{1/24*(\x)^3-0.5*\x+2});
		\draw [domain = -2:8, samples = 2] plot(\x,{4/3});
		\draw [domain = -2:8, samples = 2] plot(\x,{2-0.5*\x});
		\draw [domain = -2:8, samples = 2] plot(\x,{4.5*(\x-6)+8});
		\draw [ball color=gray] (0,2) circle (.05) (2,4/3) circle (.05) (6,8) circle(.05);
	\end{tikzpicture}
}


\begin{minipage}{10.5cm}
    \exo{}\bareme{4,5 pts}\\
	Le plan est muni d'un repère $\rep$.\\
	On considère la fonction $f$ définie sur $\R$ par $f(x)=x^2-4x+\dfrac{3}{2}$.\\
	
	$\mathcal{C}_f$ est sa courbe représentative dans le repère $\rep$.
	\begin{enumerate}[]
		\item 	Soit $h$ un réel non nul.\\
        Calculer le taux d'accroissement de $f$ entre $3$ et $3+h$.
		\item En déduire $f'(3)$.
		\item 	Donner l'équation de la tangente $(T_3)$ à $\mathcal{C}_f$ au point d'abscisse 3 et la tracer sur ce graphique.\\
	\end{enumerate}
\end{minipage}
\begin{minipage}{6.5cm}
	\begin{center}
		\begin{tikzpicture}
			\clip (-1,-3) rectangle (5,4);
			\reperev{-1}{-3}{5}{4}
			\draw [thick, domain = -2:7, samples = 1000] plot(\x,{\x*\x-4*\x+1.5});
			\draw (4,3.5) node{$\courbe{f}$};
		\end{tikzpicture}
	\end{center}
\end{minipage}
\vspace{1cm}
\carreauxseyes{16.8}{17.6}

\newpage

\setcounter{exoNum}{0}


\textcolor{UGLiBlue}{Jeudi 27/02/2025}\\
\classe{\premiere spé}
\titre{Interrogation - Sujet B}
\maketitle
\vspace*{.1cm}

\exo{ Questions de cours}\bareme{4 pts}\\
On se donne une fonction $f$ définie que $\R$, $a$ un nombre réel et $h$ un nombre réel non nul.\\
On note $\mathcal{C}_f$ la courbe représentative de $f$ dans un repère orthonormé.
\begin{enumerate}
    \item Donner le taux de variation de $f$ entre $a$ et $a+h$.\\[.5em]
    \carreauxseyes{16}{1.6}
    \item Donner deux manières de définir le nombre dérivé de $f$ en $a$.\\[.5em]
    \carreauxseyes{16}{3.2}
    \item Donner l'équation de la tangente à $\mathcal{C}_f$ en $a$.\\[.5em]
    \carreauxseyes{16}{1.6}
\end{enumerate}


\dleft{10cm}{
    \exo{}\bareme{1,5 pt}\\
    $f$ est une fonction dérivable sur $\R$ dont la courbe $\mathcal{C}_f$ est représentée ci-contre ainsi que trois de ses tangentes.\\[.5em]	
	Lire $f'(0)$, $f'(2)$ et $f'(6)$.\\[.5em]
	\carreauxseyes{9.6}{4}
}
{
    \begin{tikzpicture}[scale=0.6]
		\reperevl{-2}{-2}{8}{10}
		\clip (-2,-2) rectangle (8,10);
		\draw [thick, domain = -2:8, samples = 1000] plot(\x,{1/24*(\x)^3-0.5*\x+2});
		\draw [domain = -2:8, samples = 2] plot(\x,{4/3});
		\draw [domain = -2:8, samples = 2] plot(\x,{2-0.5*\x});
		\draw [domain = -2:8, samples = 2] plot(\x,{4.5*(\x-6)+8});
		\draw [ball color=gray] (0,2) circle (.05) (2,4/3) circle (.05) (6,8) circle(.05);
	\end{tikzpicture}
}


\begin{minipage}{10.5cm}
    \exo{}\bareme{4,5 pts}\\
	Le plan est muni d'un repère $\rep$.\\
	On considère la fonction $f$ définie sur $\R$ par $f(x)=x^2-2x-\dfrac{3}{2}$.\\
	
	$\mathcal{C}_f$ est sa courbe représentative dans le repère $\rep$.
	\begin{enumerate}[]
		\item 	Soit $h$ un réel non nul.\\
        Calculer le taux d'accroissement de $f$ entre $3$ et $3+h$.
		\item En déduire $f'(3)$.
		\item 	Donner l'équation de la tangente $(T_3)$ à $\mathcal{C}_f$ au point d'abscisse 3 et la tracer sur ce graphique.\\
	\end{enumerate}
\end{minipage}
\begin{minipage}{6.5cm}
	\begin{center}
		\begin{tikzpicture}
			\clip (-1,-3) rectangle (5,4);
			\reperev{-1}{-3}{5}{4}
			\draw [thick, domain = -2:7, samples = 1000] plot(\x,{\x*\x-2*\x-1.5});
			\draw (3,3.5) node{$\courbe{f}$};
		\end{tikzpicture}
	\end{center}
\end{minipage}
\vspace{1cm}
\carreauxseyes{16.8}{17.6}
\end{document}