\documentclass[a4paper,12pt,french]{article}
\usepackage[margin=2cm]{geometry}
\usepackage[thinfonts,latinmath]{uglix2}
\pagestyle{empty}
\begin{document}
\titreinterro{Interrogation de cours - A}{\premiere}{13/09/2021}

\begin{center}
	\textbf{Calculatrice interdite}\\
	Toutes les réponses doivent être bien rédigées.\\
\end{center}

\exonote{}
Vérifier que $-3$ est une solution de $\quad 7x+12 = -x-12$.\\

\petitscarreaux{17}{7}
	
Vérifier que $3$ est une solution de $\quad \dfrac{3}{4}x+\dfrac{3}{5}=\dfrac{5}{2}x-\dfrac{93}{20}$.\\


\petitscarreaux{17}{7}
	
\exonote{}
Résoudre les équations suivantes :
\begin{multicols}{2}
	\begin{enumerate}[\bfseries 1.]
		\item	$2x+7 = 2x -7$	
		\item	$\dfrac{1}{2}x+1=\dfrac{1}{4}x+\dfrac{1}{5}$
	\end{enumerate}
\end{multicols}

\petitscarreaux{17}{20}\\

\eject
\titreinterro{Interrogation de cours - B}{\premiere}{13/09/2021}

\begin{center}
	\textbf{Calculatrice interdite}\\
	Toutes les réponses doivent être bien rédigées.\\
\end{center}

\setcounter{exoNum}{0}
\exonote{}
Vérifier que $-4$ est une solution de $\quad 6x+13 = -x-15$.\\

\petitscarreaux{17}{7}

Vérifier que $3$ est une solution de $\quad \dfrac{5}{4}x+\dfrac{2}{5}=\dfrac{7}{2}x-\dfrac{127}{20}$.\\


\petitscarreaux{17}{7}

\exonote{}
Résoudre les équations suivantes :
\begin{multicols}{2}
	\begin{enumerate}[\bfseries 1.]
		\item	$5x+4 = 5x -4$	
		\item	$\dfrac{1}{2}x+1=\dfrac{1}{4}x+\dfrac{1}{3}$
	\end{enumerate}
\end{multicols}

\petitscarreaux{17}{20}\\
\end{document}
