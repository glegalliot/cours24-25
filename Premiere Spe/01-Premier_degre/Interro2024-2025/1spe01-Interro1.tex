\documentclass[a4paper,11pt,eval]{nsi} 


\pagestyle{empty}


\newcounter{exoNum}
\setcounter{exoNum}{0}
%
\newcommand{\exo}[1]
{
	\addtocounter{exoNum}{1}
	{\titlefont\color{UGLiBlue}\Large Exercice\ \theexoNum\ \normalsize{#1}}\smallskip	
}



\begin{document}



\textcolor{UGLiBlue}{Jeudi 19/09/2024}\\
\classe{\premiere spé}
\titre{Interrogation 1 - Sujet A}
\maketitle
\begin{center}
	Calculatrice interdite
\end{center}

\vspace{1cm}

\dleft{7.5cm}{
    \exo{}\\

	On définit sur $\R$ la fonction $f$ par 
	$$f(x)=\dfrac{1}{2}x-3.$$
    Représenter $\courbe{f}$, la courbe représentative de $f$ dans le repère ci-contre.
	}
{

\def\xmin{-5} \def\ymin{-5}\def\xmax{7}\def\ymax{5}
\begin{center}
	\begin{tikzpicture}[scale=.7]
		\reperevl{\xmin}{\ymin}{\xmax}{\ymax}
		\clip (\xmin,\ymin) rectangle (\xmax,\ymax);
	\end{tikzpicture}
\end{center}
}

\exo{}\\

\dleft{8cm}{
	\def\xmin{-8} \def\ymin{-8}\def\xmax{8}\def\ymax{8}
	\begin{tikzpicture}[scale=.5]
		\reperevl{\xmin}{\ymin}{\xmax}{\ymax}
		\clip (\xmin,\ymin) rectangle (\xmax,\ymax);
		\draw[domain=\xmin:\xmax,smooth,variable=\x] plot ({\x},{-2*\x+6});
		\draw[domain=\xmin:\xmax,smooth,variable=\x] plot ({\x},{3*\x-7});
		\draw[domain=\xmin:\xmax,smooth,variable=\x] plot ({\x},{2/3*\x+1});
		\draw (4,-4) node[below]{$\mathcal{C}_f$};
		\draw (-7,-4) node[below]{$\mathcal{C}_g$};
		\draw (-1,-7) node[below]{$\mathcal{C}_h$};
		
\end{tikzpicture}}
{	On a représenté ici trois fonctions affines $f$, $g$ et $h$.\\
	
	Compléter  sans justifier:\\
	
	$f(x)=\quad$\dotfill\\
	
	$g(x)=\quad$\dotfill\\
	
	$h(x)=\quad$\dotfill\\	
	
}

\newpage
\exo{}\\
\begin{enumerate}
    \item Donner le tableau de signes des expressions suivantes :
    \begin{multicols}{3}
        \begin{enumalph}
            \item   $3x-2$
            \item 	$(2x+1)(-3x+2)$
            \item	$\dfrac{-5x-2}{x+8}$
        \end{enumalph}
    \end{multicols}
    \carreauxseyes{16}{16}
    \item En déduire les solutions des inéquations suivantes :
    \begin{multicols}{3}
        \begin{enumalph}
            \item   $3x-2\leqslant 0$
            \item 	$(2x+1)(-3x+2)<0$
            \item	$\dfrac{-5x-2}{x+8}\geqslant 0$
        \end{enumalph}
    \end{multicols}
    \carreauxseyes{16}{3.2}
\end{enumerate}

\newpage
\setcounter{exoNum}{0}
\textcolor{UGLiBlue}{Jeudi 19/09/2024}\\
\classe{\premiere spé}
\titre{Interrogation 1 - Sujet B}
\maketitle
\begin{center}
	Calculatrice interdite
\end{center}

\vspace{1cm}

\dleft{7.5cm}{
    \exo{}\\

	On définit sur $\R$ la fonction $f$ par 
	$$f(x)=-\dfrac{1}{2}x+3.$$
    Représenter $\courbe{f}$, la courbe représentative de $f$ dans le repère ci-contre.
	}
{

\def\xmin{-5} \def\ymin{-5}\def\xmax{7}\def\ymax{5}
\begin{center}
	\begin{tikzpicture}[scale=.7]
		\reperevl{\xmin}{\ymin}{\xmax}{\ymax}
		\clip (\xmin,\ymin) rectangle (\xmax,\ymax);
	\end{tikzpicture}
\end{center}
}

\exo{}\\

\dleft{8cm}{
	\def\xmin{-8} \def\ymin{-8}\def\xmax{8}\def\ymax{8}
	\begin{tikzpicture}[scale=.5]
		\repereal{\xmin}{\ymin}{\xmax}{\ymax}
		\clip (\xmin,\ymin) rectangle (\xmax,\ymax);
		\draw[domain=\xmin:\xmax,smooth,variable=\x] plot ({\x},{3*\x-4});
		\draw[domain=\xmin:\xmax,smooth,variable=\x] plot ({\x},{-2*\x+5});
		\draw[domain=\xmin:\xmax,smooth,variable=\x] plot ({\x},{3/2*\x+2});
		\draw (-2.2,-6) node[below right]{$\mathcal{C}_f$};
		\draw (7,-6) node[below]{$\mathcal{C}_g$};
		\draw (-6,-6) node[below left]{$\mathcal{C}_h$};
		
\end{tikzpicture}}
{	On a représenté ici trois fonctions affines $f$, $g$ et $h$.\\
	
	Compléter  sans justifier:\\
	
	$f(x)=\quad$\dotfill\\
	
	$g(x)=\quad$\dotfill\\
	
	$h(x)=\quad$\dotfill\\	
	
}

\newpage
\exo{}\\
\begin{enumerate}
    \item Donner le tableau de signes des expressions suivantes :
    \begin{multicols}{3}
        \begin{enumalph}
            \item   $2x-5$
            \item 	$(2x-1)(-5x-3)$
            \item	$\dfrac{-4x+1}{x+3}$
        \end{enumalph}
    \end{multicols}
    \carreauxseyes{16}{16}
    \item En déduire les solutions des inéquations suivantes :
    \begin{multicols}{3}
        \begin{enumalph}
            \item   $2x-5\geqslant 0$
            \item 	$(2x-1)(-5x-3)>0$
            \item	$\dfrac{-4x+1}{x+3}\leqslant 0$
        \end{enumalph}
    \end{multicols}
    \carreauxseyes{16}{3.2}
\end{enumerate}

\end{document}