\documentclass[a4paper,11pt,eval]{nsi} 


\pagestyle{empty}


\newcounter{exoNum}
\setcounter{exoNum}{0}
%
\newcommand{\exo}[1]
{
	\addtocounter{exoNum}{1}
	{\titlefont\color{UGLiBlue}\Large Exercice\ \theexoNum\ \normalsize{#1}}\smallskip	
}



\begin{document}



\textcolor{UGLiBlue}{Jeudi 03/10/2024}\\
\classe{\premiere spé}
\titre{Interrogation 2 - Sujet A}
\maketitle
\begin{center}
	Calculatrice interdite
\end{center}

%\vspace{.5cm}
\exo{}\bareme{3 pts}\\
\dleft{8cm}{
    Résoudre graphiquement le système suivant : $$\left\{
			\begin{array}{l}
				\ y=-2x+3 \\
				\ x-2y=4 \\
			\end{array} \right.$$

    \carreauxseyes{8}{5.6}
}
{
    \def\xmin{-5}	\def\xmax{7}	\def\ymin{-5}	\def\ymax{7}
    \begin{tikzpicture}[scale = .7]
        \draw[fill=white](\xmin,\ymin) rectangle (\xmax,\ymax);
        \reperevl{\xmin}{\ymin}{\xmax}{\ymax}
        \clip (\xmin,\ymin) rectangle (\xmax,\ymax);
        %\draw[UGLiOrange,domain=\xmin:\xmax,samples=2,variable=\x] plot ({\x},{2*\x+1});
        %\draw[UGLiDarkGreen,domain=\xmin:\xmax,samples=2,variable=\x] plot ({\x},{-3*\x+6});
        %\node[UGLiOrange,rotate=60] () 	at 	(-3,-3){$(d_1)$};
        %\node[UGLiDarkGreen,rotate=-60]() at 	(4,-3){$(d_2)$};
        %\draw (1,3)\ball  node[right]{A};
    \end{tikzpicture}
}

\exo{}\bareme{3 pts}\\
Sophia a travailé durant l'été 45 jours dans deux entreprises. Dans la première, elle a gagné 85 € par jour et dans la deuxième, 72 € par jour. Au total, elle a gagné 3487 €.\\
On cherche à savoir le nombre de jours pendant lesquels Sophia a travaillé dans chaque entreprise.\\[.5em]
Modéliser cette situation par un système. \textit{On ne demande pas de le résoudre.}\\[.5em]
\carreauxseyes{16.8}{5.6}

\exo{}\bareme{8 pts}
%\begin{multicols}{2}
    \begin{enumerate}
        \item Résoudre le système suivant par substitution : $\left\{
			\begin{array}{l}
				\ 3x+y=4 \\
				\ -2x+3y=-10 \\
			\end{array} \right.$
        \item Résoudre le système suivant par combinaison linéaire : $\left\{
			\begin{array}{l}
				\ 3x-4y=6 \\
				\ -5x+2y=4 \\
			\end{array} \right.$
    \end{enumerate}
%\end{multicols}

\carreauxseyes{16.8}{21.6}\\
\carreauxseyes{16.8}{25.6}

%\newpage
\exo{}\bareme{6 pts}\\
Soit $x$ un nombre réel strictement supérieur à $1$.\\
$A, B$ et $C$ sont trois points tels que $\quad AB=4x+3$, $\quad BC=3(x-1)\quad$ et $\quad AC=5x+1$.\\
On considère le point $D$ tel que $ABCD$ est un parallélogramme.
\begin{enumerate}
	\item 	Faire un schéma codé.\\
	\vspace*{2.2cm}
	\item   Résoudre l'équation $\quad 25x^2+10x+1=25x^2+6x+18$.\\[.5em]
	\carreauxseyes{16}{4}
	\item 	Démontrer que le parallélogramme $ABCD$ est un rectangle si, et seulement si $\quad x=\dfrac{17}{4}$.\\
	\textit{Raisonner par équivalences.}\\[.5em]
	\carreauxseyes{16}{8}
	\item	 Quelle est alors la longueur $BD$ ?\\[.5em]
	\carreauxseyes{16}{4}
\end{enumerate}

\newpage
\setcounter{exoNum}{0}
\textcolor{UGLiBlue}{Jeudi 03/10/2024}\\
\classe{\premiere spé}
\titre{Interrogation 2 - Sujet B}
\maketitle
\begin{center}
	Calculatrice interdite
\end{center}

%\vspace{.5cm}
\exo{}\bareme{3 pts}\\
\dleft{8cm}{
    Résoudre graphiquement le système suivant : $$\left\{
			\begin{array}{l}
				\ y=2x-1 \\
				\ x+2y=8 \\
			\end{array} \right.$$

    \carreauxseyes{8}{5.6}
}
{
    \def\xmin{-5}	\def\xmax{7}	\def\ymin{-5}	\def\ymax{7}
    \begin{tikzpicture}[scale = .7]
        \draw[fill=white](\xmin,\ymin) rectangle (\xmax,\ymax);
        \reperevl{\xmin}{\ymin}{\xmax}{\ymax}
        \clip (\xmin,\ymin) rectangle (\xmax,\ymax);
        %\draw[UGLiOrange,domain=\xmin:\xmax,samples=2,variable=\x] plot ({\x},{2*\x+1});
        %\draw[UGLiDarkGreen,domain=\xmin:\xmax,samples=2,variable=\x] plot ({\x},{-3*\x+6});
        %\node[UGLiOrange,rotate=60] () 	at 	(-3,-3){$(d_1)$};
        %\node[UGLiDarkGreen,rotate=-60]() at 	(4,-3){$(d_2)$};
        %\draw (1,3)\ball  node[right]{A};
    \end{tikzpicture}
}

\exo{} \bareme{3 pts}\\
Sophia a travailé durant l'été 52 jours dans deux entreprises. Dans la première, elle a gagné 65 € par jour et dans la deuxième, 82 € par jour. Au total, elle a gagné 3652 €.\\
On cherche à savoir le nombre de jours pendant lesquels Sophia a travaillé dans chaque entreprise.\\[.5em]
Modéliser cette situation par un système. \textit{On ne demande pas de le résoudre.}\\[.5em]
\carreauxseyes{16.8}{5.6}

\exo{}\bareme{8 pts}
%\begin{multicols}{2}
    \begin{enumerate}
        \item Résoudre le système suivant par substitution : $\left\{
			\begin{array}{l}
				\ x-2y=5 \\
				\ 3x+2y=7 \\
			\end{array} \right.$
        \item Résoudre le système suivant par combinaison linéaire : $\left\{
			\begin{array}{l}
				\ 3x-4y=5 \\
				\ -6x+5y=-4 \\
			\end{array} \right.$
    \end{enumerate}
%\end{multicols}

\carreauxseyes{16.8}{21.6}\\
\carreauxseyes{16.8}{25.6}

%\newpage
\exo{}\bareme{6 pts}\\
Soit $x$ un nombre réel strictement supérieur à $1$.\\
$A, B$ et $C$ sont trois points tels que $\quad AB=4x+3$, $\quad BC=3(x-1)\quad$ et $\quad AC=5x+1$.\\
On considère le point $D$ tel que $ABCD$ est un parallélogramme.
\begin{enumerate}
	\item 	Faire un schéma codé.\\
	\vspace*{2.2cm}
	\item   Résoudre l'équation $\quad 25x^2+10x+1=25x^2+6x+18$.\\[.5em]
	\carreauxseyes{16}{4}
	\item 	Démontrer que le parallélogramme $ABCD$ est un rectangle si, et seulement si $\quad x=\dfrac{17}{4}$.\\
	\textit{Raisonner par équivalences.}\\[.5em]
	\carreauxseyes{16}{8}
	\item	 Quelle est alors la longueur $BD$ ?\\[.5em]
	\carreauxseyes{16}{4}
\end{enumerate}


\end{document}