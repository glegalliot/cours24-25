\documentclass[a4paper,11pt,exos]{nsi} % COMPILE WITH DRAFT
\usepackage{pifont}
\usepackage{fontawesome5}


\pagestyle{empty}
\begin{document}
\classe{\premiere spé}
\titre{Devoir en temps libre}
\maketitle





Soit $\ell$ un nombre réel.\\
$A, B$ et $C$ sont trois points tels que $\quad AB=20\ell +12$, $\quad BC=15(\ell+1)\quad$ et $\quad AC=25\ell +19$.\\
On considère le point $D$ tel que $ABCD$ est un parallélogramme.
\begin{enumerate}
	\item 	Faire un schéma et rappeler une condition nécessaire et suffisante pour qu'un parallélogramme soit un rectangle.
	\item 	Déterminer toutes les valeurs de $\ell$ pour lesquelles $ABCD$ est un rectangle.
	\item	 Quelle est alors la longueur $BD$ ?\\
\end{enumerate}


\vspace*{1cm}


\maketitle

Soit $\ell$ un nombre réel.\\
$A, B$ et $C$ sont trois points tels que $\quad AB=20\ell +12$, $\quad BC=15(\ell+1)\quad$ et $\quad AC=25\ell +19$.\\
On considère le point $D$ tel que $ABCD$ est un parallélogramme.
\begin{enumerate}
	\item 	Faire un schéma et rappeler une condition nécessaire et suffisante pour qu'un parallélogramme soit un rectangle.
	\item 	Déterminer toutes les valeurs de $\ell$ pour lesquelles $ABCD$ est un rectangle.
	\item	 Quelle est alors la longueur $BD$ ?\\
\end{enumerate}

\vspace*{1cm}

\maketitle

Soit $\ell$ un nombre réel.\\
$A, B$ et $C$ sont trois points tels que $\quad AB=20\ell +12$, $\quad BC=15(\ell+1)\quad$ et $\quad AC=25\ell +19$.\\
On considère le point $D$ tel que $ABCD$ est un parallélogramme.
\begin{enumerate}
	\item 	Faire un schéma et rappeler une condition nécessaire et suffisante pour qu'un parallélogramme soit un rectangle.
	\item 	Déterminer toutes les valeurs de $\ell$ pour lesquelles $ABCD$ est un rectangle.
	\item	 Quelle est alors la longueur $BD$ ?\\
\end{enumerate}


\end{document}
