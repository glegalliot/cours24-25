\documentclass[a4paper,12pt,french]{article}
\usepackage[margin=2cm]{geometry}
\usepackage[thinfonts,latinmath]{uglix2}
\pagestyle{empty}
\begin{document}
\titreinterro{Interrogation de cours - A}{\premiere}{04/10/2021}

\exonote{}

\double{
	On définit sur $\R$ la fonction $f$ par 
	$$f(x)=\dfrac{5x+10}{6}.$$
	
	\textbf{1.} Pourquoi $f$ est-elle affine ? \textbf{Justifier}\\
	
	\carreauxseyes{8}{4.8}
	}
{
\textbf{2.} Représenter $\courbe{f}$ dans le repère ci-dessous.
\def\xmin{-5} \def\ymin{-4}\def\xmax{7}\def\ymax{6}
\begin{center}
	\begin{tikzpicture}[scale=.7]
		\repereal{\xmin}{\ymin}{\xmax}{\ymax}
		\clip (\xmin,\ymin) rectangle (\xmax,\ymax);
	\end{tikzpicture}
\end{center}
}{8.5cm}

\vspace{1cm}
\exonote{}

\double{
	\def\xmin{-8} \def\ymin{-8}\def\xmax{8}\def\ymax{8}
	\begin{tikzpicture}[scale=.5]
		\repereal{\xmin}{\ymin}{\xmax}{\ymax}
		\clip (\xmin,\ymin) rectangle (\xmax,\ymax);
		\draw[domain=\xmin:\xmax,smooth,variable=\x] plot ({\x},{-2*\x+6});
		\draw[domain=\xmin:\xmax,smooth,variable=\x] plot ({\x},{3*\x-7});
		\draw[domain=\xmin:\xmax,smooth,variable=\x] plot ({\x},{2/7*\x+1});
		\draw (4,-4) node[below]{$\mathcal{C}_f$};
		\draw (-7,-4) node[below]{$\mathcal{C}_g$};
		\draw (-1,-7) node[below]{$\mathcal{C}_h$};
		
\end{tikzpicture}}
{	On a représenté ici trois fonctions affines $f$, $g$ et $h$.\\
	
	Compléter  sans justifier:\\
	
	$f(x)=\qquad$\dotfill\\
	
	$g(x)=\qquad$\dotfill\\
	
	$h(x)=\qquad$\dotfill\\	
	
}
{8.5cm}
\eject
\exonote{}

$f$ est une fonction affine telle que $f(2)=1$ et $f(5)=3$.\\
Déterminer l'expression algébrique de $f(x)$.\\

\carreauxseyes{16.8}{19.2}\\

\setcounter{exoNum}{0}
\titreinterro{Interrogation de cours - B}{\premiere}{04/10/2021}

\exonote{}

\double{
	On définit sur $\R$ la fonction $f$ par 
	$$f(x)=\dfrac{-5x+20}{6}.$$
	
	\textbf{1.} Pourquoi $f$ est-elle affine ? \textbf{Justifier}\\
	
	\carreauxseyes{8}{4.8}
}
{
	\textbf{2.} Représenter $\courbe{f}$ dans le repère ci-dessous.
	\def\xmin{-5} \def\ymin{-4}\def\xmax{7}\def\ymax{6}
	\begin{center}
		\begin{tikzpicture}[scale=.7]
			\repereal{\xmin}{\ymin}{\xmax}{\ymax}
			\clip (\xmin,\ymin) rectangle (\xmax,\ymax);
		\end{tikzpicture}
	\end{center}
}{8.5cm}

\vspace{1cm}
\exonote{}

\double{
	\def\xmin{-8} \def\ymin{-8}\def\xmax{8}\def\ymax{8}
	\begin{tikzpicture}[scale=.5]
		\repereal{\xmin}{\ymin}{\xmax}{\ymax}
		\clip (\xmin,\ymin) rectangle (\xmax,\ymax);
		\draw[domain=\xmin:\xmax,smooth,variable=\x] plot ({\x},{2*\x-6});
		\draw[domain=\xmin:\xmax,smooth,variable=\x] plot ({\x},{-3*\x+7});
		\draw[domain=\xmin:\xmax,smooth,variable=\x] plot ({\x},{-2/7*\x-1});
		\draw (4,-4) node[below]{$\mathcal{C}_f$};
		\draw (-7,-4) node[below]{$\mathcal{C}_g$};
		\draw (-1,-7) node[below]{$\mathcal{C}_h$};
		
\end{tikzpicture}}
{	On a représenté ici trois fonctions affines $f$, $g$ et $h$.\\
	
	Compléter  sans justifier:\\
	
	$f(x)=\qquad$\dotfill\\
	
	$g(x)=\qquad$\dotfill\\
	
	$h(x)=\qquad$\dotfill\\	
	
}
{8.5cm}
\eject
\exonote{}

$f$ est une fonction affine telle que $f(1)=5$ et $f(4)=3$.\\
Déterminer l'expression algébrique de $f(x)$.\\

\carreauxseyes{16.8}{19.2}\\
\end{document}
