\documentclass[a4paper,11pt,exos]{nsi} % COMPILE WITH DRAFT
\usepackage{pifont}
\usepackage{fontawesome5}

%\pagestyle{empty}

\begin{document}
\classe{\premiere spé}
\titre{Second degré - Corrigé des exercices}
\maketitle

\subsection*{Plusieurs formes pour un même polynôme}

\exo{}
\textcolor{UGLiBlue}{
Donner la forme développée des fonctions définies sur $\R$ par :
\begin{multicols}{3}
	\begin{enumerate}[label=\textbullet]
		\item 	$f(x)=(x-1)(x-2)$.
		\item 	$g(x)=(x-3)(x-4)$.
		\item	$l(x)=7(x+2)(x-5)$.
		\item 	$m(x)=-3x(x-2)$.
		\item 	$k(x)=(x-\sigma_1)(x-\sigma_2)$ \\où $\sigma_1$ et $\sigma_2$ sont deux réels.\columnbreak
	\end{enumerate}
\end{multicols}}

\begin{multicols}{2}
    Soit $x$ un réel.
    \begin{tabbing}
        $f(x)$  \= $=(x-1)(x-2)$\\
        \>  $=x\times x+x\times (-2)-1\times x-1\times (-2)$\\
        \>  $=x^2-2x-x+2$\\
        \>  $=x^2-3x+2$
    \end{tabbing}
    
    \begin{tabbing}
        $g(x)$  \=$=(x-3)(x-4)$\\
        \>  $=x\times x+x\times (-4)-3\times x-3\times (-4)$\\
        \>  $=x^2-4x-3x+12$\\
        \>  $=x^2-7x+12$
    \end{tabbing}

    \vfill\null \columnbreak
    \begin{tabbing}
        $l(x)$  \=$=7(x+2)(x-5)$\\
        \>  $=7\left(x^2-5x+2x-10\right)$\\
        \>  $=7\left(x^2-3x-10\right)$\\
        \>  $=7x^2-21x-30$
    \end{tabbing}

    \begin{tabbing}
        $m(x)$ \=$=-3x(x-2)$\\
        \>  $=-3x\times x-3x\times (-2)$\\
        \>  $=-3x^2+6x$
    \end{tabbing}

    \begin{tabbing}
        $k(x)$  \=$=(x-\sigma_1)(x-\sigma_2)$\\
        \>  $=x\times x +x\times (-\sigma_2)-\sigma_1\times x-\sigma_1\times\sigma_2$\\
        \>  $=x^2-\sigma_2 x-\sigma_1 x+\sigma_1\ \sigma_2$\\
        \>  $=x^2-\left(\sigma_1+\sigma_2\right)x+ \sigma_1\ \sigma_2$
    \end{tabbing}
\end{multicols}

\exo{}
\textcolor{UGLiBlue}{
Donner la forme développée des fonctions définies sur $\R$ par :
\begin{multicols}{3}
	\begin{enumerate}[label=\textbullet]
		\item 	$f(x)=(x+1)^2+1$
		\item 	$g(x)=3(x-1)^2+7$.
		\item 	$h(x)=-2(x+7)^2+2$.
		\item 	$k(x)=\left(x-\dfrac{1}{2}\right)^2+\dfrac{3}{4}$.
		\item	$l(x)=(x-\alpha)^2+\beta$\\ où $\alpha$ et $\beta$ sont deux réels.
	\end{enumerate}
\end{multicols}}

Soit $x$ un réel.
\begin{multicols}{2}
        \begin{tabbing}
        $f(x)$  \=$=\textcolor{UGLiOrange}{(x+1)^2}+1$\\
        \>  $=\textcolor{UGLiOrange}{x^2+2\times x\times 1+1^2}+1$\\
        \>  $=x^2+2x+1+1$\\
        \>  $=x^2+2x+2$
    \end{tabbing}

    \vfill\null \columnbreak

    \begin{tabbing}
        $g(x)$  \=$=3\textcolor{UGLiOrange}{(x-1)^2}+7$\\
        \>  $=3\left(\textcolor{UGLiOrange}{x^2-2\times x\times 1+1^2}\right)+7$\\
        \>  $=\textcolor{UGLiPurple}{3\left(x^2-2x+1\right)}+7$\\
        \>  $=\textcolor{UGLiPurple}{3x^2-6x+3}+7$\\
        \>  $=3x^2-6x+10$
    \end{tabbing}

    \newpage

    \begin{tabbing}
        $h(x)$  \=$=-2\textcolor{UGLiOrange}{(x+7)^2}+2$\\
        \>  $=-2\left(\textcolor{UGLiOrange}{x^2+2\times x\times 7+7^2}\right)+2$\\
        \>  $=\textcolor{UGLiPurple}{-2\left(x^2+14x+49\right)}+2$\\
        \>  $=\textcolor{UGLiPurple}{-2x^2-28x-98}+2$\\
        \>  $=-2x^2-28x-96$
    \end{tabbing}

    \vfill\null\columnbreak
    \begin{tabbing}
        $k(x)$  \=$=\textcolor{UGLiOrange}{\left(x-\dfrac{1}{2}\right)^2}+\dfrac{3}{4}$\\
        \>  $=\textcolor{UGLiOrange}{x^2-2\times x\times \dfrac{1}{2}+\left(\dfrac{1}{2}\right)^2}+\dfrac{3}{4}$\\%[.5em]
        \>  $=x^2-x+\dfrac{1}{4}+\dfrac{3}{4}$\\%[.5em]
        \>  $=x^2-x+1$
    \end{tabbing}

    

    \begin{tabbing}
        $l(x)$  \=$=\textcolor{UGLiOrange}{(x-\alpha)^2}+\beta$\\
        \>  $=\textcolor{UGLiOrange}{x^2-2\times x\times \alpha+\alpha^2}+\beta$\\
        \>  $=x^2-2\alpha x+\alpha^2+\beta$
    \end{tabbing}
\end{multicols}


\exo{}
\textcolor{UGLiBlue}{
Regrouper les expressions égales
\begin{multicols}{3}
	\begin{enumerate}[label=\textbullet]
		\item 	$2x^2-10x+12$
		\item 	$2x^2-14x+20$
		\item 	$2x^2-16x+30$
	\columnbreak
		\item 	$2(x-2)(x-5)$
		\item 	$2(x-2)(x-3)$
		\item 	$2(x-3)(x-5)$
	\columnbreak
		\item	$2\left(x-4\right)^2-2$
		\item	$2\left(x-\dfrac{7}{2}\right)^2-\dfrac{9}{2}$
		\item 	$2\left(x-\dfrac{5}{2}\right)^2-\dfrac{1}{2}$
	\end{enumerate}
\end{multicols}}

\begin{multicols}{2}
    On développe les formes factorisées :
\begin{enumerate}[label=\textbullet]
    \item 	\begin{tabbing}
        $2(x-2)(x-5)$   \=  $=2\left(x^2-5x-2x+10\right)$\\
        \>  $=2\left(x^2-7x+10\right)$\\
        \>  $=2x^2-14x+20$
    \end{tabbing}
	\item 	\begin{tabbing}
        $2(x-2)(x-3)$   \=  $=2\left(x^2-3x-2x+6\right)$\\
        \>  $=2\left(x^2-5x+6\right)$\\
        \>  $=2x^2-10x+12$
    \end{tabbing}
	\item 	\begin{tabbing}
        $2(x-3)(x-5)$   \=  $=2\left(x^2-5x-3x+15\right)$\\
        \>  $=2\left(x^2-8x+15\right)$\\
        \>  $=2x^2-16x+30$
    \end{tabbing}
\end{enumerate}

    On développe les formes canoniques :
\begin{enumerate}[label=\textbullet]
    \item \begin{tabbing}
        $2\left(x-4\right)^2-2$ \=  $=2\left(x^2-2\times x\times 4+4^2\right)-2$\\
        \>  $=2\left(x^2-8x+16\right)-2$\\
        \>  $=2x^2-16x+32-2$\\
        \>  $=2x^2-16x+30$
    \end{tabbing}

    \item \begin{tabbing}
        $2\left(x-\dfrac{7}{2}\right)^2-\dfrac{9}{2}$   \=  $=2\left(x^2-2\times x\times \dfrac{7}{2}+\left(\dfrac{7}{2}\right)^2\right)-\dfrac{9}{2}$\\[.5em]
        \>  $=2\left(x^2-7x+\dfrac{49}{4}\right)-\dfrac{9}{2}$\\[.5em]
        \>  $=2x^2-14x+\dfrac{49}{2}-\dfrac{9}{2}$\\[.5em]
        \>  $=2x^2-14x+\dfrac{40}{2}$\\[.5em]
        \>  $=2x^2-14x+20$
    \end{tabbing}

    \item \begin{tabbing}
        $2\left(x-\dfrac{5}{2}\right)^2-\dfrac{1}{2}$   \=  $=2\left(x^2-2\times x\times \dfrac{5}{2}+\left(\dfrac{5}{2}\right)^2\right)-\dfrac{1}{2}$\\[.5em]
        \>  $=2\left(x^2-5x+\dfrac{25}{4}\right)-\dfrac{1}{2}$\\[.5em]
        \>  $=2x^2-10x+\dfrac{25}{2}-\dfrac{1}{2}$\\[.5em]
        \>  $=2x^2-10x+\dfrac{24}{2}$\\[.5em]
        \>  $=2x^2-10x+12$
    \end{tabbing}
\end{enumerate}
\end{multicols}
D'où :
\begin{enumerate}[label=\textbullet]
    \item $2x^2-10x+12\ =\ 2(x-2)(x-3)\ =\ 2\left(x-\dfrac{5}{2}\right)^2-\dfrac{1}{2}$
    \item $2x^2-14x+20\ =\ 2(x-2)(x-5)\ =\ 2\left(x-\dfrac{7}{2}\right)^2-\dfrac{9}{2}$
    \item $=2x^2-16x+30\ = \ 2(x-3)(x-5)\ = \ 2\left(x-4\right)^2-2$
\end{enumerate}


\exo{}
\textcolor{UGLiBlue}{Suivre le modèle suivant pour écrire les polynômes suivants sous forme canonique.\\
\textbf{\textsc{Modèle :}}
\begin{tabbing}
	$f(x)$	\=	$=x^2+2x+7\qquad\qquad\qquad\qquad\qquad$				\=	on isole les 2 premiers termes ;\\
	\>	$=\left(x^2+2x\right)+7$						\>	entre parenthèses, on voit le début \\
	\>													\>	d'une identité remarquable ;\\
	\>	$=\left(x^2+2\times x\times 1\right)+7$			\>	il manque juste le $1^2$, donc on écrit\\
	\>	$=\left(x^2+2\times x\times 1\boxed{+1^2-1^2}\right)+7$						\>	et grâce à cette astuce\\
	\>	$=\left((x+1)^2-1\right)+7$						\>	rassemblons les constantes ;\\
	\>	$=(x+1)^2+6$									\>	et voilà une forme canonique !
\end{tabbing}
\begin{multicols}{3}
	\begin{enumerate}[label=\textbullet]
		\item 	$f(x)=x^2+2x-3$.
		\item 	$g(x)=x^2+4x+1$.
		\item 	$h(x)=x^2+6x-9$.
		\item 	$k(x)=x^2-2x-1$
		\item	$l(x)=x^2-8x+10$.
		\item 	$m(x)=x^2+7x-1$.
	\end{enumerate}
\end{multicols}}

Soit $x$ un nombre réel.
\begin{multicols}{2}
    \begin{enumerate}[label=\textbullet]
        \item \begin{tabbing}
            $f(x)$  \=  $=\textcolor{UGLiPurple}{x^2+2x}-3$\\
            \>  $=\textcolor{UGLiPurple}{x^2+2\times x\times 1 +1^2} \textcolor{UGLiOrange}{-1^2}-3$\\
            \>  $=\textcolor{UGLiPurple}{(x+1)^2}-1-3$\\
            \>  $=(x+1)^2-4$
        \end{tabbing}
    
        \item \begin{tabbing}
            $g(x)$  \=  $=\textcolor{UGLiPurple}{x^2+4x}+1$\\
            \>  $=\textcolor{UGLiPurple}{x^2+2\times x\times 2+2^2}\textcolor{UGLiOrange}{-2^2}+1$\\
            \>  $=\textcolor{UGLiPurple}{(x+2)^2}-4+1$\\
            \>  $=(x+2)^2-3$
        \end{tabbing}
    
        \item \begin{tabbing}
            $h(x)$  \=  $=\textcolor{UGLiPurple}{x^2+6x}-9$\\
            \>  $=\textcolor{UGLiPurple}{x^2+2\times x\times 3+3^2}\textcolor{UGLiOrange}{-3^2}-9$\\
            \>  $=\textcolor{UGLiPurple}{(x+3)^2}-9-9$\\
            \>  $=(x+3)^2-18$
        \end{tabbing}
    \columnbreak
        \item \begin{tabbing}
            $k(x)$  \=  $=\textcolor{UGLiPurple}{x^2-2x}-1$\\
            \>$=\textcolor{UGLiPurple}{x^2-2\times x\times 1+1^2}\textcolor{UGLiOrange}{-1^2}-1$\\
            \>  $=\textcolor{UGLiPurple}{(x-1)^2}-1-1$\\
            \>  $=(x-1)^2-2$
        \end{tabbing}
    
        \item \begin{tabbing}
            $l(x)$  \=  $=\textcolor{UGLiPurple}{x^2-8x}+10$\\
            \>  $=\textcolor{UGLiPurple}{x^2-2\times x\times 4+4^2}\textcolor{UGLiOrange}{-4^2}+10$\\
            \>  $=\textcolor{UGLiPurple}{(x-4)^2}-16+10$\\
            \>  $=(x-4)^2-6$
        \end{tabbing}
    
        \item \begin{tabbing}
            $m(x)$  \=  $=\textcolor{UGLiPurple}{x^2+7x}-1$\\
            \>  $=\textcolor{UGLiPurple}{x^2-2\times x\times \dfrac{7}{2}+\left(\dfrac{7}{2}\right)^2}\textcolor{UGLiOrange}{-\left(\dfrac{7}{2}\right)^2}-1$\\[.5em]
            \>  $=\textcolor{UGLiPurple}{\left(x-\dfrac{7}{2}\right)^2}-\dfrac{49}{4}-\dfrac{4}{4}$\\[.5em]
            \>  $=\left(x-\dfrac{7}{2}\right)^2-\dfrac{53}{4}$
        \end{tabbing}
    \end{enumerate}
\end{multicols}


\end{document}