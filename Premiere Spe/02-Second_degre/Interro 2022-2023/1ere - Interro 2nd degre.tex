\documentclass[a4paper,11pt,french]{article}
\usepackage[margin=2cm]{geometry}
\usepackage[thinfonts,latinmath]{uglix2}
\pagestyle{empty}
\begin{document}
\titreinterro{Interro cours - Second degré}{\premiere}{18/10/2022}

\textbf{Question de cours :}\\

On définit $p$ une fonction polynôme du second degré par : $\quad p(x)=ax^2+bx+c$.\\[.5em]
Donner la forme canonique de $p$ : \dotfill\\[.5em]
Donner le discriminant de $p$ : \dotfill\\

À quelle condition $p$ a-t-elle :
\begin{enumerate}[\textbullet]
	\item 	Aucune racine : \dotfill \\[.5em]
			Dans ce cas $p$ a-t-elle une forme factorisée ? Si oui, la donner.\\[.5em]
			\carreauxseyes{16}{1.6}\\
			
	\item 	Une seule racine : \dotfill\\[.5em]
			Quelle est-elle ? \dotfill\\[.5em]
			Dans ce cas $p$ a-t-elle une forme factorisée ? Si oui, la donner.\\[.5em]
			\carreauxseyes{16}{1.6}\\
			
	\item	Deux racines distinctes : \dotfill\\[.5em]
			Quelles sont-elles ? \dotfill\\[.5em]
			Dans ce cas $p$ a-t-elle une forme factorisée ? Si oui, la donner.\\[.5em]
			\carreauxseyes{16}{1.6}
\end{enumerate}

\newpage
\textbf{Application :}\\

On définit $f$ par $f(x)=3x^2+5x-2$.\\
$f$ admet-elle une forme factorisée ? Si oui la donner.\\

\carreauxseyes{16.8}{16}
\end{document}
