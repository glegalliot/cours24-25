\documentclass[a4paper,10pt,exos]{nsi} 
\pagestyle{empty}


%%%%%%%%%%%%%%%%%%%%%%%%%%%%%%%%%%%%%%%%%%%%%%%%%
% Olympiades académisues 2021 - Exercice 1 (tous les candidats)
%%%%%%%%%%%%%%%%%%%%%%%%%%%%%%%%%%%%%%%%%%%%%%%%%

\begin{document}
    \classe{Olympiades}
    \titre{Figure $(1,a)$}
    \maketitle

    \dleft{12cm}{Soit $a$ un nombre réel donné. On suppose que $a > 1$ dans toute la suite du sujet.\\
    Une figure $(1, a)$ est un ensemble fini de points du plan tels que la distance entre deux d’entre eux vaut $1$ ou $a$.\\}
    {\includegraphics[width=4.5cm]{polygons.jpg}}

    {\titlefont\color{UGLiBlue}{Partie A}}\\
    \question Donner un exemple de figure $\left(1,\dfrac{3}{2}\right)$ à trois points.\\[.5em]
    \question Existe-t-il une figure $(1, a)$ à trois points telle que ces trois points soient les sommets d’un triangle rectangle ?\\[.5em]
    \question Donner un exemple de figure $\left(1, \sqrt{2} \right)$ à quatre points.\\

    Le but du problème est d’obtenir une classification de l’ensemble des figures $(1, a)$ en fonction du nombre de ses points et de leurs positions relatives.\\

    \question On considère une figure $(1, a)$ à trois points. Quelle est la nature du triangle formé par ces points ? On décrira les différents cas possibles.\\

    {\titlefont\color{UGLiBlue}{Partie B}}\\
    Dans cette partie, ABC est un triangle équilatéral de coté 1.\\
    On suppose que D est un point tel que les points A, B, C et D forment une figure $(1, a)$.\\[.5em]
    \question Montrer qu’il est impossible que les distances de D à chacun des trois autres points soient toutes égales à 1.\\[.5em]
    \question Est-il possible que la distance entre D et A soit égale à $a$ et que toutes les autres distances entre deux points distincts de la figure valent 1 ? Si oui, indiquer la ou les valeur(s) correspondante(s) de $a$.\\[.5em]
    \question Comment placer D de telle sorte que la distance entre A et D soit égale à $a$, et, que toutes les autres distances entre deux points distincts de la figure valent 1 sauf une ? Déterminer la ou les valeur(s) correspondantes pour $a$.\\[.5em]
    \question Peut-on placer D de telle sorte que trois distances, exactement, entre deux points distincts de la figure soient égales à $a$ ? Si oui, calculer la ou les valeur(s) correspondante(s) de $a$.\\

    {\titlefont\color{UGLiBlue}{Partie C}}\\
    Dans cette partie, ABC est un triangle isocèle en B tel que $\text{AB} = \text{BC} = 1$ et $\text{AC} = a$.\\
    On suppose que D est un point tel que les points A, B, C et D forment une figure $(1, a)$.\\[.5em]
    \question Déterminer les nouvelles figures $(1, a)$ à quatre points A, B, C et D en calculant $a$ à chaque fois.\\

    {\titlefont\color{UGLiBlue}{Partie D}}\\
    Dans cette partie, ABC est un triangle équilatéral de coté $a$.\\
    On suppose que D est un point tel que les points A, B, C et D forment une figure $(1, a)$.\\[.5em]
    \question Déterminer les nouvelles figures $(1, a)$ à quatre points A, B, C et D en calculant $a$ à chaque fois.\\

    {\titlefont\color{UGLiBlue}{Partie E}}\\
    \question Montrer qu’il existe une figure $(1, a)$ à cinq points.


\end{document}    