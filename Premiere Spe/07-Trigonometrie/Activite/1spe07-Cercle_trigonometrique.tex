\documentclass[a4paper,11pt,exos]{nsi} % COMPILE WITH DRAFT
\usepackage{pifont}
\usepackage{fontawesome5}
\usepackage{hyperref}

\usepackage{yhmath}

\pagestyle{empty}

\begin{document}
\classe{\premiere spé}
\titre{Cercle trigonométrique}
\maketitle

$\repaff$ est un repère orthonormé du plan.\\
Le sens direct est le sens trigonométrique (sens contraire des aiguilles d'une montre).\\
\begin{center}
	\begin{tikzpicture}[scale = 4]	
		\draw 	(0,0) circle(1);
		\draw   (0,0)--(000:1)\ball;
		\draw   (0,0)--(030:1)\ball;
		\draw   (0,0)--(045:1)\ball;
		\draw   (0,0)--(060:1)\ball;
		\draw   (0,0)--(090:1)\ball;
		\draw   (0,0)--(120:1)\ball; 
		\draw   (0,0)--(135:1)\ball; 
		\draw   (0,0)--(150:1)\ball;
		\draw   (0,0)--(180:1)\ball;
		\draw   (0,0)--(210:1)\ball;
		\draw   (0,0)--(225:1)\ball;
		\draw 	(0,0)--(240:1)\ball;
		\draw   (0,0)--(270:1)\ball;
		\draw   (0,0)--(300:1)\ball;
		\draw   (0,0)--(315:1)\ball;
		\draw   (0,0)--(330:1)\ball;
		
		\draw   (000:1.2) node{I};
		\draw   (030:1.2) node{A};
		\draw   (045:1.2) node{B};
		\draw   (060:1.2) node{C};
		\draw   (090:1.2) node{J};
		\draw   (120:1.2) node{D}; 
		\draw   (135:1.2) node{E};
		\draw   (150:1.2) node{F};
		\draw   (180:1.2) node{K};
		\draw   (210:1.2) node{G};
		\draw   (225:1.2) node{H};
		\draw 	(240:1.2) node{M};
		\draw   (270:1.2) node{L};
		\draw   (300:1.2) node{P};
		\draw   (315:1.2) node{Q};
		\draw   (330:1.2) node{R};
		\draw (0,0) node[below left]{O};
	\end{tikzpicture}
\end{center}
\subsection*{En enroulant dans le sens direct.}
\begin{enumerate}
	\item Calculer le périmètre du cercle.
	\item En déduire les longueurs respectives des arcs $\wideparen{IJ}$, $\wideparen{IK}$ et $\wideparen{IL}$.
	\item Donner, en degrés, les mesures des angles suivants : $\widehat{IOA}$, $\widehat{IOB}$, 
	$\widehat{IOC}$,$\dots$,$\widehat{IOR}$.\\
	En déduire les longueurs respectives des arcs $\wideparen{IA}$, $\wideparen{IB}$,$\dots$,$\wideparen{IR}$.
	\begin{center}
        \tabstyle[UGLiOrange]
 \begin{tabular}{|c|c|c|c|c|c|c|c|c|}
 \hline
 \ccell Angle & $\widehat{IOA}$ & $\widehat{IOB}$ & $\widehat{IOC}$ & $\widehat{IOJ}$ & $\widehat{IOD}$ & $\widehat{IOE}$ & $\widehat{IOF}$ & $\widehat{IOK}$ \\\hline
 \ccell Mesure (degré) & & & & & & & \\\hline
 \ccell Arc & $\wideparen{IA}$ & $\wideparen{IB}$ & $\wideparen{IC}$ & $\wideparen{IJ}$ & $\wideparen{ID}$ & $\wideparen{IE}$ & $\wideparen{IF}$ & $\wideparen{IK}$ \\\hline
 \ccell Longueur & & & & & & & \\\hline
 \ccell Angle & $\widehat{IOG}$ & $\widehat{IOH}$ & $\widehat{IOM}$ & $\widehat{IOL}$ & $\widehat{IOP}$ & $\widehat{IOQ}$ & $\widehat{IOR}$ & \bcell \\\hline
 \ccell Mesure (degré) & & & & & & & \bcell\\\hline
 \ccell Arc  & $\wideparen{IG}$ & $\wideparen{IH}$ & $\wideparen{IM}$ & $\wideparen{IL}$ & $\wideparen{IP}$ & $\wideparen{IQ}$ & $\wideparen{IR}$ & \bcell\\\hline
 \ccell Longueur & & & & & & & \bcell \\\hline
 \end{tabular}
    \end{center}
	\item Placer en \textsc{rouge} ces nombres à coté des points associés du cercle.
\end{enumerate}

\subsection*{Et maintenant dans l'autre sens.}
\begin{enumerate}
	\item En parcourant l'arc $\wideparen{IR}$, à quel nombre négatif peut-on associer $R$ ?
	\item En raisonnant de même, placer en \textsc{vert} les nombres (négatifs) associés aux points du cercle.
\end{enumerate}

\subsection*{À quelques tours près, dans un sens ou dans l'autre.}
\begin{enumerate}
	\item En parcourant l'arc $\wideparen{IA}$ dans le sens direct en faisant trois tours « de trop » , à quel autre nombre peut-on associer $R$ ?
	\item En raisonnant de même (avec des tours en plus dans un sens ou dans l'autre), placer en \textsc{bleu} d'autres nombres associés aux points du cercle.
\end{enumerate}

\subsection*{Prenons le problème à l'envers}
Indiquer à quels points sont associés les nombres suivants :
\begin{multicols}{5}
	\begin{enumerate}[label=\textbullet]
		\item $\dfrac{37\pi}{4}$
		\item $\dfrac{23\pi}{6}$
		\item $\dfrac{19\pi}{3}$
		\item $\dfrac{41\pi}{2}$
		\item $-\dfrac{29\pi}{6}$
		\item $-\dfrac{101\pi}{4}$
		\item $\dfrac{59\pi}{3}$
		\item $-\dfrac{31\pi}{2}$
		\item $-\dfrac{12\pi}{8}$
		\item $\dfrac{57\pi}{12}$
	\end{enumerate}
\end{multicols}


\end{document}