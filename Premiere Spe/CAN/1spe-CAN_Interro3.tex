\documentclass[a4paper,11pt,landscape,exos]{nsi} % COMPILE WITH DRAFT
\usepackage{hyperref}

\pagestyle{empty}
\setlength{\columnseprule}{0.5pt}
\setlength{\columnsep}{1cm}
\begin{document}

\begin{multicols}{2}
    \textcolor{UGLiBlue}{NOM, Prénom :\\
    Date : vendredi 25/04/2025}\\
\classe{\premiere spe}
\titre{\includegraphics[width=3cm]{CAN.png} Interrogation 3}
\maketitle

\begin{enumerate}[itemsep=1em]
	\item $0{,}9 \times 6$ 
	\item Forme développée et réduite de $(x-6)(x+5)$
	\item Médiane de la série :\\$16$\,;\,$24$\,;\,$7$\,;\,$11$\,;\,$2$ 
	\item Signe de  $(-3)^{-5}$ 
    
    	$\square\;$ Positif\qquad $\square\;$ Négatif\qquad 
	\item  Factoriser  $x^2-81$.
	\item $6{,}7$  h $ = 6$ h $\ldots$ min
	\item La moyenne de $6$, $10$, $14$ et d'un nombre inconnu $n$ est égale à $12$.\\$n=\ldots$
	\item Rémi a couru $3$ km en $10$ minutes, sa vitesse moyenne est de   $\ldots$ km/h
	\item Soit $f$ : $x\longmapsto \dfrac{1}{x^6}$\\$f'(x)=\ldots$
	\item Solution(s) de l'équation  $x^2-4\,900=0$
	\item  $600 + \cos(33\pi)$
	\item $\renewcommand{\arraystretch}{1}
\begin{array}{|c|c|c|c|c|}
\hline
\cellcolor{lightgray} x_i & \cellcolor{lightgray} -1 & \cellcolor{lightgray} 0 & \cellcolor{lightgray} 1 & \cellcolor{lightgray} 2\\
\hline
 \rule[-2ex]{0pt} {6ex}\ \cellcolor{lightgray} P(X=x_i) & \dfrac{1}{9} & \dfrac{2}{9} & \dfrac{2}{9} & \ldots\\
\hline
 \end{array}
\renewcommand{\arraystretch}{1}$


\medskip
 $P(X=2)=$ $\ldots$
	\item $S$ est l'ensemble des  solutions  de l'inéquation
             $-2\,025(x+2\,025)^2 \geqslant 0$.\\$S=\ldots$
	\item $A(4\,000\,;\,-10)$ et $B(10\,;\,4\,020)$\\
         Déterminer les coordonnées de $M$, milieu de $[AB]$.\\$M(\ldots;\ldots)$
	\item $f(x)=x^2-8x-8$ \\
           La représentation graphique $\mathcal{C}_f$ de la fonction $f$ a pour axe de symétrie la droite d'équation $x=$$\ldots$      
	
\end{enumerate}


\vfill\null
\end{multicols}

\newpage

\begin{multicols}{2}
    \classe{\premiere spe}
\titre{\includegraphics[width=3cm]{CAN.png} Corrigé Interro 3}
\maketitle

\begin{enumerate}[itemsep=1em]
    \item $0{,}9 \times 6={\color[HTML]{f15929}\boldsymbol{5{,}4}}$
\item $\begin{aligned}
      (x-6)(x+5)&=x^2+5x-6x-30\\
      &={\color[HTML]{f15929}\boldsymbol{x^2-x-30}}
      \end{aligned}$\\Le terme en $x^2$ vient de $x\times x=x^2$.\\Le terme en $x$ vient de la somme de $x \times 5$ et de $-6 \times x$.\\Le terme constant vient de $-6\times 5= -30$.
\item On ordonne la série :  $2$\,;\,$7$\,;\,$11$\,;\,$16$\,;\,$24$.\\
      La série comporte $5$ valeurs donc la médiane est la troisième valeur : ${\color[HTML]{f15929}\boldsymbol{11}}$.
\item $(-3)^{-5}=\dfrac{1}{(-3)^{5}}$\\
     Comme  $(-3)^{5}$ est  négatif (puissance impaire d'un nombnre négatif), on en déduit que  $\dfrac{1}{(-3)^{5}}$ est négatif.\\
    Ainsi, $(-3)^{-5}$ est {\bfseries \color[HTML]{f15929}négatif}.
\item On utilise l'égalité remarquable ${\color{red} a}^2-{\color{blue} b}^2=({\color{red} a}-{\color{blue} b})({\color{red} a}+{\color{blue} b})$ avec $a={\color{red} x}$  et $b={\color{blue} 9}$.\\$\begin{aligned}
 x^2-81&=\underbrace{{\color{red} x}^2-{\color{blue} 9}^2}_{a^2-b^2}\\
 &=\underbrace{({\color{red} x}-{\color{blue} 9})({\color{red} x}+{\color{blue} 9})}_{(a-b)(a+b)}
 \end{aligned}$ \\
    Une expression factorisée de $x^2-81$ est ${\color[HTML]{f15929}\boldsymbol{(x-9)(x+9)}}$.
\item $6{,}7 = 6 \text{ h } + 0{,}7\times 60 \text{ min } = 6 \text{ h }  {\color[HTML]{f15929}\boldsymbol{42}} \text{ min}$
\item Puisque la moyenne de ces quatre nombres est $12$, la somme de ces quatre nombres est $4\times 12=48$.\\
             La valeur de $n$ est donnée par :  $48-6-10-14={\color[HTML]{f15929}\boldsymbol{18}}$.
\item $10\times 6= 60$ min $=1$ h\\
    Rémi court $6$ fois plus de km en $1$ heure.\\
   $3\times 6=18$\\
   Rémi court à ${\color[HTML]{f15929}\boldsymbol{18}}$ km/h.
\item D'après le cours, si $f=\dfrac{1}{u}$ alors $f'=\dfrac{-u'}{u^2}$.\\
    $f'(x)=\dfrac{-6x^{5}}{x^{12}}={\color[HTML]{f15929}\boldsymbol{-\dfrac{6}{x^{7}}}}$.
\item Puisque $4900>0$, l'équation a deux solutions :  $-\sqrt{4\,900}$ et $\sqrt{4\,900}$, soit $-70$ et $70$.\\
    Ainsi, $S={\color[HTML]{f15929}\boldsymbol{\{-70\,;\,70\}}}$.
\item Si $n$ est pair $\cos(n\pi)=1$ et si $n$ est impair, $\cos(n\pi)=-1$.\\$600 + \cos(33\pi)=600 + (-1)={\color[HTML]{f15929}\boldsymbol{599}}$
\item  La somme des probabilités doit être égale à $1$.\\
    Ainsi, $P(X=2)=1-\dfrac{1}{9}-\dfrac{2}{9}-\dfrac{2}{9}={\color[HTML]{f15929}\boldsymbol{\dfrac{4}{9}}}$.
\item Pour tout réel $x$, $-2\,025(x+2\,025)^2$ est négatif et s'annule en $-2\,025$.\\
               Ainsi, l'ensemble $S$ des solutions de l'inéquation est  ${\color[HTML]{f15929}\boldsymbol{\{-2\,025\}}}$.
\item Les coordonnées du milieu sont données par la moyenne des abscisses et la moyenne des ordonnées : \\
         $x_M=\dfrac{4\,000+10}{2}={\color[HTML]{f15929}\boldsymbol{2\,005}}$ et $y_M=\dfrac{-10+4\,020}{2}={\color[HTML]{f15929}\boldsymbol{2\,005}}$.\\
         Ainsi,  $M({\color[HTML]{f15929}\boldsymbol{2\,005\,;\,2\,005}})$.
\item $f$ est une fonction polynôme du second degré écrite sous forme développée $ax^2+bx+c$.\\
       Le sommet de la parabole a pour abscisse $-\dfrac{b}{2a}$.\\
           L'axe de symétrie a donc pour équation $x=-\dfrac{b}{2a}$. \\
       On obtient alors  $x=-\dfrac{-8}{2\times 1}$, soit $x=\dfrac{8}{2}$ ou encore  $x={\color[HTML]{f15929}\boldsymbol{4}}$.
\end{enumerate}
\end{multicols}

\end{document}