\documentclass[a4paper,11pt,eval]{nsi} 

\usepackage{pifont}
\usepackage{fontawesome5}
%\pagestyle{empty}


\newcounter{exoNum}
\setcounter{exoNum}{0}
%
\newcommand{\exo}[1]
{
	\addtocounter{exoNum}{1}
	{\titlefont\color{UGLiBlue}\Large Exercice\ \theexoNum\ \normalsize{#1}}\smallskip	
}



\begin{document}



\textcolor{UGLiBlue}{Jeudi 15/05/2025}\\
\classe{\premiere spé}
\titre{Interrogation de cours - Sujet A}
\maketitle
\begin{center}
	Calculatrice interdite
\end{center}

On considère $f$ , la fonction définie sur $\R$ par $\quad f(x)=4x^3-7x^2+30x-5$.\\
Déterminer la fonction dérivée de $f$.\\[.5em]
\carreauxseyes{16.8}{3.2}\\


On considère $g$ la fonction définie par $\quad g(x)=\dfrac{3x^2+2x+7}{x-1}$.\\[.5em]
Quel est l'ensemble de définition de $g$ ?\\
Déterminer la fonction dérivée de $g$.\\[.5em]
\carreauxseyes{16.8}{11.2}\\


On considère la fonction $h$ définie par $\quad h(x)=\sqrt{x}(x^2+1)$.\\[.5em]
Quel est l'ensemble de définition de $h$ ?\\
Déterminer la fonction dérivée de $h$.\\[.5em]
\carreauxseyes{16.8}{20}


\newpage
\textcolor{UGLiBlue}{Jeudi 15/05/2025}\\
\classe{\premiere spé}
\titre{Interrogation de cours - Sujet B}
\maketitle
\begin{center}
	Calculatrice interdite
\end{center}

On considère $f$ , la fonction définie sur $\R$ par $\quad f(x)=5x^3+3x^2-20x+7$.\\
Déterminer la fonction dérivée de $f$.\\[.5em]
\carreauxseyes{16.8}{3.2}\\


On considère $g$ la fonction définie par $\quad g(x)=\dfrac{2x^2-3x+5}{x+1}$.\\[.5em]
Quel est l'ensemble de définition de $g$ ?\\
Déterminer la fonction dérivée de $g$.\\[.5em]
\carreauxseyes{16.8}{11.2}\\


On considère la fonction $h$ définie par $\quad h(x)=\sqrt{x}(x^2-1)$.\\[.5em]
Quel est l'ensemble de définition de $h$ ?\\
Déterminer la fonction dérivée de $h$.\\[.5em]
\carreauxseyes{16.8}{20}
\end{document}