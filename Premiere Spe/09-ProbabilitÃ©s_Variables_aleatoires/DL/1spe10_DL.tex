\documentclass[a4paper,11pt,exos]{nsi} % COMPILE WITH DRAFT
\usepackage{pifont}
\usepackage{fontawesome5}
\usepackage{pst-node,pst-plot,pst-tree}

\pagestyle{empty}

\begin{document}
\classe{\premiere spé}
\titre{DL}
\maketitle

Dans une école d'ingénieurs, certains étudiants s'occupent de la gestion des associations
comme par exemple le BDS (bureau des sports).

Sur les cinq années d'études, le cycle « licence »  dure les trois premières années, et les deux
dernières années sont celles du cycle de «   spécialisation ».

On constate que, dans cette école, il y a 40\,\% d'étudiants dans le cycle « licence » et 60\,\%
dans le cycle de « spécialisation ».

\begin{list}{\textbullet}{}
\item Parmi les étudiants du cycle « licence », 8\,\% sont membres du BDS ;
\item Parmi les étudiants du cycle de « spécialisation », 10\,\% sont membres du BDS.
\end{list}

On considère un étudiant de cette école choisi au hasard, et on considère les évènements suivants:

\begin{list}{}{}
\item $L$: « L'étudiant est dans le cycle  licence »; $\overline{L}$ est son évènement contraire.
\item $B$: « L'étudiant est membre du BDS »; $\overline{B}$ est son évènement contraire.
\end{list}
 
La probabilité d'un évènement $A$ est notée $P(A)$.\\


\textbf{Partie A}

\medskip

\begin{enumerate}
\item Recopier et compléter l'arbre pondéré modélisant la situation.

\begin{center}
\bigskip
  \pstree[treemode=R,nodesepA=0pt,nodesepB=4pt,levelsep=2.5cm]{\TR{}}
 {
 	\pstree[nodesepA=4pt]{\TR{$L$}\naput{$0,4$}}
 	  { 
 		  \TR{$B$}\naput{$\cdots$}
 		  \TR{$\overline{B}$}\nbput{$\cdots$}	   
 	  }
 	\pstree[nodesepA=4pt]{\TR{$\overline{L}$}\nbput{$\cdots$}}
 	  {
 		  \TR{$B$}\naput{$\cdots$}
 		  \TR{$\overline{B}$}\nbput{$\cdots$}	   
     }
}
\bigskip
\end{center}

\item Calculer la probabilité que l'étudiant choisi soit en cycle « licence » et membre du BDS.
\item En utilisant l'arbre pondéré, montrer que $P(B)=0,092$.
\end{enumerate}

\medskip

\textbf{Partie B}

\medskip

Le BDS décide d'organiser une randonnée en montagne. Cette sortie est proposée à tous les étudiants de cette école mais le prix qu'ils auront à payer pour y participer est variable. Il est de 60 € pour les étudiants qui ne sont pas membres du BDS, et de 20 € pour les étudiants qui sont membres du BDS.

On désigne par $X$ la variable aléatoire donnant la somme à payer pour un étudiant qui désire
faire cette randonnée.

\begin{enumerate}
\item Quelles sont les valeurs prises par $X$ ?
\item Donner la loi de probabilité de $X$, et calculer l'espérance de $X$.
\end{enumerate}

\newpage

\classe{\premiere spé}
\titre{Corrigé du DL de probabilités}
\maketitle



\textbf{Partie A}

\medskip

\begin{enumerate}
\item On complète l'arbre pondéré modélisant la situation.

\begin{center}
\bigskip
  \pstree[treemode=R,nodesepA=0pt,nodesepB=4pt,levelsep=2.5cm,nrot=:U]{\TR{}}
 {
 	\pstree[nodesepA=4pt]{\TR{$L$}\naput{$0,4$}}
 	  { 
 		  \TR{$B$}\naput{$\blue 0,08$}
 		  \TR{$\overline{B}$}\nbput{$\blue 0,92$}	   
 	  }
 	\pstree[nodesepA=4pt]{\TR{$\overline{L}$}\nbput{$\blue 0,6$}}
 	  {
 		  \TR{$B$}\naput{$\blue 0,10$}
 		  \TR{$\overline{B}$}\nbput{$\blue 0,90$}	   
     }
}
\bigskip
\end{center}

\item La probabilité que l'étudiant choisi soit en cycle « licence » et membre du BDS est:

\begin{tabbing}
	$P(L\cap B)$ \=$=P(L) \times P_l(B)$\\
	\>	$=0,4\times 0,08$\\ 
	\>	$= 0,032$
\end{tabbing}


\item \begin{tabbing}
	$P(B)$\=	$=  P(L\cap B) + P(\overline L\cap B)$\\
	\>	$ = 0,032 + 0,6\times 0,1$\\
	\>	$ = 0,092$.
\end{tabbing}

\end{enumerate}

\medskip

\textbf{Partie B}

\medskip



On désigne par $X$ la variable aléatoire donnant la somme à payer (en euros) pour un étudiant qui désire
faire cette randonnée.

\begin{enumerate}
\item Les valeurs prises par $X$ sont 20 et 60.

\item% Donner la loi de probabilité de $X$, et calculer l'espérance de $X$.
La somme à payer est de 20 € si l'étudiant est membre du BDS, c'est-à-dire avec une probabilité de $0,092$, ou de 50 € si l'étudiant n'est pas membre du BDS, c'est-à-dire avec une probabilité de $1-0,092 = 0,908$. D'où la loi de probabilité de la variable aléatoire $X$:

\begin{center}
{\renewcommand{\arraystretch}{1.5}
\begin{tabular}{|c|*{2}{>{\centering\arraybackslash}p{1cm}|}}
\hline
$x_i$  & 20 & 60\\
\hline
$p_i = P(X=x_i)$ & $0,092$  & $0,908$\\
\hline
\end{tabular}}
\end{center}

L'espérance mathématique de la variable aléatoire $X$ est:

$E(X) = \sum (x_i\times p_i) = 20\times 0,092 + 60\times 0,908 = 56,32$.

\end{enumerate}

\end{document}