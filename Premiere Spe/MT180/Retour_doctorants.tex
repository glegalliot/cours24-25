\documentclass[a4paper,11pt,exos]{nsi} % COMPILE WITH DRAFT
\usepackage{pifont}
\usepackage{fontawesome5}
\geometry{margin=2cm}




\begin{document}
\classe{\premiere spé}
\titre{MT 180 - Retour aux doctorants }
\maketitle

Le concours \textit{Ma thèse en 180 secondes} est inspiré du concours \textit{Three minute thesis} (3MTMC) qui a eu lieu pour la
première fois en 2008 à l’Université du Queensland, en Australie.\\

\dleft{3cm}{\includegraphics[width=3cm]{MT180surCPUCNRS_NOIR-1024x1024.jpg}}
{Le concours \textbf{Ma thèse en 180 secondes} permet à des \textbf{doctorantes et doctorants} de présenter leur sujet de
recherche en termes simples à un auditoire profane et diversifié. Chaque participant doit faire, en \textbf{trois minutes},
un exposé clair, concis et surtout convaincant autour de son projet de recherche, le tout avec l’appui d’une seule
diapositive.}

Votre travail consiste à visionner une vidéo de présentation d'une thèse en 180 secondes et à lui faire un retour sur les points suivants :

\subsection*{Talent d'orateur}
\begin{enumerate}[label=\textbullet]
    \item Le participant a-t-il réussi à vous captiver ?
    \item La personnalité du participant ressort-elle de cette présentation ?
\end{enumerate}

\subsection*{Médiation du sujet}
\begin{enumerate}[label=\textbullet]
    \item Le participant a-t-il rendu compréhensible les enjeux, les méthodes et les éventuels résultats de ses recherches ?
    \item Si diapositive utilisée, apporte-t-elle une plus-value à la médiation du sujet ?
\end{enumerate}

Pour chacun de ces points, préciser :
\begin{enumerate}[label=\textbf{\ding{241}}]
    \item Les points forts de la présentation.
    \item Vos conseils pour améliorer la présentation.
\end{enumerate}
\end{document}