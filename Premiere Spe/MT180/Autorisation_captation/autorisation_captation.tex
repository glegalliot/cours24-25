\documentclass[10pt, a4paper,article]{nsi}
\pagestyle{empty}
\geometry{top=1cm,bottom=1cm}
\begin{document}
\picright{0.15}{img/logo}{\textbf{\'Etablissement :} Lycée Rabelais\\
\textbf{Adresse :} 8 rue Rabelais - BP 2255 - 22022 Saint-Brieuc\\
\textbf{Téléphone :} 0296683270\\
\textbf{Email :} ce.0220056s@ac-rennes.fr\\
}

\scriptsize
La présente demande est destinée à recueillir le consentement et les autorisations nécessaires dans le cadre de l'enregistrement, la captation, l'exploitation et l'utilisation de l'image des élèves (photographie, voix) quel que soit le procédé envisagé. Elle est formulée dans le cadre du projet spécifié ci-dessous et explique les objectifs de celui-ci aux élèves et à leurs responsables légaux.\\

\textit{Vu le Code Civil (article 9), la Déclaration universelle des droits de l'homme (article 12), la Convention européenne des droits de l'homme (article 8) et la Charte des droits fondamentaux de l'Union européenne (article 7).}

\textit{Vu le règlement général européen N°2016/679 du 27 avril 2016 relatif à la protection des personnes physiques à l'égard du traitement des données à caractère personnel et à la libre circulation des données (RGPD) et à la loi n°78-17 du 06 janvier 1978 modifiée le 29 juin 2018 relative à l'informatique, aux fichiers et aux libertés.}\\

\normalsize

\begin{center}
    \Large\color{UGLiBlue}\titlefont Autorisation de captation
\end{center}

Dans le cadre du projet « Ma Thèse en 180 secondes » mené en enseignement de spécialité Mathématiques, Madame Le Galliot, enseignante, souhaite réaliser des captations audio et / ou vidéo des élèves. \\
L'objectif est de permettre aux élèves de progresser à l'oral et d'échanger avec des doctorants sur un même format vidéo.\\


Les captations vidéos sont réalisées sur un dispositif d'enregistrement non connecté et ne seront jamais publiées sur Internet. Elles seront effacées à la fin de l'année scolaire et seuls l'enseignante de mathématiques, le doctorant et Julien Le Bonheur (responsable MT180 à l'université de Rennes) y auront accès.\\
Les captations audio sont réalisés \textit{via} \texttt{monoral.net}, application en ligne française, libre et en accord avec le Règlement Général sur la Protection des Données.
\subsection*{Consentement de l'élève}

\dleft{14cm}
{On m'a expliqué et j'ai compris ce que sont et à quoi servent les captations.\\
Je suis d'accord pour que l'on enregistre ma voix.\\
Je suis d'accord pour que l'on enregistre mon image.\\
NOM Prénom de l'élève :}
{
OUI / NON\\
OUI / NON\\
OUI / NON\\
Signature:
}

\subsection*{Autorisation parentale}
Je (Nous) soussigné(e)(s) : {\scriptsize(NOM Prénom)}\dotfill  \\
Demeurant : \dotfill\\
Et  {\scriptsize(Nom Prénom)}\dotfill\\
Demeurant: {\scriptsize(adresse à préciser si différente)}\dotfill\\
Agissant en qualité de représentant(s) légal(aux) de : {\scriptsize(NOM Prénom de l'élève)}\dotfill\\

Reconnais(sons) être entièrement investi(s) de mes(nos) droits civils à son égard. Je(nous) reconnais(sons) expressément que le mineur que je(nous) représente(ons) n'est lié par aucun contrat exclusif pour l'utilisation de son image et/ou de sa voix, voire de son nom et\\

$\square$  autorise(ons) la captation de l'image / de la voix de l'enfant et l'utilisation qui en sera faite par son établissement scolaire.\\
    
$\square$ n'autorise(ons) pas la captation de l'image / de la voix de l'enfant. Merci alors d'écrire lisiblement le mot « REFUS » : \\


Fait à \hspace*{5cm} le \hspace*{7.5cm} Signature : 

\end{document}
