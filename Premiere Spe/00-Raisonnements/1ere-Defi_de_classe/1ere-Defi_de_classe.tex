\documentclass[a4paper,11pt,exos]{nsi} % COMPILE WITH DRAFT
\usepackage{pifont}
\pagestyle{empty}

%\pb{intitulé}
%
\newcounter{pbNum}
\setcounter{pbNum}{0}
%
\newcommand{\pb}[1]
{
	\addtocounter{pbNum}{1}
	{\titlefont\color{UGLiBlue}\Large Problème\ \thepbNum\ \normalsize{#1}}\smallskip	
}

\begin{document}
\classe{\premiere spé}
\titre{Défi de classe}
\maketitle

\pb{}\\
Voici le début d'une suite.
$$1\ ;\quad 3\ ;\quad 7\ ;\quad 15\ ;\quad 31\ ;\quad 63\ ;\quad 127$$ 
Quel est le nombre suivant ?\\


\pb{ Vrai ou faux ?}\\
Dans un repère orthonormé, on donne :
$$\pc{A}{-2}{3}\ ;\quad \pc{B}{5}{-1}\ ;\quad \pc{C}{2}{-4}\ ;\quad \pc{D}{-5}{0}.$$
$ABCD$ est un parallélogramme.\\

\pb{ Vrai ou faux ?}\\
« Tous les mathématiciens célèbres sont des hommes. »\\

\pb{}\\
On construit des carrés. On a :
\begin{multicols}{3}
    \includegraphics[width=3cm]{carre1}\\
    1 carré à l'épape 1 ;\\
    \includegraphics[width=3cm]{carre2}\\
    8 carrés à l'épape 2 ;\\
    \includegraphics[width=3cm]{carre3}\\
    ... carrés à l'épape 3 ;
\end{multicols}
Combien a-t-on de carrés à l'étape 10 ?\\

\pb{}\\
Quel est l’ensemble des solutions de l’équation $x^2-4=0$ ?\\

\pb{}\\
$a, b$ et $c$ sont trois nombres relatifs non nuls dont le produit est négatif.\\
$a$ est l’inverse de $c$. Quel est le signe de $b$ ?\\

\pb{ Vrai ou faux ?}\\
« La somme des chiffres d’un nombre premier n’est jamais un multiple de 3. »\\

\dleft{7cm}
{
    \pb{}\\
    Voici le début d'une suite :\\[.5em]
    Quel est le polygone suivant ?\\
}
{\includegraphics[width=8cm]{polynomes.png}}



\pb{ Vrai ou faux ?}\\
« Tout nombre positif est plus grand que sa racine carrée. »\\

\pb{}\\
IJK a un périmètre de 18 cm et ses côtés ont une longueur entière.\\
Si IJ = 7 cm, quelle peut être la longueur de JK et KI ?\\
Y-a-t-il plusieurs cas possibles ? Si oui, les donner tous.\\

\pb{ Vrai ou faux ?}\\
« La somme des chiffres d’un nombre premier n’est jamais un multiple de 9. »\\

\pb{ Vrai ou faux ?}\\
« Si un nombre est divisible par 6 alors son carré est divisible par 9. »\\

\pb{ Vrai ou faux ?}\\
Dans un repère orthonormé, on donne :
$$\pc{E}{-2}{-1}\ ;\quad \pc{F}{0}{2}\ ;\quad \pc{G}{4}{0}\ ;\quad \pc{H}{6}{3}.$$
$EFGH$ est un parallélogramme.\\

\pb{}\\
On donne le script Python suivant :
\begin{pyc}
    \begin{minted}{python}
n = input('Choisir un nombre entier compris entre 1 et 10')
for i in range(2) :
    if n % 2 == 0 :   # a % b donne le reste de la division euclidienne de a par b
        n = n/2
    else :
        n = 3*n+1
print(n)
    \end{minted}
\end{pyc}

Le programme affiche 11.\\
Quel est le nombre qui a été choisi au départ ?\\

\pb{}\\
Quel est l’ensemble des solutions de l’inéquation $4x-3>5x+7$ ?\\

\pb{ Vrai ou faux ?}\\
L’aire d’un carré est proportionnelle à la longueur de son côté.



\end{document}