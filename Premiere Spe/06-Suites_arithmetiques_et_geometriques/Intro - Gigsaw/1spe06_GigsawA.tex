\documentclass[a4paper,11pt,exos]{nsi} % COMPILE WITH DRAFT


\pagestyle{empty}
\begin{document}


\classe{\premiere spe}
\titre{Suites arithmétiques et géométriques}
\maketitle
\section*{Partie A}

\exo{ : Compléter les suites logiques}
\renewcommand{\arraystretch}{1.8}
\begin{tabular}{|c|c|c|c|c|c|c|c|c|c|}
	\hline
	\rowcolor{UGLiOrange}Rang & \hspace{0.5cm}0\hspace{0.5cm} & \hspace{0.5cm}1\hspace{0.5cm} & \hspace{0.5cm}2\hspace{0.5cm} & \hspace{0.5cm}3\hspace{0.5cm} & \hspace{0.5cm}4\hspace{0.5cm} & \hspace{0.5cm}5\hspace{0.5cm} & \hspace{0.5cm}6\hspace{0.5cm} & \hspace{0.5cm}7\hspace{0.5cm} & \hspace{0.5cm}8\hspace{0.5cm}  \\
	\hline
	\cellcolor{UGLiOrange}Suite 1 : & $1$ & $3$ & $5$ & $7$ & $9$ & & & & \\
	\hline
	\cellcolor{UGLiOrange}Suite 2 : & $81$ & $27$ & $9$ & $3$ & $1$ & & & & \\
	\hline
	\cellcolor{UGLiOrange}Suite 3 : & $15$ & $10$ & $5$ & $0$ & $-5$ & & & & \\
	\hline
	\cellcolor{UGLiOrange}Suite 4 : & $8$ & $-4$ & $2$ & $-1$ & $\dfrac{1}{2}$ & & & & \\
	\hline
	\cellcolor{UGLiOrange}Suite 5 : & $1$ & $5$ & $13$ & $29$ & $61$ & & & & \\
	\hline
\end{tabular}

\vspace{.5cm}
\begin{definition}[]
	Une suite $(u_n)$ est dite \textbf{arithmétique} s'il existe un réel $r$ tel que pour tout entier $n$ on a  $u_{n+1}=u_n+r$.\\
	Cette expression est appelée formule de récurrence.\\
	Le nombre $r$ est appelé \textbf{raison} de la suite $(u_n)$.
\end{definition}

\begin{definition}[]
	Une suite $(u_n)$ est dite \textbf{géométrique} s'il existe un réel $q$ tel que pour tout entier $n$ on a  $u_{n+1}=q\times u_n$.\\
	Cette expression est appelée formule de récurrence.\\
	Le nombre $q$ est appelé \textbf{raison} de la suite $(u_n)$.
\end{definition}

\begin{propriete}[s]
\begin{enumerate}[label=\textbullet]
	\item 	Si une suite $(u_n)$ est \textbf{arithmétique} de raison $r$ alors pour tout entier $n$ on a :\\
	$u_n=u_0+n\times r$ (formule explicite).
	\item 	Si une suite $(u_n)$ est \textbf{géométrique} de raison $q$ alors pour tout entier $n$ on a :\\
	$u_n=u_0\times q^n$ (formule explicite).
\end{enumerate}
\end{propriete}

\newpage
\exo{ : Compléter lorsque la suite est soit arithmétique, soit géométrique}
\begin{tabular}{|c|p{2.5cm}|p{2.5cm}|p{2.5cm}|c|c|c|c|c|}
	\hline
	 & Relation de récurrence de la suite & \centering Nature & Forme explicite pour tout entier $n\geqslant0$ & $u_0$ & $u_1$ & $u_{10}$ & $u_{15}$ & $u_{20}$ \\
	 \hline
	 Suite 1 & & & & \hspace{.9cm} & \hspace{.9cm} & \hspace{.9cm} & \hspace{.9cm} & \hspace{.9cm} \\
	 \hline
	 Suite 2 & & & & \hspace{.9cm} & \hspace{.9cm} & \hspace{.9cm} & \hspace{.9cm} & \hspace{.9cm} \\
	 \hline
	 Suite 3 & & & & \hspace{.9cm} & \hspace{.9cm} & \hspace{.9cm} & \hspace{.9cm} & \hspace{.9cm} \\
	 \hline
	 Suite 4 & & & & \hspace{.9cm} & \hspace{.9cm} & \hspace{.9cm} & \hspace{.9cm} & \hspace{.9cm} \\
	 \hline
	 Suite 5 & & & & \hspace{.9cm} & \hspace{.9cm} & \hspace{.9cm} & \hspace{.9cm} & \hspace{.9cm} \\
	 \hline
\end{tabular}

\vspace{1cm}

\exo{ Schématiser les propriétés concernant les suites 1 et 2}
\begin{minipage}{2cm}
	Suite 1 :\\
	\vspace{1.8cm}
	
	Suite 2 :
\end{minipage}
\begin{minipage}{15cm}
	\includegraphics[width=15cm]{schema}
\end{minipage}

\end{document}
