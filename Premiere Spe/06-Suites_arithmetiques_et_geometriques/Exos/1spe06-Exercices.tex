\documentclass[a4paper,11pt,exos]{nsi} % COMPILE WITH DRAFT


%\pagestyle{empty}
\begin{document}

\setlength{\columnseprule}{0.5pt}
\setlength{\columnsep}{1cm}

\classe{\premiere spe}
\titre{Suites arithmétiques et géométriques}
\maketitle

\exo{ Reconnaître une suite arithmétique ou géométrique avec une formule de récurrence}
Les suites suivantes sont-elles arithmétiques ? géométriques ? Si oui, en donner la raison.
		\begin{enumerate}
			\item 	$(u_n)$ définie par : $\left\{
			\begin{array}{llll}
				u_0 & = & 2 &\\ 
				u_{n+1} & = & u_n-7&\text{pour tout}\  n\in\N\\
			\end{array}\right. $
		
			\item 	$(v_n)$ définie par : $\left\{
			\begin{array}{llll}
				v_0 & = & 14 &\\ 
				v_{n+1} & = & \dfrac{v_n}{7}&\text{pour tout}\  n\in\N\\
			\end{array}\right. $
			
			\item 	$(w_n)$ définie par : $\left\{
			\begin{array}{llll}
				w_0 & = & 14 &\\ 
				w_{n+1} & = & w_n+n&\text{pour tout}\  n\in\N\\
			\end{array}\right. $
		
			\item 	$(z_n)$ définie par $\left\{
			\begin{array}{llll}
				z_0 & = & 2 &\\ 
				z_{n+1} & = & z_n\times 2^n&\text{pour tout}\  n\in\N\\
			\end{array}\right. $
		
		\item	$(t_n)$ définie par $\left\{
		\begin{array}{llll}
			t_0 & = & 1 &\\ 
			t_{n+1} & = & 2t_n+7&\text{pour tout}\  n\in\N\\
		\end{array}\right. $
			
		\end{enumerate}



\exo{ Reconnaître une suite arithmétique ou géométrique avec une formule explicite}
Les suites suivantes sont-elles arithmétiques ? géométriques ? Si oui, en donner la raison.
\begin{multicols}{2}
	\begin{enumerate}
		\item 	$(u_n)$ définie sur $\N$ par : $\quad u_n = 6-n$.
		\item 	$(v_n)$ définie sur $\N$ par : $\quad v_n= 6 + 3n$.
		\item 	$(w_n)$ définie sur $\N$ par : $\quad w_n = 6n$.
		\item 	$(z_n)$ définie sur $\N$ par : $\quad z_n=\dfrac{6}{n}$.
		\item 	$(t_n)$ définie sur $\N$ par : $\quad	t_n=3\times 6^n$.
		\item 	$(s_n)$ définie sur $\N$ par : $\quad s_n =\dfrac{6}{3^n}$.
	\end{enumerate}
\end{multicols}




\exo{ Calculer un terme d'une suite arithmétique ou géométrique}
Pour chacune des suites suivantes, calculer $u_{20}$.
\begin{enumerate}
	\item 	$(u_n)$ est une suite arithmétique de premier terme $10$ et de raison $-3$.
	\item	$(u_n)$ est une suite géométrique de premier terme $-1$ et de raison $2$.
	\item	La suite $(u_n)$ est définie par : $\left\{
	\begin{array}{llll}
		u_0 & = & 2048 &\\ 
		u_{n+1} & = & \dfrac{1}{2}u_n&\text{pour tout}\  n\in\N\\
	\end{array}\right. $
	\item	La suite $(u_n)$ est définie par : $\left\{
	\begin{array}{llll}
		u_0 & = & 30 &\\ 
		u_{n+1} & = & u_n-3&\text{pour tout}\  n\in\N\\
	\end{array}\right. $
\end{enumerate}

\definecolor{ao(english)}{rgb}{0.0, 0.5, 0.0}
\definecolor{darkmagenta}{rgb}{0.55, 0.0, 0.55}


\begin{minipage}{10cm}
	\exo{}
	Dans le repère orthonormé ci-contre, on a représenté quelques termes de trois suites arithmétiques \textcolor{ao(english)}{$(u_n)$}, \textcolor{darkmagenta}{$(v_n)$} et \textcolor{red}{$(w_n)$}.
	Pour chacune d'elles, donner :
		\begin{enumerate}
			\item 	Son premier terme et sa raison ;
			\item 	Une formule explicite permettant de la définir ;
			\item	Les termes de rang 3 et de rang 6.
		\end{enumerate}
	
\end{minipage}
\begin{minipage}{7cm}
	\flushright \includegraphics[width=6cm]{repere}
\end{minipage}



\subsection*{Modéliser à l'aide d'une suite arithmétique ou géométrique}

\exo{}
Une famille décide de d'épargner afin de s'offrir un voyage.\\
En 2022, elle a économisé 500 €. Chaque mois à partir du 1$^{\text{er}}$ janvier 2023, elle augmente la somme épargnée de 100 €.\\
Pour chaque entier $n$, on note $s_n$ la somme épargnée après $n$ mois.
\begin{enumerate}
	\item 	Déterminer $s_0, s_1$ et $s_2$
	\item 	Pour tout $n \in \N$, exprimer $s_{n+1}$ en fonction de $s_n$.
	\item	En déduire l'expression de $s_n$ en fonction de l'entier $n$.
	\item	Le voyage que prépare la famille coûte 4200 €. Déterminer à partir de quelle date la famille pourra partir en voyage.
\end{enumerate}




\exo{ En médecine}
Afin de greffer 10 cm$^2$ de peau à une personne brûlée, on lui en prélève 20 mm$^2$. La culture permet d'augmenter de 15 \% la surface de peau chaque jour.\\
On cherche à déterminer au bout de combien de jour la greffe sera possible.
\begin{enumerate}
	\item 	Calculer la surface de peau après un jour et deux jours de culture.
	\item 	Pour tout entier naturel $n$, $v_n$ modélise la surface de peau après $n$ jours de culture. \'Ecrire une relation entre $v_{n+1}$ et $v_n$.
	\item	Quelle est la nature de la suite $(v_n)$ ?
	\item	Donner l'expression de $v_n$ en fonction de $n$.
	\item	Répondre au problème posé.
\end{enumerate}

\newpage

\subsection*{Sens de variation}
\exo{ Sens de variation d'une suite arithmétique }
Déterminer les variations des suites définies ci-dessous :
\begin{multicols}{2}
	\begin{enumerate}
		\item 	$\left\{
		\begin{array}{llll}
			u_0 & = & 2 &\\ 
			u_{n+1} & = & u_n+\pi-3&\text{pour tout}\  n\in\N\\
		\end{array}\right. $
		\item 	$\left\{
		\begin{array}{llll}
			v_0 & = & 12 &\\ 
			v_{n+1} & = & v_n+1-\sqrt{2}&\text{pour tout}\  n\in\N\\
		\end{array}\right. $
	\end{enumerate}
\end{multicols}


\exo{ Sens de variation d'une suite géométrique}
Donner les variations des suites géométriques définies ci-dessous :
\begin{multicols}{2}
	\begin{enumerate}
		\item 	$(u_n)$ de premier terme 2 et de raison 0,3.
		\item 	$(v_n)$ de premier terme 3 et de raison $-$5.
		\item 	$(w_n)$ de premier terme $-$6 et de raison 14.
		\item 	$(z_n)$ de premier terme 3 et de raison $\sqrt{2}$.
		\item	$(t_n)$ de premier terme $-$5 et de raison $\sqrt{\dfrac{10}{\pi^2}}$.
		\item 	$(r_n)$ de premier terme 0 et de raison 12.
	\end{enumerate}
\end{multicols}



\subsection*{ Somme des premiers termes d'une suite arithmétique ou géométrique}
\exo{}
Soit $u$ la suite définie par : pour tout $n\in\N$, $u_n=3+4n$.
\begin{enumerate}
	\item 	Calculer $u_0+u_1+\ldots+u_{40}$.
	\item 	Calculer $\displaystyle\sum_{k=0}^{20} u_k$.
	\item 	Calculer de deux manières $u_{21}+u_{22}+\ldots+u_{40}$.
\end{enumerate}


\exo{}
Un étudiant loue une chambre pendant 2 ans. Le loyer initial est de 200 euros par mois mais tous les mois il augmente de 2\%.
\begin{enumerate}
	\item 	Exprimer les loyers à l'aide d'une suite géométrique.
	\item 	En déduire la somme totale que l'étudiant aura à payer sur deux ans.
	\item 	Quel est le loyer moyen payé par l'étudiant sur deux ans ?
\end{enumerate}
%\newpage

\setlength{\columnseprule}{0pt}

\exo{}
Le film \emph{Avatar} est sorti aux \'Etats-Unis le 18 décembre 2009. La recette lors de la première semaine s'est élevée à 77 millions de dollars. Cette recette a ensuite diminué en moyenne 
de 15\% chaque semaine. Le réalisateur James Cameron a investi 500 millions de dollars pour la réalisation du film.\\
Pour les calculs, l'unité est le million de dollars.
\begin{enumerate}
	\item 	Soit $R_0$ la recette obtenue la première semaine.
	Calculer $R_1$ et $R_2$ (ne pas justifier).
	\item 	Pour tout $n\in\N$, exprimer $R_{n+1}$ en fonction de $R_n$ en justifiant.
	\item 	Exprimer $R_n$ en fonction de $n$ et de $R_0$.
	\item 	Quel est le sens de variation de la suite $(R_n)$ ? Justifier.
	\item 	Quelle est la recette pour la vingtième semaine (arrondir au centième) ?
	\item 	Exprimer en fonction de $n$ le total $T_n$ des recettes engrangées de la première semaine à la (n+1)-ième de la manière la plus simple possible.
	
	\begin{multicols}{2}
		\item 	Quand $n$ devient très grand, de quelle valeur limite $T_n$ se rapproche-t-il ?
		\item 	On considère l'algorithme ci contre :\\
		Que fait cet algorithme ?
		\item  	On l'exécute et l'algorithme affiche 22.\\
		Interpréter ce résultat.\\
        \begin{pyc}
            \begin{minted}{python}
n = 0
r = 77
t = 77
while t < 500 :
   n = n + 1
   r = 0.85 * r
   t = t + r
print(n)                
            \end{minted}
        \end{pyc}
		
	\end{multicols}
	
\end{enumerate}

\subsection*{ Pour approfondir}
\exo{}
On considère la suite $(u_n)$ définie par $u_1=1$ et pour tout $ n\in\N^*$, $u_{n+1}  =  \dfrac{nu_n+4}{n+1}$.

\begin{enumerate}
	\item 	Calculer $u_2$ et $u_3$.
	\item 	Démontrer que la suite $(v_n)$ définie sur $\N^*$ par $v_n=nu_n$ est une suite arithmétique dont on précisera le premier terme et la raison.
	\item 	En déduire l'expression du terme général de $(v_n)$.
	\item 	En déduire l'expression du terme général de $(u_n)$.
\end{enumerate}



\exo{}
$n$ est un entier naturel. À l'aide de suites arithmétiques :
\begin{enumerate}
	\item 	Calculer  $0+1+\dots+(2n-1)+2n$, somme des entiers de 0 à $2n$.
	\item 	Calculer  $0+2+4+\ldots (2n-2)+2n$, somme des entiers pairs de 0 à $2n$.
	\item 	En déduire $1+3+5+\ldots+(2n-3)+(2n-1)$, somme des entiers impairs compris entre 0 et $2n$.
\end{enumerate}

\exo{}
$n$ est un entier naturel. Calculer $2\times 2^2\times 2^3\ldots\times 2^n$.

\end{document}

