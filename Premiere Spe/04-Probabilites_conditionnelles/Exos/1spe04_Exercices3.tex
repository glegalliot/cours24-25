\documentclass[a4paper,11pt,exos]{nsi} % COMPILE WITH DRAFT
\usepackage{pifont}
\usepackage{fontawesome5}
\usepackage{hyperref}



\begin{document}
\classe{\premiere spé}
\titre{Exo 3 - Indépendance}
\maketitle

\exo{}
Dans chaque cas, déterminer si les événements $A$ et $B$ sont indépendants.
\begin{enumerate}
    \item $P(A)=0,2$, $P(B)=0,8$ et $P(A\cap B)=0,2$.
    \item $P(A)=0,3$, $P(B)=0,7$ et $P(A\cap B)=0,21$.
    \item $P(A)=0,5$, $P(B)=0,3$ et $P(A\cup B)=0,65$.
    \item $P(A)=0,48$, $P(B)=0,25$ et $P(A\cup B)=0,73$.
\end{enumerate}



\exo{}
Soient $A$ et $B$ deux événements indépendants tels que $P(\overline{A})=0,6$ et $P(A\cap B)=0,3$.\\
Calculer $P(A)$ puis $P(B)$.



\exo{ \faStar}
$A$ et $B$ sont deux événements incompatibles de probabilité non nulle.\\
Démontrer que $A$ et $B$ ne sont pas indépendants.



\exo{}
On lance un dé non truqué à six faces et on note les événements suivants :
\begin{enumerate}[label=\textbullet]
    \item $A$ : « le résultat est 4 ; 5 ou 6 ».
    \item $B$ : « le résultat est un nombre pair ».
\end{enumerate}
Les événements $A$ et $B$ sont-ils indépendants ?



\exo{}
Dans un magasin de décoration, 20 \% des clients à la caisse achètent de la peinture, les autres achètent du papier peint.\\
Parmi les clients qui achètent de la peinture, la moitié paie à crédit. Parmi les clients qui achètent du papier peint, les trois quarts paient à crédit.
On choisi au hasard un client à la caisse.
\begin{enumerate}
    \item Décrire la situation par un arbre de probabilité ou un tableau.
    \item Les événements « le client achète de la peinture » et « le client paye à crédit » sont-ils indépendants ?
\end{enumerate}


\exo{}
André est un piètre  pêcheur : la probabilité qu'il réussisse à pêcher un poisson est égale à 0,3 chaque jour.
\begin{enumerate}
    \item En supposant que le résultat de sa pêche est indépendant du résultat du jour précédent, déterminer la probabilité qu'il attrappe un poisson quatre jours de suite.
    \item En supposant cette fois que la probabilité d'une pêche fructueuse augmente de 0,5 le jour suivant un échec et de 0,15 le jour suivant une réussite (et vaut 1 si ce nombre devait dépasser 1 avec les instructions précédentes), calculer la probabilité qu'il attrape un poisson chacun des deux premiers jours puis la probabilité qu'il en attappe un chacun des trois premiers jours.
\end{enumerate}



\exo{}
Soit $x\in\fif{0}{1}$.\\
On considère deux événements $A$ et $B$ tels que $P(A)=x$, $P(B)=1-x$ et $P(A\cap B)=\dfrac{1}{4}$.\\
Déterminer les valeurs de $x$ pour lesquelles $A$ et $B$ sont indépendants.



\exo{}
Soit $p\in\oio{0}{1}$. On considère deux événements $A$ et $B$ tels que $$P(A)=p,\quad P(B)=P(\overline{A})\quad \text{et}\quad P(A\cap B)=0,2p+0,15.$$
\begin{enumerate}
    \item Résoudre dans $\R$ l'équation $\quad -x^2+0,8x-0,15=0$.
    \item En déduire les valeurs de $p$ pour lesquelles $A$ et $B$ sont indépendants.
\end{enumerate}




\exo{ \faStar\faStar}
On considère deux événements $A$ et $B$ tels que $P(A\cap B)=0,8$ et $P(A\cup B)=0,9$.
\begin{enumerate}
    \item Résoudre dans $\R$ l'équation $x^2-1,7x+0,8=0$.
    \item Monter que $A$ et $B$ ne peuvent pas être indépendants.
\end{enumerate}


\end{document}