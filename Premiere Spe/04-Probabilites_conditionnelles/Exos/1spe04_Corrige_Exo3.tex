\documentclass[a4paper,11pt,exos]{nsi} % COMPILE WITH DRAFT
\usepackage{pifont}
\usepackage{fontawesome5}
\usepackage{hyperref}



\begin{document}
\classe{\premiere spé}
\titre{Corrigés exo 3 - Indépendance}
\maketitle

\exo{}
\textcolor{UGLiBlue}{Dans chaque cas, déterminer si les événements $A$ et $B$ sont indépendants.
\begin{enumerate}
    \item $P(A)=0,2$, $P(B)=0,8$ et $P(A\cap B)=0,2$.
    \item $P(A)=0,3$, $P(B)=0,7$ et $P(A\cap B)=0,21$.
    \item $P(A)=0,5$, $P(B)=0,3$ et $P(A\cup B)=0,65$.
    \item $P(A)=0,48$, $P(B)=0,25$ et $P(A\cup B)=0,73$.
\end{enumerate}}

\begin{enumerate}
    \item $P(A)\times P(B)=0,2\times 0,8=0,16\neq 0,2=P(A\cap B)$\\
    Donc $A$ et $B$ ne sont pas indépendants.
    \item $P(A)\times P(B)=0,3\times 0,7=0,21=0,21=P(A\cap B)$\\
    Donc $A$ et $B$ sont indépendants.	
    \item $P(A\cap B)=P(A)+P(B)-P(A\cup B)=0,5+0,3-0,65=0,15$\\
    $P(A)\times P(B)=0,5\times 0,3=0,15=0,15=P(A\cap B)$\\
    Donc $A$ et $B$ sont indépendants.
    \item $P(A\cap B)=P(A)+P(B)-P(A\cup B)=0,48+0,25-0,73=0$\\
    $P(A)\times P(B)=0,48\times 0,25=0,12\neq 0=P(A\cap B)$\\
    Donc $A$ et $B$ ne sont pas indépendants.
\end{enumerate}

\exo{}
\textcolor{UGLiBlue}{Soient $A$ et $B$ deux événements indépendants tels que $P(\overline{A})=0,6$ et $P(A\cap B)=0,3$.\\
Calculer $P(A)$ puis $P(B)$.}

\begin{multicols}{2}
    \begin{tabbing}
        $P(A)$  \=$=1-P(\overline{A})$\\
                \>$=1-0,6$\\
                \>$0,4$
    \end{tabbing}
    \vfill\null
    \columnbreak
    \begin{tabbing}
        $P(B)$  \=$=P_A(B)\quad$ car $A$ et $B$ sont indépendants\\
                \>$=\dfrac{P(A\cap B)}{P(A)}$\\
                \>$=\dfrac{0,3}{0,4}$\\
                \>$=0,75$
    \end{tabbing}
\end{multicols}

\exo{ \faStar}
\textcolor{UGLiBlue}{$A$ et $B$ sont deux événements incompatibles de probabilité non nulle.\\
Démontrer que $A$ et $B$ ne sont pas indépendants.}\\

$A$ et $B$ sont incompatibles donc $A\cap B=\varnothing$ et $P(A\cap B)=0$.\\
$P_B(A)=\dfrac{P(A\cap B)}{P(B)}=\dfrac{0}{P(B)}=0\neq P(A)$\\
Donc $A$ et $B$ ne sont pas indépendants.

\exo{}
\textcolor{UGLiBlue}{On lance un dé non truqué à six faces et on note les événements suivants :
\begin{enumerate}[label=\textbullet]
    \item $A$ : « le résultat est 4 ; 5 ou 6 ».
    \item $B$ : « le résultat est un nombre pair ».
\end{enumerate}
Les événements $A$ et $B$ sont-ils indépendants ?}\\

On a $A\cap B$ est l'événement « le résultat est 4 ou 6 ».\\[.5em]
D'où $P(A)=\dfrac{3}{6}=\dfrac{1}{2}$, $\quad P(B)=\dfrac{3}{6}=\dfrac{1}{2}\quad$ et $\quad P(A\cap B)=\dfrac{2}{6}=\dfrac{1}{3}$.\\[.5em]
$P(A)\times P(B)=\dfrac{1}{2}\times \dfrac{1}{2}=\dfrac{1}{4}\neq \dfrac{1}{3}=P(A\cap B)$.\\[.5em]
Donc $A$ et $B$ ne sont pas indépendants.

\exo{}
\textcolor{UGLiBlue}{Dans un magasin de décoration, 20 \% des clients à la caisse achètent de la peinture, les autres achètent du papier peint.\\
Parmi les clients qui achètent de la peinture, la moitié paie à crédit. Parmi les clients qui achètent du papier peint, les trois quarts paient à crédit.
On choisi au hasard un client à la caisse.
\begin{enumerate}
    \item Décrire la situation par un arbre de probabilité ou un tableau.
    \item Les événements « le client achète de la peinture » et « le client paye à crédit » sont-ils indépendants ?
\end{enumerate}}

\begin{enumerate}
    \item On note $P$ l'événement « le client achète de la peinture » et $C$ l'événement « le client paye à crédit ».\\
    On obtient l'arbre de probabilité suivant :
    \def\abun{$P$}
 \def\alun{0,2}
 \def\abdeux{$\barmaj{P}$}
 \def\aldeux{0,8}
 \def\abtrois{$C$}
 \def\altrois{0,5}
 \def\abquatre{$\barmaj{C}$}
 \def\alquatre{0,5}
 
\def\abcinq{$C$}
 \def\alcinq{0,75}
 \def\absix{$\barmaj{C}$}
 \def\alsix{0,25}
 \begin{center}
 \arbreproba
 \end{center}
    \item \begin{multicols}{3}
        \begin{tabbing}
            $P(P\cap C)$ \=$=P(P)\times P_P(C)$\\
                        \>$=0,2\times 0,5$\\
                        \>$=0,1$
        \end{tabbing}
        \vfill\null
        \columnbreak
        \begin{tabbing}
            $P(P)$ \=$=0,2$\\[.5em]
            $P(C)$ \=$=P(P\cap C)+P(\barmaj{P}\cap C)$\\
                    \>$=0,1+0,8\times 0,75$\\
                    \>$=0,7$
        \end{tabbing}
        
        \begin{tabbing}
            $P(P)\times P(C)$ \=$=0,2\times 0,7$\\
                            \>$=0,14$\\
                            \>$\neq P(P\cap C)$
        \end{tabbing}
    \end{multicols}
    Donc Les événements « le client achète de la peinture » et « le client paye à crédit » ne sont pas indépendants.
\end{enumerate}

\exo{}
\textcolor{UGLiBlue}{André est un piètre  pêcheur : la probabilité qu'il réussisse à pêcher un poisson est égale à 0,3 chaque jour.
\begin{enumerate}
    \item En supposant que le résultat de sa pêche est indépendant du résultat du jour précédent, déterminer la probabilité qu'il attrappe un poisson quatre jours de suite.
    \item En supposant cette fois que la probabilité d'une pêche fructueuse augmente de 0,5 le jour suivant un échec et de 0,15 le jour suivant une réussite (et vaut 1 si ce nombre devait dépasser 1 avec les instructions précédentes), calculer la probabilité qu'il attrape un poisson chacun des deux premiers jours puis la probabilité qu'il en attappe un chacun des trois premiers jours.
\end{enumerate}}

\begin{enumerate}
    \item La probabilité qu'André attrape un poisson quatre jours de suite est $0,3^4=0,0081$.
    \item On note $P_i$ l'évenement « André attrape un poisson le jour $i$ ».\\
    On obtient l'arbre suivant :
    \def\abun{$P_1$}
\def\alun{0,3}
\def\abdeux{$\barmaj{P_1}$}
\def\aldeux{0,7}
\def\abtrois{$P_2$}
\def\altrois{0,45}
\def\abquatre{$\barmaj{P_2}$}
\def\alquatre{0,55}
\def\abcinq{$P_2$}
\def\alcinq{0,8}
\def\absix{$\barmaj{P_2}$}
\def\alsix{0,2}
\begin{center}
\arbreproba
\end{center}

    \begin{enumerate}[label=\textbullet]
        \item 
        La probabilité qu'André attrape un poisson chacun des deux premiers jours est $0,3\times 0,45=0,135$.
        \item La probabilité qu'André attrape un poisson chacun des trois premiers jours est $0,3\times 0,45\times 0,6=0,081$.
    \end{enumerate}
\end{enumerate}

\exo{}
\textcolor{UGLiBlue}{Soit $x\in\fif{0}{1}$.\\
On considère deux événements $A$ et $B$ tels que $P(A)=x$, $P(B)=1-x$ et $P(A\cap B)=\dfrac{1}{4}$.\\
Déterminer les valeurs de $x$ pour lesquelles $A$ et $B$ sont indépendants.}

\begin{tabbing}
    $A$ et $B$ sont indépendants \= $\iff P(A)\times P(B)=P(A\cap B)$\\[.5em]
                                \> $\iff x\times (1-x)=\dfrac{1}{4}$\\[.5em]
                                \> $\iff x-x^2=\dfrac{1}{4}$\\[.5em]
                                \> $\iff -x^2+x-\dfrac{1}{4}=0$\\[.5em]
                                \> $\iff x^2-x+\dfrac{1}{4}=0$\\[.5em]
                                \> $\iff x^2-x+\dfrac{1}{4}=0$\\[.5em]
                                \> $\iff (x-\dfrac{1}{2})^2=0$\\[.5em]
                                \> $\iff x=\dfrac{1}{2}$
\end{tabbing}

\exo{}
\textcolor{UGLiBlue}{Soit $p\in\oio{0}{1}$. On considère deux événements $A$ et $B$ tels que $$P(A)=p,\quad P(B)=P(\overline{A})\quad \text{et}\quad P(A\cap B)=0,2p+0,15.$$
\begin{enumerate}
    \item Résoudre dans $\R$ l'équation $\quad -x^2+0,8x-0,15=0$.
    \item En déduire les valeurs de $p$ pour lesquelles $A$ et $B$ sont indépendants.
\end{enumerate}}

\begin{enumerate}
    \item On calcule le discriminant du polynome $-x^2+0,8x-0,15$ :
    \begin{tabbing}
        $\Delta$ \=$=0,8^2-4\times (-1)\times (-0,15)$\\
                \>$=0,64-0,6$\\
                \>$=0,04$
    \end{tabbing}
    Calculons les racines du polynome :
    \begin{tabbing}
        $x_1$ \=$=\dfrac{-0,8-\sqrt{\Delta}}{-2} \qquad x_2$ \=$=\dfrac{-0,8+\sqrt{\Delta}}{-2}$\\[.5em]
                \>$=\dfrac{-0,8-0,2}{-2}$ \>$=\dfrac{-0,8+0,2}{-2}$\\[.5em]
                \>$=0,5$   \>$=0,3$            
    \end{tabbing}
    D'où $\mathcal{S}=\{0,3;0,\}$.
    \item \begin{tabbing}
        $A$ et $B$ sont indépendants \= $\iff P(A)\times P(B)=P(A\cap B)$\\
                                    \> $\iff p\times P(\overline{A})=0,2p+0,15$\\
                                    \> $\iff p\times (1-p)=0,2p+0,15$\\
                                    \> $\iff -p^2+p=0,2p+0,15$\\
                                    \> $\iff -p^2+0,8p-0,15=0$\\
                                    \> $\iff p= 0,3 \ $ ou $\ p=0,5$
    \end{tabbing}
    Donc $A$ et $B$ sont indépendants si, et seulement si $p=0,3$ ou $p=0,5$.
\end{enumerate}


\exo{ \faStar\faStar}
\textcolor{UGLiBlue}{On considère deux événements $A$ et $B$ tels que $P(A\cap B)=0,8$ et $P(A\cup B)=0,9$.
\begin{enumerate}
    \item Résoudre dans $\R$ l'équation $x^2-1,7x+0,8=0$.
    \item Monter que $A$ et $B$ ne peuvent pas être indépendants.
\end{enumerate}}

\begin{enumerate}
    \item On calcule le discriminant du polynome $x^2-1,7x+0,8$ :
    \begin{tabbing}
        $\Delta$ \=$=(-1,7)^2-4\times 1\times 0,8$\\
                \>$=2,89-3,2$\\
                \>$=-0,31$
    \end{tabbing}
    $\Delta<0$ donc le polynome n'a pas de racine réelle et l'équation $x^2-1,7x+0,8=0$ n'a pas de solution dans $\R$.
    \item \begin{enumerate}[label =\textbullet]
        \item Montrons que $A$ et $B$ sont deux événements tels que $P(A)\neq 0$ et $P(B)\neq 0$ :\\[.5em]
        Supposons par l'absurde que $P(A)=0$ ou $P(B)=0$.\\
        Alors $A$ ou $B$ et l'événement impossible et $P(A\cap B)=0$.\\
        Il y a une contradiction avec l'énoncé, donc $P(A)\neq 0$ et $P(B)\neq 0$.
        \item Supposons par l'absurde que $A$ et $B$ sont indépendants.\\[.5em]
        On a : $\quad P(A\cup B)= P(A) + P(B) - P(A\cap B)$\\
        Donc $\quad 0,9 = P(A) + P(B) - 0,8$\\
        Et $\quad P(A) + P(B) = 1,7$\\[.5em]
        Comme on a supposé que $A$ et $B$ sont indépendants,  $\quad P(A\cap B) = P(A)\times P(B)=0,8$\\
        Donc $\quad P(A)=\dfrac{0,8}{P(B)}$\\
        Et $\quad P(A)+P(B)=1,7\quad$ donne $\quad \dfrac{0,8}{P(B)}+P(B)=1,7$.\\
        Posons $p=P(B)$.\\
        $p$ est solution de l'équation $\dfrac{0,8}{p}+p=1,7$\\ 
        Soit $p$ un réel strictement positif.
        \begin{tabbing}
            $\dfrac{0,8}{p}+p=1,7$  \=$\iff 0,8+p^2=1,7p$\\
                                    \>$\iff p^2-1,7p+0,8=0$
        \end{tabbing}
        D'après la question précédente, cette équation n'a pas de solution dans $\R$.\\
        Il y a une contradition. Donc $A$ et $B$ ne peuvent pas être indépendants.
    \end{enumerate}
\end{enumerate}
\end{document}