\documentclass[a4paper,11pt,landscape,exos]{nsi} % COMPILE WITH DRAFT
\usepackage{hyperref}

\pagestyle{empty}
\setlength{\columnseprule}{0.5pt}
\setlength{\columnsep}{1cm}

\begin{document}




\begin{multicols}{2}

\textcolor{UGLiBlue}{
        Vendredi 09/05/2025\\
        NOM, Prénom :}\\
\classe{\terminale Comp}
    \titre{\includegraphics[width=3cm]{CAN.png} Interrogation 4}
    \maketitle


    Pour les questions \textbf{1.}, \textbf{2.} et \textbf{3.}, on donne les courbes de deux fonctions $f$ et $g$ :\\
    
    \def\xmin{-1}\def\xmax{5}\def\ymin{-1}\def\ymax{5}
	\def\F{-(\x-2)^2+3}
    \def\G{(\x-1)*(\x-4)*(-\x+3)+2}
	\begin{tikzpicture}
	\reperenb{\xmin}{\ymin}{\xmax}{\ymax}{$x$}{$y$}
	\clip	(\xmin,\ymin) rectangle (\xmax,\ymax);
	\draw[UGLiRed,ultra thick,domain=\xmin:\xmax,smooth,variable=\x]  plot ({\x},{\F});
    \draw[UGLiDarkBlue,ultra thick, dashed,domain=\xmin:\xmax,smooth,variable=\x]  plot ({\x},{\G});
    \draw[UGLiRed] (.2,-.5) node[right] {$\courbe{f}$};
    \draw[UGLiDarkBlue] (1,4.5) node[] {$\courbe{g}$};
	\end{tikzpicture}
    
    

\begin{enumerate}[itemsep=.9em]
    \item Image de $3$ par $f$
    \item Résoudre $f(x)\geqslant 2$.
    \item Solutions de $f(x)=g(x)$
    \item $f(x)=x^2+x+1$\\$f(-4)=\ldots$
	\item $6$ stylos identiques coûtent $9$ €. \\
        Quel est le prix de $15$ stylos ? \\$\ldots$ €
	\item Équation réduite de la droite $(d)$.\\
	\def\xmin{-4}\def\xmax{4}\def\ymin{-4}\def\ymax{2}
	\def\F{\x/3-2}
	\begin{tikzpicture}
	\reperenb{\xmin}{\ymin}{\xmax}{\ymax}{$x$}{$y$}
	\clip	(\xmin,\ymin) rectangle (\xmax,\ymax);
	\draw[UGLiRed,ultra thick,domain=\xmin:\xmax,smooth,variable=\x]  plot ({\x},{\F});
    \draw[UGLiRed] (-3,-3) node[above] {$(d)$};
	\end{tikzpicture}
    \item Multiplier une quantité par $0{,}35$ revient à la diminuer de : $\ldots\,\%$
	\item $(u_n)$ est une suite géométrique telle que $u_1=-1$ et $u_2=3$\\La raison de cette suite est :  $\ldots$
	
	\item Solution de l'équation $5x-2=13$

	
	\item $f(x)=2x^3+x^2-2\ ; \qquad$
    $f'(x)=$ $\ldots$
\end{enumerate}
\end{multicols}
\end{document}