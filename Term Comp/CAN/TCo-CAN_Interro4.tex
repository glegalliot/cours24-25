\documentclass[a4paper,11pt,landscape,exos]{nsi} % COMPILE WITH DRAFT
\usepackage{hyperref}

\pagestyle{empty}
\setlength{\columnseprule}{0.5pt}
\setlength{\columnsep}{1cm}

\begin{document}




\begin{multicols}{2}

\textcolor{UGLiBlue}{
        Vendredi 09/05/2025\\
        NOM, Prénom :}\\
\classe{\terminale Comp}
    \titre{\includegraphics[width=3cm]{CAN.png} Interrogation 4}
    \maketitle


    Pour les questions \textbf{1.}, \textbf{2.} et \textbf{3.}, on donne les courbes de deux fonctions $f$ et $g$ :\\
    
    \def\xmin{-1}\def\xmax{5}\def\ymin{-1}\def\ymax{5}
	\def\F{-(\x-2)^2+3}
    \def\G{(\x-1)*(\x-4)*(-\x+3)+2}
	\begin{tikzpicture}
	\reperenb{\xmin}{\ymin}{\xmax}{\ymax}{$x$}{$y$}
	\clip	(\xmin,\ymin) rectangle (\xmax,\ymax);
	\draw[UGLiRed,ultra thick,domain=\xmin:\xmax,smooth,variable=\x]  plot ({\x},{\F});
    \draw[UGLiDarkBlue,ultra thick, dashed,domain=\xmin:\xmax,smooth,variable=\x]  plot ({\x},{\G});
    \draw[UGLiRed] (.2,-.5) node[right] {$\courbe{f}$};
    \draw[UGLiDarkBlue] (1,4.5) node[] {$\courbe{g}$};
	\end{tikzpicture}
    
    

\begin{enumerate}[itemsep=.9em]
    \item Image de $3$ par $f$
    \item Résoudre $f(x)\geqslant 2$.
    \item Solutions de $f(x)=g(x)$
    \item $f(x)=x^2+x+1$\\$f(-4)=\ldots$
	\item $6$ stylos identiques coûtent $9$ €. \\
        Quel est le prix de $15$ stylos ? \\$\ldots$ €
	\item Équation réduite de la droite $(d)$.\\
	\def\xmin{-4}\def\xmax{4}\def\ymin{-4}\def\ymax{2}
	\def\F{\x/3-2}
	\begin{tikzpicture}
	\reperenb{\xmin}{\ymin}{\xmax}{\ymax}{$x$}{$y$}
	\clip	(\xmin,\ymin) rectangle (\xmax,\ymax);
	\draw[UGLiRed,ultra thick,domain=\xmin:\xmax,smooth,variable=\x]  plot ({\x},{\F});
    \draw[UGLiRed] (-3,-3) node[above] {$(d)$};
	\end{tikzpicture}
    \item Multiplier une quantité par $0{,}35$ revient à la diminuer de : $\ldots\,\%$
	\item $(u_n)$ est une suite géométrique telle que $u_1=-1$ et $u_2=3$\\La raison de cette suite est :  $\ldots$
	
	\item Solution de l'équation $5x-2=13$

	
	\item $f(x)=2x^3+x^2-2\ ; \qquad$
    $f'(x)=$ $\ldots$

\end{enumerate}
\vfill
\end{multicols}



\newpage

\begin{multicols}{2}
	\classe{\terminale Comp}
    \titre{\includegraphics[width=3cm]{CAN.png} Corrigé interro 3}
    \maketitle
\begin{enumerate}[itemsep=.75em]
    \item L'image de $3$ se lit sur l'axe des ordonnées. \\
              On lit $f(3)={\color[HTML]{f15929}\boldsymbol{2}}$. 
\item Les solutions de l'inéquation sont les abscisses des points de $\mathcal{C}_f$ qui se trouvent au-dessus de la droite horizontale d'équation $y=2$.\\
    $S={\color[HTML]{f15929}\boldsymbol{\fif{1}{3}}}$ 
\item Les solutions sont les abscisses des points d'intersection entre les deux courbes :
   $S=\{{\color[HTML]{f15929}\boldsymbol{1\,;\,3}}\}$. 
\item $f(-4)=\left(-4 \right)^2 -4 +1 = 13$\\
    On a donc $f(-4)={\color[HTML]{f15929}\boldsymbol{13}}$.
\item $6$ stylos coûtent $9$ €.\\
          $3$ stylo coûtent  $4{,}50$ €.\\
          Ainsi,   $15$ stylos coûtent $5\times 4{,}50 ={\color[HTML]{f15929}\boldsymbol{22{,}50}}$ €.
\item Le coefficient directeur $m$ de la droite $(AB)$ est donné par :

\medskip

            $m=\dfrac{\Delta_y}{\Delta_x}=\dfrac{{\color{blue}\boldsymbol{1}}}{{\color{red}\boldsymbol{3}}}{\color[HTML]{f15929}\boldsymbol{}}$.

\medskip
L'ordonnée à l'origine est $p=f(0)=-2$.\\
L'équation de la droite $(AB)$ est donc :
\begin{center}
    $\color[HTML]{f15929}\boldsymbol{y=\dfrac{1}{3}x-2}$.
\end{center}
\item On a $0{,}35=1-0{,}65$.\\
    Donc, multiplier une quantité par $0{,}35$ revient à la diminuer de $\color[HTML]{f15929}\boldsymbol{65\%}$.
\item La raison de la suite est donnée par :
\begin{center}
    $q=\dfrac{u_2}{u_1}=\dfrac{3}{-1}=\color[HTML]{f15929}\boldsymbol{-3}$.
\end{center}
\item \begin{tabbing}
    $5x-2=13\quad$ \= $\iff \quad 5x=13+2$ \\
    \> $\iff \quad 5x=15$\\
    \> $\iff \quad x=\dfrac{15}{5}$\\
    \> $\iff \quad x=3$.
\end{tabbing}
La solution de l'équation est donc $\color[HTML]{f15929}\boldsymbol{3}$.
\item \begin{tabbing}
    $f'(x)$ \= $=2\times 3x^2+2x+0$\\
    \> $=\color[HTML]{f15929}\boldsymbol{6x^2+2x}$
\end{tabbing}
\end{enumerate}
\vfill
\end{multicols}


\end{document}