\documentclass[a4paper,11pt,landscape,exos]{nsi} % COMPILE WITH DRAFT
\usepackage{hyperref}

\pagestyle{empty}
\setlength{\columnseprule}{0.5pt}
\setlength{\columnsep}{1cm}
\begin{document}

\begin{multicols}{2}
\classe{\terminale Comp}
\titre{\includegraphics[width=3cm]{CAN.png} Entrainement 6}
\maketitle

\begin{enumerate}[itemsep=0.3cm]
    \item $0{,}9 \times 7$ 
	\item  $13-8\times5$
	\item Forme développée et réduite de $(x-4)(x+3)$
	\item $25\,\%$ de $160$
	\item Médiane de la série :\\$9$\,;\,$5$\,;\,$19$\,;\,$10$\,;\,$21$ 
	\item Écrire sous forme d'une fraction irréductible $\dfrac{-6}{5}\times \dfrac{5}{7}$.
	\item Signe de  $6^{-4}$ 
    
    	$\square\;$ Négatif\qquad $\square\;$ Positif\qquad 
	\item $\dfrac{2^{4}}{2^{2}}=2^{\ldots}$
	\item  Factoriser  $x^2-81$.
	\item $1-\dfrac{5}{3}$ 
	\item La moyenne de $5$, $9$, $15$ et d'un nombre inconnu $n$ est égale à $11$.\\$n=\ldots$
	\item Pablo a couru $4$ km en $15$ minutes, sa vitesse moyenne est de   $\ldots$ km/h
	\item Soit $f$ : $x\longmapsto \dfrac{1}{x^6}$\\$f'(x)=\ldots$
	\item Deux diminutions successives de  $40\,\%$ correspondent à une diminution globale de  $\ldots \,\%$.
	%\item $A$ et $B$ sont deux événements tels que :\\\[\Proba[Arbre,Angle=40,Branche=3,Rayon=0.75,Incline=false]{A/$0.2$,$\overline{A}$/,B
%/$0.6$,$\overline{B}$/,B/$0.5$,$\overline{B}$/}\]$P(B)=\ldots$ 
    \item On donne l’arbre de probabilités ci-dessous :\\         \begin{tikzpicture}[baseline,scale = 0.5]

        \tikzset{
          point/.style={
            thick,
            draw,
            cross out,
            inner sep=0pt,
            minimum width=5pt,
            minimum height=5pt,
          },
        }
        \clip (-0.1,0) rectangle (14,7);
            \draw[color ={black}] (5,5.5)--(10,6.25);
        \draw (10.1,6.25) node[anchor = center] {\colorbox {white}{\tiny  \color{black}{$B$}}};
        \draw (8.5,6.03) node[anchor = center] {\colorbox {white}{\tiny  \color{black}{$0{,}9$}}};
        \draw[color ={black}] (5,5.5)--(10,4.75);
        \draw (10.1,4.75) node[anchor = center] {\colorbox {white}{\tiny  \color{black}{$\overline{B}$}}};
        \draw (8.5,4.98) node[anchor = center] {\colorbox {white}{\tiny  \color{black}{$0{,}1$}}};
        \draw[color ={black}] (0,4)--(5,5.5);
        \draw (5.1,5.5) node[anchor = center] {\colorbox {white}{\tiny  \color{black}{$A$}}};
        \draw (3.5,5.05) node[anchor = center] {\colorbox {white}{\tiny  \color{black}{$0{,}4$}}};
        \draw[color ={black}] (5,2.5)--(10,3.25);
        \draw (10.1,3.25) node[anchor = center] {\colorbox {white}{\tiny  \color{black}{$B$}}};
        \draw (8.5,3.03) node[anchor = center] {\colorbox {white}{\tiny  \color{black}{$0{,}3$}}};
        \draw[color ={black}] (5,2.5)--(10,1.75);
        \draw (10.1,1.75) node[anchor = center] {\colorbox {white}{\tiny  \color{black}{$\overline{B}$}}};
        \draw (8.5,1.98) node[anchor = center] {\colorbox {white}{\tiny  \color{black}{$0{,}7$}}};
        \draw[color ={black}] (0,4)--(5,2.5);
        \draw (5.1,2.5) node[anchor = center] {\colorbox {white}{\tiny  \color{black}{$\overline{A}$}}};
        \draw (3.5,2.95) node[anchor = center] {\colorbox {white}{\tiny  \color{black}{$0{,}6$}}};
        \draw (0.1,4) node[anchor = center] {\colorbox {white}{\tiny  \color{black}{$\phantom{ }$}}};
    
    \end{tikzpicture}\\
    Calculer $P(\overline{A}\cap B)$.
\end{enumerate}

\vfill\null



Score : \ldots\ldots / 15
\end{multicols}

\newpage

\begin{multicols}{2}
    \classe{\terminale Comp}
\titre{\includegraphics[width=3cm]{CAN.png} Corrigé 6}
\maketitle

\begin{enumerate}[]
\item $0{,}9 \times 7={\color[HTML]{f15929}\boldsymbol{6{,}3}}$
\item $13-8\times5={\color[HTML]{f15929}\boldsymbol{-27}}$
\item $\begin{aligned}
      (x-4)(x+3)&=x^2+3x-4x-12\\
      &={\color[HTML]{f15929}\boldsymbol{x^2-x-12}}
      \end{aligned}$\\Le terme en $x^2$ vient de $x\times x=x^2$.\\Le terme en $x$ vient de la somme de $x \times 3$ et de $-4 \times x$.\\Le terme constant vient de $-4\times 3= -12$.
\item $25\,\%$ de $160 = {\color[HTML]{f15929}\boldsymbol{40}}$\\ Prendre $25\,\%$  de $160$ revient à prendre le quart de $160$.\\
      Ainsi, $25\,\%$ de $160$ est égal à $160\div 4 =40$.
     
\item On ordonne la série :  $5$\,;\,$9$\,;\,$10$\,;\,$19$\,;\,$21$.\\
      La série comporte $5$ valeurs donc la médiane est la troisième valeur : ${\color[HTML]{f15929}\boldsymbol{10}}$.
\item $\dfrac{-6}{5}\times \dfrac{5}{7}=-\dfrac{6{\color[HTML]{2563a5}\boldsymbol{\times5}} }{7{\color[HTML]{2563a5}\boldsymbol{\times5}}}={\color[HTML]{f15929}\boldsymbol{-\dfrac{6}{7}}}$
\item $6^{-4}=\dfrac{1}{6^{4}}$\\
     Comme  $6^{4}$ est  positif (puissance paire d'un nombre négatif), on en déduit que  $\dfrac{1}{6^{4}}$ est positif.\\
    Ainsi, $6^{-4}$ est {\bfseries \color[HTML]{f15929}positif}.
\item On utilise la formule $\dfrac{a^n}{a^p}=a^{n - p}$
        avec $a=2$,  $n=4$ et $p=2$.\\
        $\dfrac{2^{4}}{2^{2}}=2^{4-2}=2^{{\color[HTML]{f15929}\boldsymbol{2}}}$
\item On utilise l'égalité remarquable ${\color{red} a}^2-{\color{blue} b}^2=({\color{red} a}-{\color{blue} b})({\color{red} a}+{\color{blue} b})$ avec $a={\color{red} x}$  et $b={\color{blue} 9}$.\\$\begin{aligned}
 x^2-81&=\underbrace{{\color{red} x}^2-{\color{blue} 9}^2}_{a^2-b^2}\\
 &=\underbrace{({\color{red} x}-{\color{blue} 9})({\color{red} x}+{\color{blue} 9})}_{(a-b)(a+b)}
 \end{aligned}$ \\
    Une expression factorisée de $x^2-81$ est ${\color[HTML]{f15929}\boldsymbol{(x-9)(x+9)}}$.
\item On a : \\$\begin{aligned}
      1+\dfrac{5}{3} &= \dfrac{1 \times 3}{3} - \dfrac{5}{3} \\
      &= \dfrac{3}{3} - \dfrac{5}{3}\\
      &  ={\color[HTML]{f15929}\boldsymbol{\dfrac{-2}{3}}}
      \end{aligned}$
\item Puisque la moyenne de ces quatre nombres est $11$, la somme de ces quatre nombres est $4\times 11=44$.\\
             La valeur de $n$ est donnée par :  $44-5-9-15={\color[HTML]{f15929}\boldsymbol{15}}$.
\item $15\times 4= 60$ min $=1$ h\\
    Pablo court $4$ fois plus de km en $1$ heure.\\
   $4\times 4=16$\\
   Pablo court à ${\color[HTML]{f15929}\boldsymbol{16}}$ km/h.
\item D'après le cours, si $f=\dfrac{1}{u}$ alors $f'=\dfrac{-u'}{u^2}$.\\
    $f'(x)=\dfrac{-6x^{5}}{x^{12}}={\color[HTML]{f15929}\boldsymbol{-\dfrac{6}{x^{7}}}}$.
\item  Le coefficient multiplicateur  associé à une baisse de $40\,\%$ est $0{,}6$.\\
    Le coefficient multiplicateur global associé à ces deux diminutions est $0{,}6\times 0{,}6= 0{,}36$.\\
    On en déduit que le taux d'évolution globale est $0{,}36-1=-0{,}64$.\\
    La diminution globale est donc de ${\color[HTML]{f15929}\boldsymbol{64}} \,\%$.
\item 
              $\begin{aligned}
                P(\overline{A}\cap B)&=P(\overline{A}) \times P_{\overline{A}}(B)\\
              &=0{,}6\times 0{,}3\\
              &={\color[HTML]{f15929}\boldsymbol{0{,}18}}
              \end{aligned}$              
\end{enumerate}
\end{multicols}
\end{document}

