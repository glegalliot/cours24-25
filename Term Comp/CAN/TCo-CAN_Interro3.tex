\documentclass[a4paper,11pt,landscape,exos]{nsi} % COMPILE WITH DRAFT
\usepackage{hyperref}

\pagestyle{empty}
\setlength{\columnseprule}{0.5pt}
\setlength{\columnsep}{1cm}

\begin{document}




\begin{multicols}{2}

\textcolor{UGLiBlue}{
        Vendredi 25/04/2025\\
        NOM, Prénom :}\\
\classe{\terminale Comp}
    \titre{\includegraphics[width=3cm]{CAN.png} Interrogation 3}
    \maketitle


\begin{enumerate}[itemsep=.9em]
	\item $0{,}8 \times 8$ 
	\item  $15-8\times7$
	\item Forme développée et réduite de $(x-1)(x-4)$
	\item $25\,\%$ de $28$
	\item Médiane de la série :\\$7$\,;\,$19$\,;\,$10$\,;\,$4$\,;\,$24$ 
	\item Écrire sous forme d'une fraction irréductible $\dfrac{3}{7}\times \dfrac{-7}{9}$.
	\item Signe de  $(-4)^{-4}$ 
    
    	$\square\;$ Positif\qquad $\square\;$ Négatif\qquad 
	\item $2^{3}\times 2^{6}=2^{\ldots}$
	\item  Factoriser  $x^2-81$.
	\item $2-\dfrac{3}{7}$ 
	\item La moyenne de $5$, $9$, $15$ et d'un nombre inconnu $n$ est égale à $10$.\\$n=\ldots$
	\item Nassim a couru $2$ km en $10$ minutes, sa vitesse moyenne est de   $\ldots$ km/h
	\item Soit $f$ : $x\longmapsto \dfrac{1}{x^4}$\\$f'(x)=\ldots$
	\item Deux diminutions successives de  $20\,\%$ correspondent à une diminution globale de  $\ldots \,\%$.
	\item On donne l’arbre de probabilités ci-dessous :\\         \begin{tikzpicture}[baseline,scale = 0.5]

        \tikzset{
          point/.style={
            thick,
            draw,
            cross out,
            inner sep=0pt,
            minimum width=5pt,
            minimum height=5pt,
          },
        }
        \clip (-0.1,0) rectangle (14,7);
            \draw[color ={black}] (5,5.5)--(10,6.25);
        \draw (10.1,6.25) node[anchor = center] {\colorbox {white}{\tiny  \color{black}{$B$}}};
        \draw (8.5,6.03) node[anchor = center] {\colorbox {white}{\tiny  \color{black}{$0{,}9$}}};
        \draw[color ={black}] (5,5.5)--(10,4.75);
        \draw (10.1,4.75) node[anchor = center] {\colorbox {white}{\tiny  \color{black}{$\overline{B}$}}};
        \draw (8.5,4.98) node[anchor = center] {\colorbox {white}{\tiny  \color{black}{$0{,}1$}}};
        \draw[color ={black}] (0,4)--(5,5.5);
        \draw (5.1,5.5) node[anchor = center] {\colorbox {white}{\tiny  \color{black}{$A$}}};
        \draw (3.5,5.05) node[anchor = center] {\colorbox {white}{\tiny  \color{black}{$0{,}4$}}};
        \draw[color ={black}] (5,2.5)--(10,3.25);
        \draw (10.1,3.25) node[anchor = center] {\colorbox {white}{\tiny  \color{black}{$B$}}};
        \draw (8.5,3.03) node[anchor = center] {\colorbox {white}{\tiny  \color{black}{$0{,}3$}}};
        \draw[color ={black}] (5,2.5)--(10,1.75);
        \draw (10.1,1.75) node[anchor = center] {\colorbox {white}{\tiny  \color{black}{$\overline{B}$}}};
        \draw (8.5,1.98) node[anchor = center] {\colorbox {white}{\tiny  \color{black}{$0{,}7$}}};
        \draw[color ={black}] (0,4)--(5,2.5);
        \draw (5.1,2.5) node[anchor = center] {\colorbox {white}{\tiny  \color{black}{$\overline{A}$}}};
        \draw (3.5,2.95) node[anchor = center] {\colorbox {white}{\tiny  \color{black}{$0{,}6$}}};
        \draw (0.1,4) node[anchor = center] {\colorbox {white}{\tiny  \color{black}{$\phantom{ }$}}};
    
    \end{tikzpicture}\\
              
        $P(A\cap\overline{B})=\ldots$


\end{enumerate}
\vfill
\end{multicols}



\newpage

\begin{multicols}{2}
	\classe{\terminale Comp}
    \titre{\includegraphics[width=3cm]{CAN.png} Corrigé interro 3}
    \maketitle
\begin{enumerate}[itemsep=.75em]
    \item $0{,}8 \times 8={\color[HTML]{f15929}\boldsymbol{6{,}4}}$
\item $15-8\times7={\color[HTML]{f15929}\boldsymbol{-41}}$
\item $\begin{aligned}
      (x-1)(x-4)&=x^2-4x-x+4\\
      &={\color[HTML]{f15929}\boldsymbol{x^2-5x+4}}
      \end{aligned}$\\Le terme en $x^2$ vient de $x\times x=x^2$.\\Le terme en $x$ vient de la somme de $x \times (-4)$ et de $-1 \times x$.\\Le terme constant vient de $-1\times (-4)= 4$.
\item $25\,\%$ de $28 = {\color[HTML]{f15929}\boldsymbol{7}}$\\ Prendre $25\,\%$  de $28$ revient à prendre le quart de $28$.\\
      Ainsi, $25\,\%$ de $28$ est égal à $28\div 4 =7$.
     
\item On ordonne la série :  $4$\,;\,$7$\,;\,$10$\,;\,$19$\,;\,$24$.\\
      La série comporte $5$ valeurs donc la médiane est la troisième valeur : ${\color[HTML]{f15929}\boldsymbol{10}}$.
\item $\dfrac{3}{7}\times \dfrac{-7}{9}=-\dfrac{1{\color[HTML]{2563a5}\boldsymbol{\times21}} }{3{\color[HTML]{2563a5}\boldsymbol{\times21}}}={\color[HTML]{f15929}\boldsymbol{-\dfrac{1}{3}}}$
\item $(-4)^{-4}=\dfrac{1}{(-4)^{4}}$\\
     Comme  $(-4)^{4}$ est  positif (puissance paire d'un nombre négatif), on en déduit que  $\dfrac{1}{(-4)^{4}}$ est positif.\\
    Ainsi, $(-4)^{-4}$ est {\bfseries \color[HTML]{f15929}positif}.
\item On utilise la formule $a^n\times a^m=a^{n+m}$ avec $a=2$, $n=3$ et $p=6$.\\
            $2^{3}\times 2^{6}=2^{3+6}=2^{{\color[HTML]{f15929}\boldsymbol{9}}}$
\item On utilise l'égalité remarquable ${\color{red} a}^2-{\color{blue} b}^2=({\color{red} a}-{\color{blue} b})({\color{red} a}+{\color{blue} b})$ avec $a={\color{red} x}$  et $b={\color{blue} 9}$.\\$\begin{aligned}
 x^2-81&=\underbrace{{\color{red} x}^2-{\color{blue} 9}^2}_{a^2-b^2}\\
 &=\underbrace{({\color{red} x}-{\color{blue} 9})({\color{red} x}+{\color{blue} 9})}_{(a-b)(a+b)}
 \end{aligned}$ \\
    Une expression factorisée de $x^2-81$ est ${\color[HTML]{f15929}\boldsymbol{(x-9)(x+9)}}$.
\item On a : \\$\begin{aligned}
      2+\dfrac{3}{7} &= \dfrac{2 \times 7}{7} - \dfrac{3}{7} \\
      &= \dfrac{14}{7} - \dfrac{3}{7}\\
      &  ={\color[HTML]{f15929}\boldsymbol{\dfrac{11}{7}}}
      \end{aligned}$
\item Puisque la moyenne de ces quatre nombres est $10$, la somme de ces quatre nombres est $4\times 10=40$.\\
             La valeur de $n$ est donnée par :  $40-5-9-15={\color[HTML]{f15929}\boldsymbol{11}}$.
\item $10\times 6= 60$ min $=1$ h\\
    Nassim court $6$ fois plus de km en $1$ heure.\\
   $2\times 6=12$\\
   Nassim court à ${\color[HTML]{f15929}\boldsymbol{12}}$ km/h.
\item D'après le cours, si $f=\dfrac{1}{u}$ alors $f'=\dfrac{-u'}{u^2}$.\\
    $f'(x)=\dfrac{-4x^{3}}{x^{8}}={\color[HTML]{f15929}\boldsymbol{-\dfrac{4}{x^{5}}}}$.
\item  Le coefficient multiplicateur  associé à une baisse de $20\,\%$ est $0{,}8$.\\
    Le coefficient multiplicateur global associé à ces deux diminutions est $0{,}8\times 0{,}8= 0{,}64$.\\
    On en déduit que le taux d'évolution globale est $0{,}64-1=-0{,}36$.\\
    La diminution globale est donc de ${\color[HTML]{f15929}\boldsymbol{36}} \,\%$.
\item $\begin{aligned}
            P(A\cap\overline{B}) & =P(A)\times P_A(\overline{B})\\
            &=0{,}4\times 0{,}1\\
              &={\color[HTML]{f15929}\boldsymbol{0{,}04}}
              \end{aligned}$
\end{enumerate}
\end{multicols}
\end{document}