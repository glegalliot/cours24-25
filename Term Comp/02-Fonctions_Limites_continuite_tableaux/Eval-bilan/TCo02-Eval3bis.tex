\documentclass[a4paper,11pt,eval]{nsi} 


%\pagestyle{empty}


\newcounter{exoNum}
\setcounter{exoNum}{0}
%
\newcommand{\exo}[1]
{
	\addtocounter{exoNum}{1}
	{\titlefont\color{UGLiBlue}\Large Exercice\ \theexoNum\ \normalsize{#1}}\smallskip	
}



\begin{document}



\textcolor{UGLiBlue}{Vendredi 10/01/2025}\\
\classe{\terminale Comp}
\titre{Évaluation-bilan 3 bis}
\maketitle
\begin{center}
	Calculatrice autorisée. Toutes les réponses doivent être justifiées.
\end{center}

\vspace*{.2cm}




\exo{}\bareme{12 pts}
\begin{enumerate}
    \item Soit $f$ la fonction définie sur $I=\fif{0}{40}$ par $\quad f(x)=(10x-10)e^{-0,1x}$.
    \begin{enumalph}
        \item Calculer $f(0)$ et $f(40)$.
        \item Démontrer que pour tout $x\in I, \quad f'(x)=(11-x)e^{-0,1x}$.
        \item Dresser le tableau de variations de la fonction $f$ sur $I=\fif{0}{40}$.
        \item Démontrer que l'équation $f(x)=20$ admet exactement deux solutions sur l'intervalle $\fif{0}{40}$.
    \end{enumalph}
    \carreauxseyes{16}{14.4}\\
    \carreauxseyes{16}{10.4}
    \item Une entreprise fabrique $x$ centaines d'ordinateurs, où $x$ appartient à l'intervalle $\fif{0}{40}$.
    On suppose que toute la production de l'entreprise est vendue et que le bénéfice, en milliers d'euros, de cette entreprise peut être modélisé par la fonction $f$ définie sur $\fif{0}{40}$ par $\quad f(x)=(10x-10)e^{-0,1x}$. 
    \begin{enumalph}
        \item Déterminer la perte de l'entreprise lorsqu'il n'y a pas de production.
        \item Déterminer le bénéfice maximal de l'entreprise. À quel nombre d'ordinateurs produits cela correspond-il ?	
        \item L'entreprise souhaite réaliser un bénéfice d'au moins 20 000 euros. Pour quel nombre d'ordinateurs produits cela est-il possible ?
    \end{enumalph}
    \carreauxseyes{16}{9.6}
\end{enumerate}

\exo{}\bareme{10 pts}\\
On définit la fonction $g$ sur $\oio{1}{+\infty}$ par $\quad g(x)=\dfrac{x^2+3}{x-1}$.
\begin{enumerate}
    \item Montrer que pour tout $x\in\oio{1}{+\infty}, \quad g'(x)=\dfrac{x^2-2x-3}{(x-1)^2}$.
    \item Calculer la limite de la fonction $g$ en $+\infty$.
    \item Calculer la limite de la fonction $g$ en 1.
    \item Étudier le signe de $x^2-2x-3$ pour $x$ appartenant à $\oio{1}{+\infty}$ puis dresser le tableau de variations de $g$.
\end{enumerate}
\carreauxseyes{16.8}{20}\\
\carreauxseyes{16.8}{25.6}
\end{document}