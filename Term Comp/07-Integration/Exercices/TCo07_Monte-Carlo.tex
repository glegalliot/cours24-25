\documentclass[a4paper,11pt,exos]{nsi} % COMPILE WITH DRAFT
\usepackage{pifont}
\usepackage{fontawesome5}
\usepackage{hyperref}

\pagestyle{empty}

\begin{document}
\classe{\terminale Comp}
\titre{Estimer une aire par la méthode de Monte-Carlo}
\maketitle

\begin{encadrecolore}{Un peu d'histoire}{UGLiDarkBlue}
    Le physicien gréco-américain \textbf{Nicholas Metropolis} (1915-1999) a inventé une méthode pour obtenir une estimation de l'aire de surfaces à l'aide de probalilités.\\
    C'est Stanislaw Ulam et John von Neumann qui donnèrent le nom de \textbf{Monte-Carlo} à cette méthode, en référence aux jeux de hasard pratiqués au casino de Monte-Carlo.
\end{encadrecolore}

\subsection*{Partie A - Principe de la méthode}
\dleft{11cm}{
    Dans un repère orthogonal $\repaff$, on considère la courbe d'équation $y=\sqrt{x}$ sur l'intervalle $\fif{0}{1}$ et le point $\pc{K}{1}{1}$.\\[.5em]
    On choisit au hasard un point $M$ dans le carré $OIKJ$.\\
    La probabilité que le point $M$ se trouve dans le domaine hachuré est 
    $$p=\dfrac{\text{aire du domaine hachuré}}{\text{aire du carré OIKJ}}.$$
}
{
    \def\xmin{-0.25}	\def\xmax{1.25}	\def\ymin{-0.25}	\def\ymax{1.25}
    \def\F{\x^(.5)}
    \begin{tikzpicture}[scale=3]
        \fill[UGLiOrange] (0,0) -- plot[thick,domain=0:1,smooth,variable=\x] ({\x},{\F}) -- (0,1) --cycle;
	    \repereal{\xmin}{\ymin}{\xmax}{\ymax}
	    \clip	(\xmin,\ymin) rectangle (\xmax,\ymax);
        \draw (1,1) node[above right] {K};
	    \fill[pattern =north east lines, pattern color = UGLiDarkBlue] (0,0) -- plot[thick,domain=0:1,smooth,variable=\x] ({\x},{\F}) -- (1,0) --cycle;
	    \draw[UGLiRed,ultra thick,domain=0:1,smooth,variable=\x]  plot ({\x},{\F});
    \end{tikzpicture}
}

\begin{enumerate}
    \item Donner l'aire du carré OIKJ.
    \item Soit un point $\pc{M}{x}{y}$ du plan.
    \begin{enumerate}
        \item À quelles conditions portant sur $x$ et $y$ le point $M$ appartient-il au carré OIKJ ?
        \item À quelles conditions portant sur $x$ et $y$ le point $M$ appartient-il au domaine hachuré ? 
    \end{enumerate}
\end{enumerate}

\subsection*{Partie B - Utilisation d'une fonction Python}
À l'aide de l'activité Capytale \textbf{77c8-6837146}, donner une valeur approchée de l'aire du domaine hachuré.

\subsection*{Partie C - Calcul exact de l'aide du domaine}
\begin{enumerate}
    \item Donner l'expression de l'aire, en unités d'aires, du domaine hachuré à l'aide d'une intégrale.
    \item Démontrer que la fonction $F$ définie sur $\fio{0}{+\infty}$ par $F(x)=\dfrac{2}{3}x\sqrt{x}$ est une primitive sur $\oio{0}{+\infty}$ de la fonction $f$ définie sur $\fio{0}{+\infty}$ par $f(x)=\sqrt{x}$.
    \item Calculer l'aire du domaine hachuré et comparer le résultat avec l'estimation obtenue à l'aide de la méthode de Monte-Carlo dans la \textbf{partie B}.
\end{enumerate}
\end{document}


