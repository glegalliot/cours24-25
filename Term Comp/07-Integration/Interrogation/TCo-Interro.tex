\documentclass[a4paper,11pt,eval]{nsi} 

\usepackage{pifont}
\usepackage{fontawesome5}
%\pagestyle{empty}


\newcounter{exoNum}
\setcounter{exoNum}{0}
%
\newcommand{\exo}[1]
{
	\addtocounter{exoNum}{1}
	{\titlefont\color{UGLiBlue}\Large Exercice\ \theexoNum\ \normalsize{#1}}\smallskip	
}



\begin{document}



\textcolor{UGLiBlue}{Vendredi 16/05/2025}\\
\classe{\premiere spé}
\titre{Interrogation }
\maketitle
\begin{center}
	Calculatrice interdite
\end{center}

\begin{enumerate}
    \item \hspace*{1cm}\\
    \dleft{8cm}{
    \begin{enumalph}
    \item Représenter dans le repère ci-contre la fonction $f$ définie sur $\mathbb{R}$ par : $\quad f(x) = 2$.
    \item En déduire la valeur de l'intégrale $\displaystyle\int_{-3}^{1} 2 \, dx$.\\[.5em]
\end{enumalph}
}
{
    \def\xmin{-5} \def\ymin{-3}\def\xmax{5}\def\ymax{4}
    \def\F{.5*\x}
    \begin{tikzpicture}[xscale=.725,yscale=.65]
    \draw[fill=white] (\xmin,\ymin) rectangle (\xmax,\ymax);
		%\fill[UGLiOrange!30] 	(2,0)-- (-2,0) -- plot[thick,domain=-2:2,smooth,variable=\x] ({\x},{\F})  --cycle;
        %\draw[fill = UGLiOrange!30](-3,0) rectangle (1,2);
        \repereal{\xmin}{\ymin}{\xmax}{\ymax}
        \clip (\xmin,\ymin) rectangle (\xmax,\ymax);
        %\draw[UGLiRed,domain=\xmin:\xmax,smooth,variable=\x,thick] plot ({\x},{\F});
    \end{tikzpicture}\\}
    \carreauxseyes{16cm}{2.4cm}

    \item \hspace*{1cm}\\
    \dleft{8cm}{
    \begin{enumalph}
    \item Représenter dans le repère ci-contre la fonction $g$ définie sur $\mathbb{R}$ par : $\quad g(x) = \dfrac{1}{2}x+1$.
    \item En déduire la valeur de l'intégrale $\displaystyle\int_{0}^{4} \left(\dfrac{1}{2}x+1\right) \, dx$.\\[.5em]
\end{enumalph}
}
{
    \def\xmin{-3} \def\ymin{-2}\def\xmax{7}\def\ymax{5}
    \def\F{.5*\x}
    \begin{tikzpicture}[xscale=.725,yscale=.65]
    \draw[fill=white] (\xmin,\ymin) rectangle (\xmax,\ymax);
		%\fill[UGLiOrange!30] 	(2,0)-- (-2,0) -- plot[thick,domain=-2:2,smooth,variable=\x] ({\x},{\F})  --cycle;
        %\draw[fill = UGLiOrange!30](-3,0) rectangle (1,2);
        \repereal{\xmin}{\ymin}{\xmax}{\ymax}
        \clip (\xmin,\ymin) rectangle (\xmax,\ymax);
        %\draw[UGLiRed,domain=\xmin:\xmax,smooth,variable=\x,thick] plot ({\x},{\F});
    \end{tikzpicture}\\
}
    \carreauxseyes{16cm}{2.4cm}
\newpage
    \item \hspace*{1cm}\\
    \dleft{8cm}{
    \begin{enumalph}
    \item Représenter dans le repère ci-contre la fonction $h$ définie sur $\mathbb{R}$ par : $\quad h(x) = -x+2$.
    \item En déduire la valeur de l'intégrale $\displaystyle\int_{-1}^{4} \left(-x+2\right) \, dx$.\\[.5em]
\end{enumalph}
}
{
    \def\xmin{-3} \def\ymin{-4}\def\xmax{7}\def\ymax{5}
    \def\F{.5*\x}
    \begin{tikzpicture}[xscale=.725,yscale=.65]
    \draw[fill=white] (\xmin,\ymin) rectangle (\xmax,\ymax);
		%\fill[UGLiOrange!30] 	(2,0)-- (-2,0) -- plot[thick,domain=-2:2,smooth,variable=\x] ({\x},{\F})  --cycle;
        %\draw[fill = UGLiOrange!30](-3,0) rectangle (1,2);
        \repereal{\xmin}{\ymin}{\xmax}{\ymax}
        \clip (\xmin,\ymin) rectangle (\xmax,\ymax);
        %\draw[UGLiRed,domain=\xmin:\xmax,smooth,variable=\x,thick] plot ({\x},{\F});
    \end{tikzpicture}\\
}
    \carreauxseyes{16cm}{4cm}
\end{enumerate}


\end{document}