\documentclass[a4paper,11pt,eval]{nsi} 
\usepackage{fontawesome5}

%\pagestyle{empty}


\newcounter{exoNum}
\setcounter{exoNum}{0}
%
\newcommand{\exo}[1]
{
	\addtocounter{exoNum}{1}
	{\titlefont\color{UGLiBlue}\Large Exercice\ \theexoNum\ \normalsize{#1}}\smallskip	
}



\begin{document}



\textcolor{UGLiBlue}{Mercredi 28/05/2025}\\
\classe{\terminale Comp}
\titre{Évaluation-bilan 7}
\maketitle
\begin{center}
	Calculatrice autorisée. Toutes les réponses doivent être justifiées.
\end{center}

%\vspace*{1cm}





\exo{}\bareme{10 pts}\\
\textbf{Dans cet exercice, les questions 1, 2, 3 et 4 peuvent être traitées de façon indépendante les unes des autres.}

Un parachutiste est en chute libre dans l'air jusqu'à l'instant $t = 0$ où il ouvre son parachute. Sa vitesse est alors de 50~m.s$^{-1}$. On admet par la suite que sa vitesse $v$, en m.s$^{-1}$, en fonction du temps $t$, en $s$, est solution de l'équation différentielle sur l'intervalle $[0\;;\;+\infty[$:

\[(E)\;:\quad y'=-5y+10.\]



\begin{enumerate}
    \item La fonction constante $g$ définie sur l'intervalle $[0\;;\;+\infty[$ par $g(t)=2$ est-elle une solution de l'équation différentielle $(E)$ ? 
    Justifier la réponse.\\[.5em]
    \carreauxseyes{16}{6.4}

    \item Montrer que les solutions de l'équation différentielle $(E)$ sur l'intervalle $[0\;;\;+\infty[$ sont les fonctions $f$ définies sur cet intervalle par $f(t)=k\e^{-5t}+2$, où $k$ est un nombre réel donné.\\[.5em]
    \carreauxseyes{16}{4}\\
    \carreauxseyes{16}{5.6}

    \item En admettant le résultat de la question précédente, montrer que la fonction $v$ est donnée sur $[0\;;\;+\infty[$ par $v(t) = 48 \e^{-5t} + 2$.\\[.5em]
    \carreauxseyes{16}{6.4}

    \item La distance parcourue, en mètre, par le parachutiste pendant les 10 premières secondes après ouverture du parachute est donnée  par l'intégrale :
    \[\displaystyle\int_{0}^{10} \left ( 48 \e^{-5t} +2 \right ) \ dt\]
    Calculer cette intégrale (arrondir à $10^{-1}$).\\[.5em]
    \carreauxseyes{16.8}{8}
\end{enumerate}

\newpage
\exo{}\bareme{10 pts}\\
Le tableau suivant donne l'évolution du prix $p_i$, en euros, d'une action en Bourse depuis son introduction.
\begin{center}
    \tabstyle[UGLiBlue]
    \begin{tabular}{|c|c|c|c|c|c|c|}
        \hline
        \ccell Jour $j_i$ & 2 & 7 & 12 & 16 & 20 & 25 \\
        \hline
        \ccell $p_i$ & 13,6 & 13,8 & 14,3 & 14,2 & 14,9 & 15,3 \\
        \hline
    \end{tabular}
\end{center}

\begin{enumerate}
    \item À l'aide de la calculatrice, déterminer le coefficient de corrélation linéaire entre les variables $p$ et $j$. \textit{Arrondir au millième.}\\[.5em]
    \carreauxseyes{16}{4}
    \item Peut-on envisager un ajustement affine du nuage ? Si oui, donner l'équation de la droite de régression de $p$ en $j$. \textit{Arrondir les coefficients au millième.}\\[.5em]
    \carreauxseyes{16}{4}
    \item Peut-on estimer le prix de cette action au 30$^{\text{e}}$ jour ? Si oui, donner la valeur estimée.\\[.5em]
    \carreauxseyes{16}{5.6}\\
    Quelle fiabilité peut-on accorder à cette estimation ?\\[.5em]
    \carreauxseyes{16}{3.2}
    \item Estimer le nombre de jours nécessaires pour que le prix de l'action atteigne 14 euros.\\[.5em]
    \carreauxseyes{16}{24.8}
\end{enumerate}

\newpage
\begin{flushleft}
\textbf{Question 1}
\end{flushleft}

Soit $g$ la fonction constante définie sur l'intervalle $[0\;;\;+\infty[$ par $g(t)=2$.

$g'(t)=0$ et $-5g(t)+10=-5\times 2 +10=0$ donc $g'(t)=-5g(t)+10$.

Donc $g$ est solution de l'équation différentielle $(E)$.

\begin{flushleft}
\textbf{Question 2}
\end{flushleft}

D'après le cours, les solutions de l'équation différentielle $y'=ay$ sur l'intervalle $[0\;;\;+\infty[$ sont les fonctions $f$ définies sur cet intervalle par $f(t)=k\e^{at}$,  où $k$ est un nombre réel quelconque, donc les solutions de l'équation différentielle $y'=-5y$ sur l'intervalle $[0\;;\;+\infty[$ sont les fonctions $f$ définies sur cet intervalle par $f(t)=k\e^{-5t}$,  où $k$ est un nombre réel quelconque.

Une solution de l'équation différentielle $y'=-5y+10$ est la somme d'une solution de l'équation différentielle $y'=-5y$ et d'une solution constante de l'équation différentielle $y'=-5y+10$, donc les solutions de l'équation différentielle $(E)$ sur l'intervalle $[0\;;\;+\infty[$ sont les fonctions $f$ définies sur cet intervalle par $f(t)=k\e^{-5t}+2$,  où $k$ est un nombre réel quelconque.

\begin{flushleft}
\textbf{Question 3}
\end{flushleft}

On sait que $v$ est solution de $(E)$ et que $v(0)=50$; donc $k\e^{0}+2 = 50$ donc $k=48$.

La fonction $v$ est donc donnée sur $[0\;;\;+\infty[$ par $v(t) = 48 \e^{-5t} + 2$.

\begin{flushleft}
\textbf{Question 4}
\end{flushleft}

La distance parcourue, en mètre, par le parachutiste pendant les 10 premières secondes après ouverture du parachute est donnée  par l'intégrale:
$\displaystyle\int_{0}^{10} \left ( 48 \e^{-5t} +2 \right ) \d t$.

Pour calculer cette intégrale, il faut trouver une primitive de la fonction $v$.

La fonction $t\longmapsto \e^{at}$ avec $a\neq 0$,  a pour primitive la fonction $t\longmapsto \dfrac{\e^{at}}{a}$, donc la fonction $v$ a pour primitive la fonction $V$ définie par $V(t) = 48 \dfrac{\e^{-5t}}{-5} + 2t$ soit $V(t) = - 9,6 \e^{-5t} +2t$.

$\begin{aligned}
\displaystyle\int_{0}^{10} \left ( 48 \e^{-5t} +2 \right ) \ dt&
= \left [ V(t) \strut\right ]_{0}^{10}
= V(10) - V(0)
= \left (-9,6 \e^{-5\times 10} + 2 \times 10 \right ) - \left ( -9,6 \e^{-5\times 0} + 2 \times 0 \right )\\
&
= -9,6 \e^{-50} + 20+9,6  = 29,6 -9,6 \e^{-50} \approx 29,6
\end{aligned}$


\end{document}