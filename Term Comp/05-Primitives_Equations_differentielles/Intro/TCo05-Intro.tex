\documentclass[a4paper,11pt,exos]{nsi} 
\usepackage{pifont}
\usepackage{fontawesome5}

\pagestyle{empty}



\begin{document}



\classe{\terminale Comp}
\titre{Dynamique des populations :\\ du discret au continu}
\maketitle

\dleft{11.5cm}{
En 1950, un pays comptait 30,5 millions d'habitants.\\
Depuis cette date, sa population a un taux annuel moyen de natalité de 20 pour 1000, c'est-à-dire qu'il y a en moyene 20 naissances enregistrées au cours d'une année pour 1000 habitants.\\
De façon analogue, depuis 1950, le taux annuel moyen de mortalité est de 15 pour 1000.\\
De plus, chaque année en moyenne, 100 000 nouveaux arrivants s'installent dans ce pays.
}
{\includegraphics[width=5cm]{baby-feet-2612403_640.jpg}}


\textbf{On cherche à modéliser l'évolution de la population de ce pays.}

\subsection*{Partie 1 : Modèle à temps discret}
On note $p(n)$ la population de ce pays, en millions d'habitants, l'année $1950+n$ avec $n$ entier naturel. Ainsi $p(0)=30,5$.
\begin{enumerate}
    \item À l'aide des informations données sur l'évolution de cette population, expliquer pourquoi pour tout entier naturel $n$, on a $\quad p(n+1)-p(n)=0,005p(n)+0,1$.
    \item Estimer la population de ce pays en 2050, si les conditions d'évolution de la population restent les mêmes.
\end{enumerate}

\subsection*{Partie 2 : Modèle à temps continu}
On considère la fonction $p$ définie sur l'intervalle $\fio{0}{+\infty}$ qui, à un instant $1950+t$, en années, associe la population de ce pays, en millions d'habitants.\\
Ainsi $p(0)=30,5$ et pour tout réel $t\geqslant 0, \quad p(t+1)-p(t)=0,005p(t)+0,1$ \textbf{$(R)$}.
\begin{enumerate}
    \item On suppose que la fonction $p$ est dérivable sur $\fio{0}{+\infty}$.\\[.5em]
    On approche $p(t+1)-p(t)=\dfrac{p(t+1)-p(t)}{1}$ par $p'(t)$.\\[.5em]
    Que devient la relation $(R)$ avec cette approximation ?
    \begin{encadrecolore}{Info :}{UGLiDarkBlue}
        Pour tout réel $t\geqslant 0$, on a $p'(t)=\lim\limits_{h\to 0}\dfrac{p(t+h)-p(t)}{h}$.\\[.5em]
        En démographie, en économie, etc., il est fréquent d'approcher $p'(t)$ par $p(t+1)-p(t)$.
    \end{encadrecolore}  
    \item Vérifier que toute fonction définie sur $\fio{0}{+\infty}$ par $p(t)=k e^{0,005t}-20$, où $k$ est un réel, vérifie la relation obtenue à la question \textbf{1.}.
    \item Utiliser la condition initiale $p(0)=30,5$ pour déterminer la valeur de $k$.
    \item En supposant que l'évolution se poursuive ainsi, estimer la population de ce pays en 2050.\\
    Comparer cette estimation à celle obtenue à la question \textbf{2.} de la partie 1.
\end{enumerate}
\end{document}