\documentclass[a4paper,11pt,eval]{nsi} 
\usepackage{fontawesome5}

%\pagestyle{empty}


\newcounter{exoNum}
\setcounter{exoNum}{0}
%
\newcommand{\exo}[1]
{
	\addtocounter{exoNum}{1}
	{\titlefont\color{UGLiBlue}\Large Exercice\ \theexoNum\ \normalsize{#1}}\smallskip	
}



\begin{document}



\textcolor{UGLiBlue}{Vendredi 02/05/2025}\\
\classe{\terminale Comp}
\titre{Évaluation-bilan 6}
\maketitle
\begin{center}
	Calculatrice autorisée. Toutes les réponses doivent être justifiées.
\end{center}

%\vspace*{1cm}





\exo{}\bareme{7 pts}\\
$(E)$ est l'équation différentielle  : 
$$y'=-5y+7$$
\begin{enumerate}
    \item Déterminer la solution constante de $(E)$.
    \item Résoudre sur $\R$ l'équation différentielle $y'=-5y$.
    \item En déduire toutes les solutions de $(E)$.
    \item Déterminer la solution $f$ de $(E)$ telle que $f(0)=2$.
\end{enumerate}

\carreauxseyes{16.8}{13.6}\\
\carreauxseyes{16.8}{4.8}\\

\exo{}\bareme{3 pts}\\
$(E)$ est l'équation différentielle  : 
$$y'=2x^3-4x+1$$
Déterminer la solution $g$ de $(E)$ telle que $g(0)=2$.\\

\carreauxseyes{16.8}{9.6}\\



\exo{}\bareme{10 pts}\\
Afin de chauffer un liquide, on fait passer un courant électrique dans une résistance.\\
La température, en °C, du liquide à l'instant $t$, en secondes, est noté $T(t)$.\\
On admet que la fonction $T$, définie sur $\fif{0}{80}$, est solution de l'équation différentielle :
$$(E) \quad T'=-0{,}02T+1$$
\begin{enumerate}
    \item Interpréter l'information $T(0)=20$.
    \item Résoudre $(E)$ sur $\fif{0}{80}$.
    \item Déterminer la solution de $(E)$ qui vérifie la condition initiale $T(0)=20$.
    \item Déterminer l'instant $t_0$, en s, à partir duquel la température du liquide dépasse 40 °C. \textit{On arrondira au dixième de seconde.}
\end{enumerate}
\carreauxseyes{16.8}{25.6}\\

\exo{}\bareme{4 pts}\\
Déterminer l'expression de la fonction dont la courbe représentative passe par le point $A$ de coordonnées $(1\,;\,3)$ et telle qu'en chaque point $M$ de cette courbe, le coefficient directeur de la tangente est égal au double de l'ordonnée du point $M$.\\

\carreauxseyes{16.8}{21.6}\\

\end{document}