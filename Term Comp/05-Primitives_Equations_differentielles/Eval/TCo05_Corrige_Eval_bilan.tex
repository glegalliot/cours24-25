\documentclass[a4paper,11pt,exos]{nsi} 
\usepackage{fontawesome5}

%\pagestyle{empty}


\begin{document}



%\textcolor{UGLiBlue}{Mercredi 05/02/2025}\\
\classe{\terminale Comp}
\titre{Corrigé de l'évaluation-bilan 6}
\maketitle



\exo{}
$(E)$ est l'équation différentielle  : 
$$y'=-5y+7$$
\begin{enumerate}
    \item Déterminer la solution constante de $(E)$.
    \item Résoudre sur $\R$ l'équation différentielle $y'=-5y$.
    \item En déduire toutes les solutions de $(E)$.
    \item Déterminer la solution $f$ de $(E)$ telle que $f(0)=2$.
\end{enumerate}

\textcolor{UGLiBlue}{
    \begin{enumerate}
        \item Soit $y_0$ une solution particulière constante de $(E)$.\\
        On a donc : $y_0' = 0$ et $-5y_0 + 7 = 0$.\\
        On en déduit : $y_0 = \dfrac{7}{5}$.
        \item On résout l'équation différentielle $y'=-5y$ :\\
        Les solution de cette équation homogène sont les fonctions définies sur $\R$ par : $y(x) = ke^{-5x}$, avec $k \in \R$.\\
        \item Les solutions de $(E)$ sont donc les fonctions définies sur $\R$ par :
        $$y(x) = ke^{-5x} + \frac{7}{5}$$
        avec $k \in \R$.
        \item Soit $f$ la solution de $(E)$ telle que $f(0)=2$.
        \begin{tabbing}
            $f(0)=2 \quad$ \= $\iff\quad ke^{-5 \times 0} + \dfrac{7}{5} = 2$\\
            \> $\iff\quad k + \dfrac{7}{5} = 2$\\
            \> $\iff\quad k = 2 - \dfrac{7}{5}$\\
            \> $\iff\quad k = \dfrac{10}{5} - \dfrac{7}{5}$\\
            \> $\iff\quad k = \dfrac{3}{5}$
        \end{tabbing}
        Donc la solution $f$ de $(E)$ telle que $f(0)=2$ est la fonction définie sur $\R$ par :
        $$f(x) = \dfrac{3}{5}e^{-5x} + \dfrac{7}{5}$$
    \end{enumerate}
}

\exo{}
$(E)$ est l'équation différentielle  : 
$$y'=2x^3-4x+1$$
Déterminer la solution $g$ de $(E)$ telle que $g(0)=2$.\\

\textcolor{UGLiBlue}{
    Les solutions de $(E)$ sont les primitives de la fonction $x\mapsto 2x^3-4a+1$.\\
    On a donc :
    \begin{tabbing}
        Les solutions de $(E)$ sont les fonctions définies sur $\R$ par :
    $f(x)$ \=$= \dfrac{2}{4}x^4 - \dfrac{4}{2}x^2 + x + C$, avec $C \in \R$.\\[.5em]
        \> $= \dfrac{1}{2}x^4 - 2x^2 + x + C$
    \end{tabbing}
    Soit $g$ la solution de $(E)$ telle que $g(0)=2$.\\
    \begin{tabbing}
        $g(0)=2 \quad$ \= $\iff\quad \dfrac{1}{2}\times 0^4 - 2 \times 0^2 + 0 + C = 2$\\
        \> $\iff\quad C = 2$
    \end{tabbing}
    Donc la solution $g$ de $(E)$ telle que $g(0)=2$ est la fonction définie sur $\R$ par :
    $$g(x) = \dfrac{1}{2}x^4 - 2x^2 + x + 2$$
}

\exo{}
Afin de chauffer un liquide, on fait passer un courant électrique dans une résistance.\\
La température, en °C, du liquide à l'instant $t$, en secondes, est noté $T(t)$.\\
On admet que la fonction $T$, définie sur $\fif{0}{80}$, est solution de l'équation différentielle :
$$(E) \quad T'=-0{,}02T+1$$
\begin{enumerate}
    \item Interpréter l'information $T(0)=20$.
    \item Résoudre $(E)$ sur $\fif{0}{80}$.
    \item Déterminer la solution de $(E)$ qui vérifie la condition initiale $T(0)=20$.
    \item Déterminer l'instant $t_0$, en s, à partir duquel la température du liquide dépasse 40 °C. \textit{On arrondira au dixième de seconde.}
\end{enumerate}

\textcolor{UGLiBlue}{
    \begin{enumerate}
        \item La température du liquide à l'instant $t=0$ est de 20 °C.
        \item On résout l'équation différentielle $(E)$ :
        \begin{enumerate}[label=\textbullet]
            \item \textbf{Recherche d’une solution particulière constante :}\\
            Soit $T_0$ une solution particulière constante de $(E)$.\\
            On a donc : $T_0' = 0$ et $-0{,}02T_0 + 1 = 0$.\\
            On en déduit : $T_0 = \dfrac{1}{0{,}02} = 50$.
            \item \textbf{Recherche des solutions de l'équation homogène :}\\
            On résout l'équation différentielle $T'=-0{,}02T$ :\\
            Les solutions de cette équation homogène sont les fonctions définies sur $\fif{0}{80}$ par :
            $$T(t) = ke^{-0{,}02t}$$
            avec $k \in \R$.
            \item \textbf{Conclusion:}\\
            Les solutions de $(E)$ sont donc les fonctions définies sur $\fif{0}{80}$ par :
            $$T(t) = ke^{-0{,}02t} + 50$$
            avec $k \in \R$.
        \end{enumerate}
        \item Soit $T$ la solution de $(E)$ telle que $T(0)=20$.
        \begin{tabbing}
            $T(0)=20 \quad$ \= $\iff\quad ke^{-0{,}02\times 0} + 50 = 20$\\
            \> $\iff\quad k + 50 = 20$\\
            \> $\iff\quad k = 20 - 50$\\
            \> $\iff\quad k = -30$
        \end{tabbing}
        Donc la solution $T$ de $(E)$ telle que $T(0)=20$ est la fonction définie sur $\fif{0}{80}$ par :
        $$T(t) = -30e^{-0{,}02t} + 50$$
        \item On cherche l'instant $t_0$, en s, à partir duquel la température du liquide dépasse 40 °C.\\
        On résout donc l'inéquation : $T(t) > 40$
        \begin{tabbing}
            $T(t) > 40 \quad$ \= $\iff\quad -30e^{-0{,}02t} + 50 > 40$\\
            \> $\iff\quad -30e^{-0{,}02t} > 40 - 50$\\
            \> $\iff\quad -30e^{-0{,}02t} > -10$\\
            \> $\iff\quad e^{-0{,}02t} < \dfrac{1}{3}$\\
            \> $\iff\quad -0{,}02t < \ln\left(\dfrac{1}{3}\right)$\\
            \> $\iff\quad t > -\dfrac{\ln\left(\dfrac{1}{3}\right)}{0{,}02}$\\
            \> $\iff\quad t > 50\ln(3)$    
        \end{tabbing}
        La valeur approchée de $t_0=50\ln(3)$ est $t_0 \approx 54{,}9$ s.\\
        Donc la température du liquide dépasse 40 °C à partir de l'instant $t_0 \approx 54{,}9$ s.
    \end{enumerate}
}

\exo{}
Déterminer l'expression de la fonction dont la courbe représentative passe par le point $A$ de coordonnées $(1\,;\,3)$ et telle qu'en chaque point $M$ de cette courbe, le coefficient directeur de la tangente est égal au double de l'ordonnée du point $M$.\\

\textcolor{UGLiBlue}{
    Soit $f$ la fonction dont la courbe représentative passe par le point $A$ de coordonnées $(1\,;\,3)$ et telle qu'en chaque point $M$ de cette courbe, le coefficient directeur de la tangente est égal au double de l'ordonnée du point $M$.\\
    On a donc : $f'(x) = 2f(x)$.
    \begin{enumerate}[label=\textbullet]
        \item On résout l'équation différentielle $f'(x) = 2f(x)$ :\\
        Les solutions de cette équation homogène sont les fonctions définies sur $\R$ par :
        $$f(x) = ke^{2x}$$
        avec $k \in \R$.
        \item Soit $f$ la solution de $(E)$ telle que $f(1)=3$.
        \begin{tabbing}
            $f(1)=3 \quad$ \= $\iff\quad ke^{2\times 1} = 3$\\
            \> $\iff\quad ke^{2} = 3$\\
            \> $\iff\quad k = 3e^{-2}$
        \end{tabbing}
        Donc la fonction dont la courbe représentative passe par le point $A$ de coordonnées $(1\,;\,3)$ et telle qu'en chaque point $M$ de cette courbe, le coefficient directeur de la tangente est égal au double de l'ordonnée du point $M$ est la fonction définie sur $\R$ par :
        $$f(x) = 3e^{-2}e^{2x}$$
    \end{enumerate}
}
\end{document}