\documentclass[a4paper,11pt,eval]{nsi} 


%\pagestyle{empty}


\newcounter{exoNum}
\setcounter{exoNum}{0}
%
\newcommand{\exo}[1]
{
	\addtocounter{exoNum}{1}
	{\titlefont\color{UGLiBlue}\Large Exercice\ \theexoNum\ \normalsize{#1}}\smallskip	
}



\begin{document}



\textcolor{UGLiBlue}{Mercredi 13/11/2024}\\
\classe{\terminale Comp}
\titre{Évaluation-bilan 2}
\maketitle
\begin{center}
	Calculatrice autorisée. Toutes les réponses doivent être justifiées.
\end{center}

\vspace*{1cm}

\exo{}\bareme{4 pts}\\
Soit $(u_n)$ la suite définie par $u_0=1$ et pour tout $n\in\N, \quad u_{n+1}=\dfrac{3}{4}u_n+1$.
\begin{enumerate}
    
        \item Représenter graphiquement la fonction $f$ définie sur $\R$ par $\quad f(x)=\dfrac{3}{4}x+1$.
        \item Représenter sur le graphique les quatre premiers termes de la suite $(u_n)$.
        \item Conjecturer les variations et la limite de la suite.
    \end{enumerate}
    \begin{center}
        \def\xmin{-3} \def\ymin{-3}\def\xmax{10}\def\ymax{6}
        \begin{tikzpicture}[scale=1]
            \clip (\xmin,\ymin) rectangle (\xmax,\ymax);
            \draw[fill = white] (\xmin,\ymin) rectangle (\xmax,\ymax);
            \reperenb{\xmin}{\ymin}{\xmax}{\ymax}{}{}
        \end{tikzpicture}
    \end{center}
    \carreauxseyes{16}{4}


\exo{}\bareme{10 pts}\\
Dans une ruche formée de 50 000 abeilles, on estime à 20 \% la proportion d'abeilles de la colonie qui décèdent chaque jour à cause d'un insecticide. On suppose que le nombre de naissances et de décès de manière naturelle reste identique (1000 naissances et 500 décès de manière naturelle par jour).\\
Pour tout entier naturel $n$, on note $u_n$ le nombre d'individus de la colonie $n$ jours après le début des pulvérisations de l'insecticide. On a donc $u_0=50\ 000$.
\begin{enumerate}
    \item Montrer que, pour tout entier naturel $n, \quad u_{n+1}=0,8u_n+500$.\\[.5em]
    \carreauxseyes{16}{6.4}
    \item Déterminer l'expression de $u_n$ en fonction de $n$.\\[.5em]
    \carreauxseyes{16}{13.6}\\
    \carreauxseyes{16}{11.2}
    \item Quel est le sens de variation de la suite $(u_n)$ ? Justifier la réponse et interpréter avec le contexte.\\[.5em]
    \carreauxseyes{16}{5.6}
    \item Des études ont montré qu'une colonie d'abeilles n'est plus en mesure d'assurer sa survie si elle compte moins de 5000 individus. La colonie étudiée va-t-elle survivre ? Justifier.\\[.5em]
    \carreauxseyes{16}{5.6}
\end{enumerate}

\exo{}\bareme{6 pts}\\
Un globe-trotter s'est fixé pour but de parcourir une distance de 4500 kilomètres.\\
Le jour du départ (jour zéro), il en parcourt 40, mais chaque jour, fatigué par l'effort, il parcourt 1\% de moins que la veille.
\begin{enumerate}
	\item 	On appelle $p_n$ la distance parcourue le n-ième jour.\\
	Déterminer la nature de la suite $(p_n)$. En déduire l'expression de $p_n$ en fonction de $n$.\\[0.5em]
	\carreauxseyes{16}{6.4}
    \item Quel est la distance totale parcourue après 15 jours ? Arrondir au kilomètre.\\[0.5em]
	\carreauxseyes{16}{6.4}
	\item 	Exprimer la distance totale parcourue au bout de $n$ jours. Le globe-trotter atteindra-t-il son but ?\\[0.5em]
	\carreauxseyes{16}{6.4}
\end{enumerate}
\end{document}