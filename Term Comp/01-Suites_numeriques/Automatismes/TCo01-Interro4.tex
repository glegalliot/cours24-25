\documentclass[a4paper,11pt,eval]{nsi} 


\pagestyle{empty}


\newcounter{exoNum}
\setcounter{exoNum}{0}
%
\newcommand{\exo}[1]
{
	\addtocounter{exoNum}{1}
	{\titlefont\color{UGLiBlue}\Large Exercice\ \theexoNum\ \normalsize{#1}}\smallskip	
}



\begin{document}



\textcolor{UGLiBlue}{Vendredi 11/10/2024}\\
\classe{\terminale Comp}
\titre{Interrogation 4}
\maketitle
\begin{center}
	Calculatrice interdite
\end{center}



\begin{enumerate}
    \item Déterminer la limite de la suite $(u_n)$ définie pour tout entier  $n$ strictement positif par : $$u_n=2+\dfrac{1}{n^9}-n^7$$
    \carreauxseyes{16}{7.2}
	\item Déterminer la limite de la suite $(v_n)$ définie pour tout entier positif $n$ par : $$v_n=5^n-4^n$$
	\carreauxseyes{16}{7.2}
\end{enumerate}
\end{document}

On sait que $\lim\limits_{n\to\infty} \dfrac{1}{n^9}=0$ et $\lim\limits_{n\to\infty} n^7=+\infty$.\\Ainsi, d'après les règles des limites de la somme, $\lim\limits_{n\to\infty} \dfrac{1}{n^9}-n^7={\color[HTML]{f15929}\boldsymbol{-\infty}}$.

\end{document}