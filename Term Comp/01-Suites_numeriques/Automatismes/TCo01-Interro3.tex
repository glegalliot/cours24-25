\documentclass[a4paper,11pt,eval]{nsi} 


\pagestyle{empty}


\newcounter{exoNum}
\setcounter{exoNum}{0}
%
\newcommand{\exo}[1]
{
	\addtocounter{exoNum}{1}
	{\titlefont\color{UGLiBlue}\Large Exercice\ \theexoNum\ \normalsize{#1}}\smallskip	
}



\begin{document}



\textcolor{UGLiBlue}{Mercredi 25/09/2024}\\
\classe{\terminale Comp}
\titre{Interrogation 3}
\maketitle
\begin{center}
	Calculatrice autorisée
\end{center}

%\vspace{1cm}

\begin{enumerate}
    \item Soit $u$ la suite arithmétique de premier terme $u_1 = 2$ et de raison $7$.
    \begin{enumalph}
        \item Pour $n\in\N$, donner $u_n$, le terme général de la suite $u$.
        \item Calculer $\mathcal{S}_{30} = u_1 + u_2 + ... + u_{30} $.
    \end{enumalph}
    \carreauxseyes{16}{7.2}
    \item Soit $v$ la suite géométrique de premier terme $v_1 = 10$ et de raison $2$.
    \begin{enumalph}
        \item Pour $n\in\N$, donner $v_n$, le terme général de la suite $v$.
        \item Calculer $\mathcal{S}_{15} = v_1 + v_2 + ... + v_{15}$.
    \end{enumalph}
    \carreauxseyes{16}{6.4}
\end{enumerate}

%\end{document}




\begin{multicols}{2}
\begin{enumerate}
    \item \begin{enumalph}
        \item \begin{tabbing}
            Soit $n\in\N.\quad u_n$\=$=u_1+(n-1)\times 7$\\
            \>  $=2+7(n-1)$\\
            \>  $=2+7n-7$\\
            \>  $=-5+7n$
        \end{tabbing}
        \item \begin{tabbing}
            $S_{30}$    \=  $=u_1 + u_2 + ... + u_{30} $\\[.5em]
            \>  $=30\times \dfrac{u_1+u_{30}}{2}$\\[.5em]
            \>  $=30\times \dfrac{2+(-5+7\times 30)}{2}$\\[.5em]
            \>  $=30\times \dfrac{2+205}{2}$\\[.5em]
            \>  $=30\times \dfrac{207}{2}$\\[.5em]
            \>  $=3105$
        \end{tabbing}
    \end{enumalph}

    \columnbreak

    \item \begin{enumalph}
        \item \begin{tabbing}
            Soit $n\in\N^*.\quad v_n$\=$=v_1\times 2^{n-1}$\\
            \>  $=10\times 2^{n-1}$
        \end{tabbing}
        \item \begin{tabbing}
            $S_{15}$    \=  $= v_1 + v_2 + ... + v_{15}$\\[.5em]
            \>  $=v_1\times \dfrac{1-2^{15}}{1-2}$\\[.5em]
            \>  $=10\times \dfrac{1-2^{15}}{1-2}$\\[.5em]
            \>  $= 10 \times 32\ 767$\\[.5em]
            \>  $= 327\ 670$
        \end{tabbing}
    \end{enumalph}
\end{enumerate}
\end{multicols}

\end{document}
