\documentclass[a4paper,11pt,exos]{nsi} % COMPILE WITH DRAFT
\usepackage{pifont}
\usepackage{fontawesome5}

\pagestyle{empty}

\begin{document}
\classe{\terminale Comp}
\titre{DL - Modèle de Malthus}
\maketitle

En 1798, Thomas Malthus publie \textit{An essay on the principle of population} dans lequel il émet l'hypothèse suivante : l'accroissement de la population, beaucoup plus rapide que celui des ressources alimentaires, conduira son pays à la famine.\\[.5em]
En 1800, la population de l'Angleterre était estimée à 8 millions d'habitants et l'agriculture anglaise pouvait nourrir 10 millions de personnes.\\
Malthus admet que la population augmente de 2,8 \% chaque année et que les progrès de l'agriculture permettent de nourrir 400 000 personnes de plus chaque année.\\[.5em]
Pour tout entier naturel $n$, on note $p_n$ la population (en millions) en 1800$+n$ et $a_n$ le nombre de personnes (en millions) pouvant être nourries par l'agriculture la même année. Ainsi $p_0=8$ et $a_0=10$. \\[.5em]
\textit{On arrondira si besoin les réponses à 0,001.}

\begin{enumerate}
    \item Donner la relation de récurrence permettant d'exprimer $p_{n+1}$ en fonction de $p_n$ pour $n\in\N$.\\
    En déduire l'expression de $p_n$ en fonction de $n$.
    
    \item Donner la relation de récurrence permettant d'exprimer $a_{n+1}$ en fonction de $a_n$ pour $n\in\N$.\\
    En déduire l'expression de $a_n$ en fonction de $n$.

    \item Calculer $p_{50}$ et $a_{50}$. 
    Interpréter ces résultats dans le contexte de l'exercice.

    \item Calculer les limites des suites $(p_n)$ et $(a_n)$.
   
    \item Représenter graphiquement les termes des suites $p$ et $a$ pour $n=0, 10, ..., 50$.\\
    Vers quelle année la situation semble-t-elle devenir critique ?
    \item À l'aide de la calculatrice, déterminer précisément en quelle année, selon ce modèle, la situation aurait dû devenir critique.
    \item Que s'est-il réellement passé en Angleterre ? \textit{Faire des recherches.}\\
    Quelles critiques peut-on faire du modèle proposé par Malthus ?
\end{enumerate}
\end{document}