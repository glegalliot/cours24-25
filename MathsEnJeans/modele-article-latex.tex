\documentclass[a4paper,french,11pt]{article}
\usepackage[francais]{babel}
\usepackage{mathtools,amssymb,amsthm}
\usepackage[utf8]{inputenc}
%%%%%%%%%%%%%%
%%%% dé-commenter pour mettre tout en police Calibri - nécessite de compiler avec XeLaTex ou LuaLaTex.
%%%% Commenter pour revenir en Fourier
%%%% dé-commenter/ commenter aussi la ligne après \begin{document}
%\usepackage[T1]{fontenc}
%\usepackage{fontspec}
%\setsansfont{Calibri}
%\renewcommand*\normalfont{\sffamily}
%%%%%%%%%%%%%%
\usepackage{fourier}
\usepackage{titlesec}
\usepackage{fancyhdr}
\usepackage{xcolor}
\usepackage{epsfig} 
\usepackage{epstopdf}
\usepackage{graphicx}
\usepackage{geometry}
\geometry{hmargin=2.5cm,tmargin=1.5cm, bmargin=3cm}
\setlength\hoffset{1em}
\setlength\headheight{32pt}
\setlength\parindent{1em}
\usepackage{hyperref}
\hypersetup{
	colorlinks=true,
	linkcolor=black, 
	urlcolor=[rgb]{0.05,0,0.25}
	}

\titleformat{\section}{\bfseries\Large}{\thesection.}{0.5em}{}
\titleformat{\subsection}{\bfseries\large}{\hspace{.5em}\thesubsection.}{0.5em}{}

\newtheorem{theo}{Théorème}
\newtheorem*{theo*}{Théorème}
\newtheorem{prop}{Proposition}
\newtheorem*{prop*}{Proposition}
\newtheorem{lemme}{Lemme}
\newtheorem*{lemme*}{Lemme}
\newtheorem{cor}{Corollaire}
\newtheorem*{cor*}{Corollaire}
\theoremstyle{definition}\newtheorem{defn}{Definition}
\newtheorem*{defn*}{Definition}
\theoremstyle{remark}\newtheorem{rem}{Remarque}
\newtheorem*{rem*}{Remarque}

%%%%%%%%%%%%%%
%%% Pour le comité d'édition, pour insérer des notes :
% À chaque appel de note, insérer : \appelnote{X} où X est un label (numéro ou autre)
% Le label sera enregistré avec le préfixe "note:" - si besoin d'une référence, utiliser \ref{note:X}.
% Les appels de note seront numérotés automatiquement dans l'ordre où ils sont insérés.
% Voir le paragraphe des notes à la fin pour ajouter le texte des notes.

\newcounter{note}
\newcommand\appelnote[1]{%
	\refstepcounter{note}\label{note:#1}\fboxsep0pt\colorbox{yellow}{\hypertarget{anote#1}{\hyperlink{note#1}{(\thenote)}}}%
}
\newcommand\note[1]{%
	\par\vspace{1ex}
 	\noindent\fboxsep0pt\colorbox{yellow}{\hypertarget{note#1}{\hyperlink{anote#1}{(\ref{note:#1})}}}\ %
}
%%%%%%%%%%%%%%
%% pied de page et entête
%
\pagestyle{fancy}
\renewcommand{\headrulewidth}{0pt}
\fancyhead{\empty}
\renewcommand{\footrulewidth}{.5pt}
\fancyfoot[L]{\sffamily MATh.en.JEANS 20..-20..\empty}
\fancyfoot[C]{\sffamily Collège de Bidule, Triffouillis-les-Oies}
\fancyfoot[R]{\sffamily Page \thepage}
%%%%%%%%%%%%%%


\begin{document}
%%%%%%%%%%%%%%
%%% dé-commenter pour mettre tout en police Calibri ou commenter pour revenir en Fourier
%\sffamily
%%%%%%%%%%%%%%
\thispagestyle{empty}
\begin{center}
\fcolorbox[gray]{0}{0.95}{%
	\begin{minipage}[c]{.975\linewidth}
		\begin{center}\sffamily
			Cet article est rédigé par des élèves. Il peut comporter des oublis ou des imperfections,\\
			autant que possible signalés par nos relecteurs dans les notes d'édition.
%		English version:
%			This article is written by students. It may include omissions and imperfections, which are reported as
%			far as possible by our reviewers in the editing notes.
		\end{center}
	\end{minipage}
}

\vspace{10ex}
\huge\sffamily\textbf{Un joli titre}

\vspace{3ex}
\large
Année 20..-20..

\vspace{3ex}
Toto et Titi, classe de 4ème
\end{center}
\vspace{2ex}
{\sffamily \large
\noindent{\'{E}tablissement(s)~:} Collège de Bidule, Triffouillis-les-Oies

\vspace{.5ex}
\noindent{Enseignant$\cdot$e(s)~:} Marcel Dugenou

\vspace{.5ex}
\noindent{Chercheur$\cdot$Chercheuse(s)~:} Eugène Duraton, Université de Perpette-les-Oies
}

%\tableofcontents  
\vspace{2ex}
\begin{itemize}\color{red}\em
	\item \'Evitez autant que possible d'utiliser d'autres packages que ceux proposés dans ce modèle.
    \item Ne diminuez pas les marges. 
    \item Les dessins et images doivent être légendés et lisibles. Les fournir dans des fichiers séparés,\\ regroupés dans une archive.
    \item N'insérez pas de textes ou de formules sous forme d'images.
\end{itemize}
\section{Introduction}

\subsection{Sujet}
{\em Le problème doit être posé en termes clairs}

\subsection{Résultats}
{\em Si votre article est un peu long, annoncez ici les principaux résultats et conjectures obtenus}

\section{Quelques pistes pour rédiger}
Penser que les lecteurs et lectrices ne connaissent rien à votre sujet, contrairement à vous qui y êtes plongé$\cdot$es depuis plusieurs mois. 

Ce qui vous parait évident, parce que vous y avez beaucoup réfléchi durant toute l’année, peut ne pas l’être pour les personnes qui liront votre article, en particulier pour des élèves du même niveau que vous.

Il faut donc tout leur expliquer : 
\begin{itemize}
    \item bien définir les termes et notations utilisés,
    \item vérifier la cohérence des notations à la fois dans le texte, et entre texte et illustrations,
    \item faire des figures claires et explicites, si nécessaire accompagnées d'une légende,
\end{itemize}
et surtout 
\begin{itemize}
    \item justifier tout ce qui est affirmé.
\end{itemize}
À éviter : 
\begin{itemize}
    \item trop de commentaires personnels, 
    \item les illustrations inutiles.
\end{itemize}
\section{{Une section}}
{\color{red}{\em Mettez autant de sections que vous le souhaitez, avec des titres appropriés \ldots}}

\subsection{Une sous-section}
{\color{red}{\em Et de même pour les sous-sections}}

\vspace{2ex}

\emph{Exemples d'environnements mathématiques (versions avec ou sans numérotation).}
\begin{defn*} Une définition
\end{defn*}
\begin{theo*}[de Pythagore]
	Un théorème classique utilisé ici.
\end{theo*}
\begin{prop*}
	Un premier résultat
\end{prop*}
\begin{theo}\label{montheo}
	Le théorème principal.
\end{theo}
\begin{lemme}\label{monlemme}
	Unrésultat intermédiaire qui servira à démontrer le théorème \ref{montheo}.
\end{lemme}
\begin{proof}
	Bla, bla, bla\ldots
	
	Bla, bla, bla\ldots
\end{proof}
\begin{proof}[Preuve du théorème \ref{montheo}]
	En appliquant le lemme \ref{monlemme}, \ldots
	
	\ldots
	
	et finalement le théorème est démontré.
\end{proof}
\begin{cor}
	Une conséquence du théorème \ref{montheo}.
\end{cor}
\begin{rem*}
	Ceci résout notre problème.
\end{rem*}


\section{Conclusion}


%%%%%%%%%%%%%%
%%% Pour le comité d'édition, dé-commenter puis ajouter pour chacune des notes
%%% \note{X} et le texte de la note.
%%% le label X doit correspondre à celui de l'appel de note.
%\vspace{2ex}
% \section*{Notes d'édition}
% \addcontentsline{toc}{section}{Notes d'édition}
% \noindent\fcolorbox[gray]{0}{0.95}{
% \begin{minipage}{.975\textwidth}
%
%%% \note{X} \ldots
%
% \end{minipage}
% }
 
\end{document}
